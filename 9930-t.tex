\documentclass[oneside]{article}
\usepackage{enumerate}
\usepackage[leqno]{amsmath}
\allowdisplaybreaks[1]
\begin{document}

\thispagestyle{empty}
\small
\begin{verbatim}

The Project Gutenberg EBook of Groups of Order p^m Which Contain Cyclic
Subgroups of Order p^(m-3), by Lewis Irving Neikirk

Copyright laws are changing all over the world. Be sure to check the
copyright laws for your country before downloading or redistributing
this or any other Project Gutenberg eBook.

This header should be the first thing seen when viewing this Project
Gutenberg file.  Please do not remove it.  Do not change or edit the
header without written permission.

Please read the "legal small print," and other information about the
eBook and Project Gutenberg at the bottom of this file.  Included is
important information about your specific rights and restrictions in
how the file may be used.  You can also find out about how to make a
donation to Project Gutenberg, and how to get involved.


**Welcome To The World of Free Plain Vanilla Electronic Texts**

**eBooks Readable By Both Humans and By Computers, Since 1971**

*****These eBooks Were Prepared By Thousands of Volunteers!*****


Title: Groups of Order p^m Which Contain Cyclic Subgroups of Order p^(m-3)

Author: Lewis Irving Neikirk

Release Date: February, 2006 [EBook #9930]
[Yes, we are more than one year ahead of schedule]
[This file was first posted on November 1, 2003]

Edition: 10

Language: English

Character set encoding: US-ASCII

*** START OF THE PROJECT GUTENBERG EBOOK GROUPS OF ORDER P^M ***




Produced by Cornell University, Joshua Hutchinson, Lee Chew-Hung,
John Hagerson, and the Online Distributed Proofreading Team.

\end{verbatim}
\normalsize
\newpage



\begin{center}
\noindent \Large GROUPS OF ORDER $p^m$ WHICH CONTAIN CYCLIC
SUBGROUPS OF ORDER $p^{m-3}$

\bigskip
\normalsize\textsc{by}
\bigskip

\large LEWIS IRVING NEIKIRK

\footnotesize\textsc{sometime harrison research fellow in
mathematics}

\bigskip
\large 1905
\end{center}

\newpage
\begin{center}
\large \textbf{INTRODUCTORY NOTE.}
\end{center}
\normalsize

This monograph was begun in 1902-3. Class I, Class II, Part I, and
the self-conjugate groups of Class III, which contain all the groups with
independent generators, formed the thesis which I presented to the Faculty
of Philosophy of the University of Pennsylvania in June, 1903, in partial
fulfillment of the requirements for the degree of Doctor of Philosophy.

The entire paper was rewritten and the other groups added while the
author was Research Fellow in Mathematics at the University.

I wish to express here my appreciation of the opportunity for scientific
research afforded by the Fellowships on the George Leib Harrison Foundation
at the University of Pennsylvania.

I also wish to express my gratitude to Professor George H.\
Hallett for his kind assistance and advice in the preparation of
this paper, and especially to express my indebtedness to Professor
Edwin S.\ Crawley for his support and encouragement, without which
this paper would have been impossible.

\begin{flushright}
\textsc{Lewis I.\ Neikirk.}
\end{flushright}

\footnotesize \textsc{ University Of Pennsylvania,} \textit{May,
1905.} \normalsize

\newpage

\begin{center}
\large GROUPS OF ORDER $p^m$, WHICH CONTAIN CYCLIC SUBGROUPS OF
ORDER $p^{(m-3)}$\footnote{Presented to the American Mathematical
Society April 25, 1903.}

\bigskip \normalsize \textsc{by}

\bigskip \textsc{lewis irving neikirk}

\bigskip\textit{Introduction.}
\end{center}

The groups of order $p^m$, which contain self-conjugate cyclic
subgroups of orders $p^{m-1}$, and $p^{m-2}$ respectively, have
been determined by \textsc{Burnside},\footnote{\textit{Theory of
Groups of a Finite Order}, pp.\ 75-81.} and the number of groups of
order $p^m$, which contain cyclic non-self-conjugate subgroups of
order $p^{m-2}$ has been given by
\textsc{Miller}.\footnote{Transactions, vol.\ 2 (1901), p.\ 259, and
vol.\ 3 (1902), p.\ 383.}

Although in the present state of the theory, the actual tabulation
of all groups of order $p^m$ is impracticable, it is of importance
to carry the tabulation as far as may be possible. In this paper
\textit{all groups of order} $p^m$ ($p$ being an odd prime)
\textit{which contain cyclic subgroups of order $p^{m-3}$ and none
of higher order} are determined. The method of treatment used is
entirely abstract in character and, in virtue of its nature, it is
possible in each case to give explicitly the generational
equations of these groups. They are divided into three classes,
and it will be shown that these classes correspond to the three
partitions: $(m-3,\, 3)$, $(m-3,\, 2,\, 1)$ and $(m-3,\, 1,\, 1,\, 1)$, of
$m$.

We denote by $G$ an abstract group $G$ of order $p^m$ containing
operators of order $p^{m-3}$ and no operator of order greater than
$p^{m-3}$. Let $P$ denote one of these operators of $G$ of order
$p^{m-3}$. The $p^3$ power of every operator in $G$ is contained
in the cyclic subgroup $\{P\}$, otherwise $G$ would be of order
greater than $p^m$. The complete division into classes is effected
by the following assumptions:
\begin{enumerate}[I.]
\item There is in $G$ at least one operator $Q_1$, such that
$Q{}_1^{p^2}$ is not contained in $\{P\}$.
\item The $p^2$ power of every operator in $G$ is contained in
$\{P\}$, and there is at least one operator $Q_1$, such that
$Q{}_1^p$ is not contained in $\{P\}$.
\item The $p$th power of every operator in $G$ is
contained in $\{P\}$.
\end{enumerate}

\newpage
The number of groups for Class I, Class II, and Class III,
together with the total number, are given in the table below:
\bigskip

\begin{tabular}{|c|c|c|c|c|c|c|c|}
\hline
 & I & II$_1$ & II$_2$ & II$_3$ & II & III & Total \\ \hline
$p>3$ & & & & & & & \\
$m>8$ & 9 & $20+p$ & $6+2p$ & $6+2p$ & $32+5p$ & 23 & $64+5p$ \\ \hline
$p>3$ & & & & & & & \\
$m=8$ & 8 & $20+p$ & $6+2p$ & $6+2p$ & $32+5p$ & 23 & $63+5p$ \\ \hline
$p>3$ & & & & & & & \\
$m=7$ & 6 & $20+p$ & $6+2p$ & $6+2p$ & $32+5p$ & 23 & $61+5p$ \\ \hline
$p=3$ & & & & & & & \\
$m>8$ & 9 & 23 & 12 & 12 & 47 & 16 & 72 \\ \hline
$p=3$ & & & & & & & \\
$m=8$ & 8 & 23 & 12 & 12 & 47 & 16 & 71 \\ \hline
$p=3$ & & & & & & & \\
$m=7$ & 6 & 23 & 12 & 12 & 47 & 16 & 69 \\ \hline
\end{tabular}

\bigskip \bigskip
\begin{center}
\Large\textit{Class} I.\normalsize
\end{center}

1. \textit{General notations and relations.}---The group $G$ is
generated by the two operators $P$ and $Q_1$. For brevity we
set\footnote{With J.~W.\ \textsc{Young}, \textit{On a certain
group of isomorphisms}, American Journal of Mathematics, vol.\ 25
(1903), p.\ 206.}
\begin{equation*}
Q{}_1^a\, P^b\, Q{}_1^c\, P^d \cdots = [a,\, b,\, c,\, d,\, \cdots].
\end{equation*}

Then the operators of $G$ are given each uniquely in the form
\begin{equation*}
[y,\, x] \quad \left( \begin{aligned}y &= 0,\, 1,\, 2,\, \cdots,\, p^3-1 \\
                                     x &= 0,\, 1,\, 2,\, \cdots,\, p^{m-3}-1
                      \end{aligned} \right) .
\end{equation*}

We have the relation
\begin{equation}
Q{}_1^{p^3} = P^{hp^3}. %% 1
\end{equation}

\noindent There is in $G$, a subgroup $H_1$ of order $p^{m-2}$, which
contains $\{P\}$ self-con\-ju\-gate\-ly.\footnote{\textsc{Burnside}:
\textit{Theory of Groups}, Art.\ 54, p.\ 64.} The subgroup $H_1$ is
generated by $P$ and some operator $Q{}_1^y P^x$ of $G$; it then
contains $Q{}_1^y$ and is therefore generated by $P$ and
$Q{}_1^{p^2}$; it is also self-conjugate in $H_2 = \{Q{}_1^p, P\}$
of order $p^{m-1}$, and $H_2$ is self-conjugate in $G$.

From these considerations we have the
equations\footnote{\textit{Ibid.}, Art.\ 56, p.\ 66.}
\begin{align}
Q{}_1^{-p^2}\,P\,Q{}_1^{p^2} &= P^{1+kp^{m-4}}, \\ %% 2
Q{}_1^{-p}\,P\,Q{}_1^p &= Q{}_1^{\beta p^2}\,P^{\alpha_1}, \\ %% 3
Q{}_1^{-1}\,P\,Q_1 &= Q{}_1^{bp}\,P^{a_1}. %% 4
\end{align}

\medskip
2. \textit{Determination of $H_1$. Derivation of a formula for
$[yp^2, x]^s$.}---From (2), by repeated multiplication we obtain
\begin{gather*}
[-p^2,\, x,\, p^2] = [0,\, x(1 + kp^{m-4})]; \\
\intertext{and by a continued use of this equation we have}
[-yp^2,\, x,\, yp^2] = [0,\, x(1 + kp^{m-4})^y] = [0,\, x(1 + kyp^{m-4})] \qquad (m > 4)
\end{gather*}
\noindent and from this last equation,
\begin{equation}
[yp^2,\, x]^s = \bigl[syp^2,\, x\{s + k \tbinom{s}{2}yp^{m-4}\}\bigr]. %% 5
\end{equation}

\medskip
3. \textit{Determination of $H_2$. Derivation of a formula for
$[yp,\, x]^s$.}---It follows from (3) and (5) that
\begin{equation*}
[-p^2,\, 1,\, p^2] = \left[\beta\frac{\alpha_1^p-1}{\alpha_1-1}p^2,\,
  \alpha_1^p\left \{ 1+\frac{\beta k}{2}
  \frac{\alpha_1^p-1}{\alpha_1-1} p^{m-4}\right \} \right] \quad (m > 4).
\end{equation*}
\noindent Hence, by (2),
\begin{gather*}
\beta\frac{\alpha_1^p - 1}{\alpha_1 - 1}p^2 \equiv 0 \pmod{p^3}, \\
\alpha{}_1^p \left \{ 1 + \frac{\beta k}{2}
  \frac{\alpha{}_1^p-1}{\alpha_1 - 1} p^{m-4} \right \} +
  \beta\frac{\alpha{}_1^p - 1}{\alpha_1 - 1}hp^2
  \equiv 1 + kp^{m-4} \pmod{p^{m-3}}. \\
\intertext{From these congruences, we have for $m > 6$}
\alpha{}_1^p \equiv 1 \pmod{p^3}, \qquad \alpha_1 \equiv 1 \pmod{p^2}, \\
\intertext{and obtain, by setting}
\alpha_1 = 1 + \alpha_2 p^2, \\
\intertext{the congruence}
\frac{(1 + \alpha_2 p^2)^p - 1}{\alpha_2 p^3}(\alpha_2 + h\beta)p^3
  \equiv kp^{m-4} \pmod{p^{m-3}}; \\
\intertext{and so}
(\alpha_2 + h\beta)p^3 \equiv 0 \pmod{p^{m-4}}, \\
\intertext{since}
\frac{(1+\alpha_2 p^2)^p-1}{\alpha_2 p^3} \equiv 1 \pmod{p^2}.
\end{gather*}
\noindent From the last congruences
\begin{gather}
(\alpha_2 + h\beta)p^3 \equiv kp^{m-4} \pmod{p^{m-3}}. \\ %% 6
\intertext{Equation (3) is now replaced by}
Q{}_1^{-p}\,P\, Q{}_1^{-p} = Q{}_1^{\beta p^2} P^{1 + \alpha_2 p^2}. %% 7
\end{gather}
\noindent From (7), (5), and (6)
\begin{equation*}
[-yp,\, x,\, yp] = \left[\beta xyp^2,\, x\{1 + \alpha_2 yp^2\}
  + \beta k \tbinom{x}{2}yp^{m-4}\right].
\end{equation*}
\noindent A continued use of this equation gives
\begin{multline}
[yp,\, x]^s = [syp + \beta \tbinom{s}{2}xyp^2, \\
  xs + \tbinom{s}{2} \{\alpha_2 xyp^2 + \beta k\tbinom{x}{2}yp^{m-4}\} +
  \beta k\tbinom{s}{3}x^2yp^{m-4}]. %% 8
\end{multline}

\medskip
4. \textit{Determination of $G$.}---From (4) and (8),
\begin{gather*}
[-p,\, 1,\, p] = [Np,\, a{}_1^p + Mp^2]. \\
\intertext{From the above equation and (7),}
a{}_1^p \equiv 1 \pmod{p^2}, \qquad a_1 \equiv 1 \pmod{p}.
\end{gather*}

Set $a_1 = 1 + a_2 p$ and equation (4) becomes
\begin{equation}
Q{}_1^{-1}\, P\, Q_1 = Q{}_1^{bp}\, P^{1 + a_2 p}. %% 9
\end{equation}
\noindent From (9), (8) and (6)
\begin{gather*}
[-p^2,\, 1,\, p^2] = \left[\frac{(1 + a_2 p)^{p^2}-1}{a_2 p}bp,
  (1 + a_2 p)^{p^2}\right], \\
\intertext{and from (1) and (2)}
\frac{(1 + a_2 p)^{p^2} - 1}{a_2 p}bp \equiv 0 \pmod{p^3}, \\
(1 + a_2 p)^{p^2} + bh\frac{(1 + a_2 p)^{p^2} - 1}{a_2 p}p
  \equiv 1 + kp^{m-4} \pmod{p^{m-3}}.
\end{gather*}
\noindent By a reduction similar to that used before,
\begin{equation}
(a_2 + bh)p^3 \equiv kp^{m-4} \pmod{p^{m-3}}. %% 10
\end{equation}

The groups in this class are completely defined by (9), (1) and (10).

These defining relations may be presented in simpler form by a suitable
choice of the second generator $Q_1$. From (9), (6), (8) and (10)
\begin{gather*}
[1,\, x]^{p^3} = [p^3,\, xp^3] = [0,\, (x + h)p^3] \quad (m > 6), \\
\intertext{and, if $x$ be so chosen that}
x + h \equiv 0 \pmod{p^{m-6}}, \\
\intertext{$Q_1\, P^x$ is an operator of order $p^3$ whose $p^2$
power is not contained in $\{P\}$. Let $Q_1\, P^x = Q$. The group
$G$ is generated by $Q$ and $P$, where}
Q^{p^3} = 1, \quad P^{p^{m-3}} = 1. \\
\intertext{Placing $h = 0$ in (6) and (10) we find}
\alpha_2 p^3 \equiv a_2 p^3 \equiv k p^{m-4} \pmod{p^{m-3}}.
\end{gather*}
\noindent Let $\alpha_2 = \alpha p^{m-7}$, and $a_2 = ap^{m-7}$.
Equations (7) and (9) are now replaced by
\begin{equation}
\left.
  \begin{aligned}
  Q^{-p}\, P\, Q^p &= Q^{\beta p^2} P^{1 + \alpha p^{m-5}},\\
  Q^{-1}\, P\, Q   &= Q^{bp} P^{1 + ap^{m-6}}.
  \end{aligned}
\right. %% 11
\end{equation}

As a direct result of the foregoing relations, the groups in this
class correspond to the partition $(m-3,\, 3)$. From (11) we
find\footnote{For $m = 8$ it is necessary to add
$a^2\binom{y}{2}p^4$ to the exponent of $P$ and for $m = 7$ the
terms $a(a + \frac{abp}{2})\binom{y}{2}p^2 + a^3\binom{y}{3}p^3$
to the exponent of $P$, and the term $ab\binom{y}{2}p^2$ to the
exponent of $Q$. The extra term $27ab^2 k\binom{y}{3}$ is to be
added to the exponent of $P$ for $m = 7$ and $p = 3$.}
\begin{equation*}
[-y,\, 1,\, y] = [byp,\, 1 + ayp^{m-6}] \qquad (m > 8).
\end{equation*}

It is important to notice that by placing $y = p$ and $p^2$ in the
preceding equation we find that\footnote{For $m = 7,\,
ap^2-\frac{a^2p^3}{2} \equiv ap^2 \pmod{p^4},\, ap^3 \equiv kp^3
\pmod{p^4}$. For $m = 7$ and $p = 3$ the first of the above
congruences has the extra terms $27(a^3 + ab\beta k)$ on the left
side.}
\begin{equation*}
b \equiv \beta \pmod{p}, \qquad a \equiv \alpha \equiv k \pmod{p^3}
  \qquad (m > 7).
\end{equation*}

A combination of the last equation with (8) yields\footnote{For $m
= 8$ it is necessary to add the term $a\binom{y}{2}xp^4$ to the
exponent of $P$, and for $m = 7$ the terms $x\{a(a +
\frac{abp}{2})\binom{y}{2}p^2 + a^3\binom{y}{3}p^3\}$ to the
exponent of $P$, with the extra term $27ab^2 k\binom{y}{3}x$ for
$p = 3$, and the term $ab\binom{y}{2}xp^2$ to the exponent of
$Q$.}
\begin{multline}
[-y,\, x,\, y] = [bxyp + b^2\tbinom{x}{2}yp^2, \\
  x(1 + ayp^{m-6}) + ab\tbinom{x}{2}yp^{m-5} +
  ab^2\tbinom{x}{3}yp^{m-4}] \qquad (m > 8). %% 12
\end{multline}

\newpage
From (12) we get\footnote{For $m = 8$ it is necessary to add the
term $\frac{1}{2} axy \binom{s}{2}[\frac{1}{3}y(2s - 1) - 1]p^4$
to the exponent of $P$, and for $m=7$ the terms
\begin{multline*}
  x \Bigl\{ \frac{a}{2} \bigl( a + \frac{ab}{2} p \bigr)
    \bigl(\frac{2s-1}{3} y - 1 \bigr) \tbinom{s}{2}yp^2 +
    \frac{a^{3}}{3!} \bigl(\tbinom{s}{2}y^2 - (2s - 1)y + 2 \bigr) yp^3 \\
  + \frac{a^2 bxy^2}{2} \tbinom{s}{3} \frac{3s-1}{2}p^3 + \frac{a^2 b}{2}
    \bigl( \frac{s(s - 1)^2 (s - 4)}{4!}y - \tbinom{s}{3} \bigr) yp^3 \Bigr\}
\end{multline*}
\noindent with the extra terms
\begin{equation*}
  27abxy \Bigl\{ \frac{bk}{3!}\bigl[\tbinom{s}{2}y^2 - (2s - 1)y + 2\bigr] \tbinom{s}{3}
    + x(b^2 k + a^2)(2y^2 + 1)\tbinom{s}{3} \Bigr\},
\end{equation*}
\noindent for $p=3$, to the exponent of $P$, and the terms
$\frac{ab}{2} \bigl\{ 2s - \frac{1}{3}y - 1 \bigr\} \tbinom{s}{2}xyp^2$
to the exponent of $Q$.} %% END OF FOOTNOTE
\begin{multline}
[y,\, x]^s = \bigl[ys + by\bigl\{(x +b\tbinom{x}{2}p)\tbinom{s}{2} + x\tbinom{s}{3}p\bigr\}p, \\
  xs + ay\bigl\{(x+b\tbinom{x}{2}p + b^2\tbinom{x}{3}p^2)\tbinom{s}{2} \\
  + (bx^2 p + 2b^2 x\tbinom{x}{2}p^2)\tbinom{s}{3} + bx^2\tbinom{s}{4}p^2\bigr\}p^{m-6}\bigr]
  \qquad (m > 8). %% 13
\end{multline}

\medskip
5. \textit{Transformation of the Groups.}---The general group $G$
of Class I is specified, in accordance with the relations (2) (11)
by two integers $a$, $b$ which (see (11)) are to be taken mod
$p^3$, mod $p^2$, respectively. Accordingly setting
\begin{gather*}
a = a_1 p^\lambda, \quad b = b_1 p^\mu,
\intertext{where}
dv[a_1,\, p] = 1, \quad dv[b_1,\, p] = 1 \qquad (\lambda = 0,\, 1,\, 2,\, 3;\; \mu = 0,\, 1,\, 2),
\intertext{we have for the group $G = G(a,\, b) = G(a,\, b)(P,\, Q)$ the
generational determination:}
G(a,\, b):\; \left \{
  \begin{gathered}
    Q^{-1}\, P\, Q = Q^{b_1 p^{\mu + 1}} P^{1 + a_1 p^{m + \lambda - 6}} \\
    Q^{p^3} = 1, \quad P^{p^{m-3}} = 1.
  \end{gathered} \right.
\end{gather*}

Not all of these groups however are distinct. Suppose that
\begin{gather*}
G(a,\, b)(P,\, Q) \sim G(a',\, b')(P',\, Q'),
\intertext{by the correspondence}
C = \left[\begin{array}{cc}
                          Q,    & P \\
                          Q'_1, & P'_1 \\
          \end{array} \right],
\intertext{where}
Q'_1 = Q'^{y'} P'^{x'p^{m-6}}, \qquad \hbox{ and } \qquad P'_1 = Q'^y P'^x,
\end{gather*}
\noindent with $y'$ and $x$ prime to $p$.

Since
\begin{gather*}
Q^{-1}\, P\, Q = Q^{bp} P^{1 + ap^{m-6}}, \\
\intertext{then}
{Q'}_1^{-1}\, P'_1\, Q'_1 = {Q'}_1^{bp} {P'}_1^{1 + ap^{m-6}}, \\
\intertext{or in terms of $Q'$, and $P'$}
\begin{aligned}
  \bigl[y + b'xy'p &+ b'^2\tbinom{x}{2}y'p^2, x(1 + a'y'p^{m-6}) + a'b'\tbinom{x}{2}y'p^{m-5} \\
    &+ a'b'^2\tbinom{x}{3}y'p^{m-4}\bigr] = [y + by'p, x + (ax + bx'p)p^{m-6}] \qquad (m > 8)
\end{aligned}
\end{gather*}
\noindent and
\begin{gather}
by' \equiv b'xy' + b'^2\tbinom{x}{2}y'p \pmod{p^2}, \\ %% 14
ax + bx'p \equiv a'y'x + a'b'\tbinom{x}{2}y'p + a'b'^2\tbinom{x}{3}y'p^2 \pmod{p^3}. %% 15
\end{gather}
\noindent The necessary and sufficient condition for the simple
isomorphism of these two groups $G(a,\, b)$ and $G(a',\, b')$ is, that
the above congruences shall be consistent and admit of solution
for $x$, $y$, $x'$ and $y'$. The congruences may be written
\begin{gather*}
b_1 p^\mu \equiv b'_1 xp^{\mu'} + {b'}_1^2\tbinom{x}{2}p^{2\mu' + 1} \pmod{p^2}, \\
\begin{aligned}
  a_1 xp^{\lambda} + b_1 x'p^{\mu + 1} &\equiv \\
    y'\{a'_1 xp^{\lambda'} &+ a'_1 b'_1\tbinom{x}{2}p^{\lambda'+\mu'+1}
    + a'_1 {b'}_1^2\tbinom{x}{3}p^{\lambda'+2\mu'+2}\} \pmod{p^3}.
\end{aligned}
\end{gather*}
\noindent Since $dv[x,\, p] = 1$ the first congruence gives $\mu =
\mu'$ and $x$ may always be so chosen that $b_1 = 1$.

We may choose $y'$ in the second congruence so that $\lambda =
\lambda'$ and $a_1 = 1$ except for the cases $\lambda' \ge \mu + 1
= \mu' + 1$ when we will so choose $x'$ that $\lambda = 3$.

The type groups of Class I for $m > 8$\footnote{For $m = 8$ the
additional term $ayp$ appears on the left side of the congruence
(14) and $G(1,\, p^2)$ and $G(1,\, p)$ become simply isomorphic. The
extra terms appearing in congruence (15) do not effect the result.
For $m = 7$ the additional term $ay$ appears on the left side of
(14) and $G(1,\, 1)$, $G(1,\, p)$, and $G(l,\, p^2)$ become simply
isomorphic, also $G(p,\, p)$ and $G(p,\, p^2)$.} are then given by
\begin{multline}
G(p^\lambda,\, p^\mu):\; Q^{-1}\, P\, Q = Q^{p^{1+\mu}}
  P^{1+p^{m-6+\lambda}},\, Q^{p^3} = 1,\, P^{p^{m-3}} = 1 \\
\left(
  \begin{aligned}
    \mu = 0,\, 1,\, 2;\;& \lambda = 0,\, 1,\, 2;\; \lambda \ge \mu; \\
    \mu = 0,\, 1,\, 2;\;& \lambda = 3
  \end{aligned} \right)
\tag{I}.
\end{multline}

Of the above groups $G(p^\lambda,\, p^\mu)$ the groups for $\mu = 2$ have
the cyclic subgroup $\{P\}$ self-conjugate, while the group $G(p^3,\, p^2)$
is the abelian group of type $(\mbox{$m-3$},\, 3)$.

\bigskip \bigskip
\begin{center}
\Large\textit{Class} II. \normalsize
\end{center}
\setcounter{equation}{0}
1. \textit{General relations.}

There is in $G$ an operator $Q_1$ such that $Q{}_1^{p^2}$ is contained in
$\{P\}$ while $Q{}_1^p$ is not.
\begin{equation}
Q{}_1^{p^2} = P^{hp^2}. %% 1
\end{equation}

The operators $Q_1$ and $P$ either generate a subgroup $H_2$ of order
$p^{m-1}$, or the entire group $G$.

\bigskip
\begin{center}
\large\textit{Section} 1. \normalsize
\end{center}

2. \textit{Groups with independent generators.}

Consider the first possibility in the above paragraph. There is in
$H_2$, a subgroup $H_1$ of order $p^{m-2}$, which contains $\{P\}$
self-conjugately.\footnote{\textsc{Burnside}, \textit{Theory of
Groups}, Art.\ 54, p.\ 64.} $H_1$ is generated by $Q{}_1^p$ and $P$.
$H_2$ contains $H_1$ self-conjugately and is itself self-conjugate
in $G$.

From these considerations\footnote{\textit{Ibid.}, Art.\ 56, p.\ 66.}
\begin{align}
Q{}_1^{-p}\, P\, Q{}_1^p &= P^{1 + kp^{m-4}}, \\ %% 2
Q{}_1^{-1}\, P\, Q &= Q{}_1^{\beta p} P^{\alpha_1}. %% 3
\end{align}

\medskip
3. \textit{Determination of $H_1$ and $H_2$.}

From (2) we obtain
\begin{equation}
[yp,\, x]^s = \bigl[syp,\, x\bigl\{s + k\tbinom{s}{2}yp^{m-4}\bigr\}\bigr] \quad (m > 4), %% 4
\end{equation}
\noindent and from (3) and (4)
\begin{equation*}
[-p,\, 1,\, p] = \left[\frac{\alpha{}_1^p - 1}{\alpha_1 - 1}\beta p,\,
  \alpha{}_1^p\left\{1 + \frac{\beta k}{2}
  \frac{\alpha{}_1^p-1}{\alpha_1-1} p^{m-4} \right\} \right].
\end{equation*}

A comparison of the above equation with (2) shows that
\begin{gather*}
\frac{\alpha{}_1^p - 1}{\alpha_1 - 1} \beta p \equiv 0 \pmod{p^2}, \\
\alpha{}_1^p \left\{1 + \frac{\beta k}{2}\frac{\alpha{}_1^p-1}{a_1-1}
  p^{m-4} \right\} + \frac{\alpha{}_1^p - 1}{\alpha_1 - 1} \beta hp
  \equiv 1 + kp^{m-4} \pmod{p^{m-3}},
\end{gather*}
\noindent and in turn
\begin{equation*}
\alpha{}_1^p \equiv 1 \pmod{p^2}, \qquad \alpha_1 \equiv 1 \pmod{p} \qquad (m > 5).
\end{equation*}

Placing $\alpha_1 = 1 + \alpha_2 p$ in the second congruence, we obtain
as in Class I
\begin{equation}
(\alpha_2 + \beta h)p^2 \equiv kp^{m-4} \pmod{p^{m-3}} \qquad (m > 5). %% 5
\end{equation}

Equation (3) now becomes
\begin{equation}
Q{}_1^{-1}\, P\, Q_1 = Q^\beta P^{1 + \alpha_2 p}. %% 6
\end{equation}
\noindent The generational equations of $H_2$ will be simplified
by using an operator of order $p^2$ in place of $Q_1$.

From (5), (6) and (4)
\begin{gather*}
[y,\, x]^s = [sy + U_s p,\, sx + W_s p]
\intertext{in which}
\begin{aligned}
U_s &= \beta \tbinom{s}{2}xy, \\
W_s &= \alpha_2 \tbinom{s}{2}xy + \Bigl\{ \beta k \bigl[\tbinom{s}{2}\tbinom{x}{2}
     + \tbinom{s}{3}x^2 y\bigr] \\
 & \qquad \qquad \qquad + \frac{1}{2}\alpha k\bigl[\frac{1}{3!}s(s - 1)(2s - 1)y^2
     - \tbinom{s}{2}y\bigr]x \Bigr\} p^{m-5}.
\end{aligned}
\end{gather*}

Placing $s = p^2$ and $y = 1$ in the above
\begin{gather*}
[Q_1\, P^x]^{p^2} = Q{}_1^{p^2} P^{xp^2} = P^{(x+h)p^2}.
\intertext{If $x$ be so chosen that}
(x + h) \equiv 0 \pmod{p^{m-5}} \qquad (m > 5)
\end{gather*}
\noindent $Q_1 P^x$ will be the required $Q$ of order $p^2$.

Placing $h = 0$ in congruence (5) we find
\begin{equation*}
\alpha_2 p^2 \equiv kp^{m-4} \pmod{p^{m-3}}.
\end{equation*}

Let $\alpha_2 = \alpha p^{m-6}$. $H_2$ is then generated by
\begin{equation*}
Q^{p^2} = 1, \quad P^{p^{m-3}} = 1.
\end{equation*}
\begin{equation}
Q^{-1}\, P\, Q = Q^{\beta p} P^{1 + \alpha p^{m-5}}. %% 7
\end{equation}

Two of the preceding formul\ae\ now become
\begin{gather}
[-y,\, x,\, y] = \bigl[\beta xyp,\, x(1 + \alpha yp^{m-5}) + \beta k\tbinom{x}{2}yp^{m-4}\bigr], \\ %% 8
[y,\, x]^s = [sy + U_s p,\, xs + W_s p^{m-5}], %% 9
\end{gather}
\noindent where
\begin{equation*}
U_s = \beta \tbinom{s}{2}xy
\end{equation*}
\noindent and\footnote{For $m = 6$ it is necessary to add the terms
$\frac{ak}{2} \left \{ \frac{s(s - 1)(2s - 1)}{3!}y^2 - \tbinom{s}{2}y \right \}p$
to $W_s$.}
\begin{equation*}
W_s = \alpha \tbinom{s}{2}xy + \beta k\bigl\{\tbinom{s}{2}\tbinom{x}{2}
  + \tbinom{s}{3}x^2\bigr\}yp \quad (m > 6).
\end{equation*}

\medskip
4. \textit{Determination of $G$.}

Let $R_1$ be an operation of $G$ not in $H_2$. $R{}_1^p$ is in $H_2$. Let
\begin{equation}
R{}_1^p = Q^{\lambda p} P^{\mu p}. %% 10
\end{equation} %% 10

Denoting $R{}_1^a\, Q^b\, P^c\, R{}_1^d\, Q^e\, P^f \cdots$ by the symbol $[a,\, b,\, c,\,
d,\, e,\, f,\, \cdots]$, all the operations of $G$ are contained in the set $[z,\,
y,\, x]$; $z = 0,\, 1,\, 2,\, \cdots,\, p - 1$; $y = 0,\, 1,\, 2,\, \cdots,\, p^2 - 1$; $x = 0,\,
1,\, 2,\, \cdots,\, p^{m-3} - 1$.

The subgroup $H_2$ is self-conjugate in $G$. From
this\footnote{\textsc{Burnside}, \textit{Theory of Groups}, Art.\
24, p.\ 27.}
\begin{gather}
R{}_1^{-1}\, P\, R_1 = Q^{b_1} P^{a_1}, \\ %% 11
R{}_1^{-1}\, Q\, R_1 = Q^{d_1} P^{c_1 p^{m-5}}. %% 12
\end{gather}
\noindent In order to ascertain the forms of the constants in (11)
and (12) we obtain from (12), (11), and (9)
\begin{gather*}
[-p,\, 1,\, 0,\, p] = [0,\, d{}_1^p + Mp,\, Np^{m-5}].
\intertext{By (10) and (8)}
R{}_1^p\, Q\, R{}_1^p = P^{-\mu p}\, Q\, P^{\mu p} = Q\, P^{-a\mu p^{m-4}}.
\intertext{From these equations we obtain}
d{}_1^p \equiv 1 \pmod p \quad \hbox{ and } \quad d_1 \equiv 1 \pmod p .
\end{gather*}
\noindent Let $d_1 = 1 + dp$. Equation (12) is replaced by
\begin{equation}
R{}_1^{-1}\, Q\, R_1 = Q^{1+dp} P^{e_1 p^{m-5}}. %% 13
\end{equation}
\noindent From (11), (13) and (9)
\begin{gather*}
[-p,\, 0,\, 1,\, p] = \left[\frac{a{}_1^p - 1}{a_1 - 1}b_1 + Kp,\, a{}_1^p + b_1 Lp^{m-5}\right]
\intertext{in which}
K = a_1 b_1 \beta \sum_1^{p-1}\tbinom{a{}_1^y}{2}.
\intertext{By (10) and (8)}
R{}_1^{-p}\, P\, R{}_1^p = Q^{-\lambda p} P\, Q^{\lambda p} = P^{1 + a \lambda p^{m-4}},
\intertext{and from the last two equations}
a{}_1^p \equiv 1 \pmod{p^{m-5}}
\intertext{and}
a_1 \equiv 1 \pmod{p^{m-6}} \quad (m > 6); \qquad a_1 \equiv 1 \pmod{p} \quad (m = 6).
\end{gather*}

Placing $a_1 = 1 + a_2 p^{m-6} \quad (m > 6)$; \qquad $a_1 = 1 + a_2 p \quad (m=6)$.
\begin{equation*}
K \equiv 0 \pmod{p},
\end{equation*}
\noindent and\footnote{$K$ has an extra term for $m = 6$ and $p =
3$, which reduces to $3b_1 c_1$. This does not affect the
reasoning except for $c_1 = 2$. In this case change $P^2$ to $P$
and $c_1$ becomes $1$.}
\begin{gather*}
\frac{a{}_1^p - 1}{a_1 - 1}b_1 \equiv b_1 p \equiv 0 \pmod{p^2},
  \qquad b_1 \equiv 0 \pmod p.
\intertext{Let $b_1 = bp$ and we find}
a{}_1^p \equiv 1 \pmod{p^{m-4}}, \qquad a_1 \equiv 1 \pmod{p^{m-5}}.
\end{gather*}

Let $a_1 = 1 + a_3 p^{m-5}$ and equation (11) is replaced by
\begin{equation}
R{}_1^{-1}\, P\, R_1 = Q^{bp} P^{1 + a_3 p^{m-5}}. %% 14
\end{equation}
\noindent The preceding relations will be simplified by taking for
$R_1$ an operator of order $p$. This will be effected by two
transformations.

From (14), (9) and (13)\footnote{The extra terms appearing in the
exponent of $P$ for $m=6$ do not alter the result.}
\begin{gather*}
[1,\, y]^p = \Bigl[p,\, yp,\, \frac{-c_1 y}{2} p^{m-4}\Bigr]
  = \Bigl[0,\, (\lambda + y)p,\, \mu p - \frac{c_1 y}{2} p^{m-4}\Bigr],
\intertext{and if $y$ be so chosen that}
\lambda + y \equiv 0 \pmod{p},
\end{gather*}
\noindent $R_2 = R_1\, Q^y$ is an operator such that $R{}_2^p$ is in
$\{P\}$.

Let
\begin{gather*}
R{}_2^p = P^{lp}.
\intertext{Using $R_2$ in the place of $R_1$, from (15), (9) and (14)}
[1,\, 0,\, x]^p = \Bigl[p,\, 0,\, xp + \frac{ax}{2} p^{m-4}\Bigr] =
  \Bigl[0,\, 0,\, (x + l)p + \frac{ax}{2} p^{m-4}\Bigr],
\intertext{and if $x$ be so chosen that}
x + l + \frac{ax}{2} p^{m-5} \equiv 0 \pmod{p^{m-4}},
\end{gather*}
\noindent then $R = R_2 P^x$ is the required operator of order $p$.

$R^p = 1$ is permutable with both $Q$ and $P$. Preceding equations now
assume the final forms
\begin{align}
Q^{-1}\, P\, Q & = Q^{\beta p} P^{1 + ap^{m-5}}, \\ %% 15
R^{-1}\, P\, R & = Q^{bp} P^{1 + ap^{m-4}}, \\ %% 16
R^{-1}\, Q\, R & = Q^{1 + dp} P^{cp^{m-4}},         %% 17
\end{align}
with $R^p = 1$, $Q^{p^2} = 1$, $P^{p^{m-3}} = 1$.

The following derived equations are necessary\footnote{For $m=6$
the term $a^2 \tbinom{x}{2} xp^2$ must be added to the exponent of
$P$ in (18).}
\begin{align}
[0,\, -y,\, x,\, 0,\, y] &= \bigl[0,\, \beta xyp,\, x(1 + \alpha yp^{m-5}) + \alpha
  \beta \tbinom{x}{2}yp^{m-4}\bigr], \\ %% 18
[-y,\, 0,\, x,\, -y] &= \bigl[0,\, bxyp,\, x(1 + ayp^{m-4})
  + ab\tbinom{x}{2} yp^{m-4}\bigr], \\ %% 19
[-y,\, x,\, 0,\, y] &= [0,\, x(1 + dyp),\, cxyp^{m-4}]. %% 20
\end{align}

From a consideration of (18), (19) and (20) we arrive at the
expression for a power of a general operator of $G$.
\begin{equation}
[z,\, y,\, x]^s = [sz,\, sy + U_s p,\, sx + V_s p^{m-5}], %% 21
\end{equation}
\noindent where\footnote{When $m = 6$ the following terms are to
be added to $V_s$: $\frac{a^2 x}{2} \left\{\frac{s(s - 1)(2s - 1)}{3!}y^2
   - \tbinom{s}{2}y\right\}p.$}
\begin{align*}
U_s &= \tbinom{s}{2} \{bxz +\beta xy + dyz \}, \\
V_s &= \tbinom{s}{2} \Bigl\{\alpha xy + \bigl[axz + \alpha \beta \tbinom{x}{2}y
  + cyz + ab\tbinom{x}{2}z\bigr]p\Bigr\} \\
  & \qquad \qquad \qquad + \alpha\tbinom{s}{3} \{bxz + \beta xy + dyz \} xp.
\end{align*}

\medskip
5. \textit{Transformation of the groups.} All groups of this
section are given by equations (15), (16), and (17) with $a,\, b,\,
\beta,\, c,\, d = 0,\, 1,\, 2,\, \cdots ,\, p - 1$, and $\alpha = 0,\, 1,\, 2,\,
\cdots ,\, p^2 - 1$, independently. Not all these groups, however,
are distinct. Suppose that $G$ and $G'$ of the above set are
simply isomorphic and that the correspondence is given by
\begin{equation*}
C = \left[
  \begin{matrix}
  R,    & Q,    & P \\
  R'_1, & Q'_1, & P'_1 \\
  \end{matrix}
\right],
\end{equation*}
\noindent in which
\begin{align*}
R'_1 &= R'^{z''} Q'^{y''p} P'^{x''p^{m-4}}, \\
Q'_1 &= R'^{z'} Q'^{y'} P'^{x'p^{m-5}}, \\
P'_1 &= R'^z Q'^y P'^x,
\end{align*}
\noindent where $x$, $y'$ and $z''$ \textit{are prime} to $p$.

The operators $R'_1$, $Q'_1$, and $P'_1$ must be independent since
$R$, $Q$, and $P$ are, and that this is true is easily verified.
The lowest power of $Q'_1$ in $\{P'_1\}$ is $Q'{}_1^{p^2} = 1$ and
the lowest power of $R'_1$ in $\{Q'_1, P'_1\}$ is $R'{}_1^p = 1$.
Let $Q'{}_1^{s'} = P'{}_1^{sp^{m-5}}$.

This in terms of $R'$, $Q'$, and $P'$ is
\begin{gather*}
\Bigl[s'z',\, y'\bigl\{s' + d'\tbinom{s'}{2}z'p\bigr\},\, s'x'p^{m-5} +
c'\tbinom{s'}{2}y'z'p^{m-4}\Bigr] = [0,\, 0,\, sxp^{m-5}]. \\
\intertext{From this equation $s'$ is determined by}
s'z' \equiv 0 \pmod{p} \\
y'\{s' + d'\tbinom{s}{2}z'p\} \equiv 0 \pmod{p^2},
\intertext{which give}
s'y' \equiv 0 \pmod{p^2}.
\intertext{Since $y'$ is prime to $p$}
s' \equiv 0 \pmod{p^2}
\end{gather*}
\noindent and the lowest power of $Q'_1$ contained in $\{P'_1\}$
is $Q'{}_1^{p^2} = 1$.

Denoting by ${R'}_1^{s''}$ the lowest power of $R'_1$ contained in
$\{Q'_1, P'_1\}$.
\begin{equation*}
{R'}_1^{s''} = {Q'}_1^{s'p} {P'}_1^{sp^{m-4}}.
\end{equation*}

This becomes in terms of $R'$, $Q'$, and $P'$
\begin{gather*}
[s''z'',\, s''y''p,\, s''x''p^{m-4}] = [0,\, s'y'p,\, \{s'x' + sx\}p^{m-4}].
\intertext{$s''$ is now determined by}
s''z'' \equiv 0 \pmod{p}
\intertext{and since $z''$ is prime to $p$}
s'' \equiv 0 \pmod{p}.
\end{gather*}
\noindent The lowest power of $R'_1$ contained in $\{Q'_1, P'\}$ is therefore
${R'}_1^p = 1$.

Since $R$, $Q$, and $P$ satisfy equations (15), (16), and (17) $R'_1$,
$Q'_1$, and $P'_1$ also satisfy them. Substituting in these equations the
values of $R'_1$, $Q'_1$, and $P'_1$ and reducing we have in terms of
$R'$, $Q'$, and $P'$
\begin{gather}
[z,\, y + \theta_1 p,\, x + \phi_1 p^{m-5}] =
  [z,\, y + \beta y'p,\, x(1 + \alpha p^{m-5}) + \beta xp^{m-4}], \\ %% 22
[z,\, y + \theta_2 p,\, x + \phi_2 p^{m-4}] =
  [z,\, y + by'p,\, x(1 + ap^{m-4}) + bx'p^{m-4}], \\ %% 23
[z',\, y' + \theta_3 p,\, (x' + \phi_3 p)p^{m-5}] =
  [z',\, y'(1 + dp),\, x(1 + dp)p^{m-5} + cxp^{m-4}],                %% 24
\end{gather}
\noindent in which
\begin{align*}
\theta_1 &= d'(yz' - y'z) + x(b'z' + \beta'y'), \\
\theta_2 &= d'yz'' + b'xz'', \\
\theta_3 &= d'y'z'', \\
\phi_1 &= \alpha'xy' + \bigl\{\alpha'(\beta'y' + b'z')\tbinom{x}{2} +
  a'xz + c'(yz'-y'z)\bigr\}p, \\
\phi_2 &= \alpha'xy'' + a'xz'' + \alpha'b'\tbinom{x}{2}z'' + c'yz'', \\
\phi_3 &= c'yz''.
\end{align*}

A comparison of the members of the above equations give six congruences
between the primed and unprimed constants and the nine indeterminates.

\begin{align*}
\theta_1 &\equiv \beta y' \pmod{p},                \tag{I} \\
\phi_1   &\equiv \alpha x + \beta x'p \pmod{p^2},  \tag{II} \\
\theta_2 &\equiv by' \pmod{p},                     \tag{III} \\
\phi_2   &\equiv ax + bx' \pmod{p},                \tag{IV} \\
\theta_3 &\equiv dy' \pmod{p},                     \tag{V} \\
\phi_3   &\equiv cx + dx' \pmod{p}.                \tag{VI}
\end{align*}

The necessary and sufficient condition for the simple isomorphism of
the two groups $G$ and $G'$ is, \textit{that the above congruences shall be
consistent and admit of solution for the nine indeterminates, with the
condition that $x$, $y'$ and $z''$ be prime to $p$.}

For convenience in the discussion of these congruences, the groups are
divided into six sets, and each set is subdivided into 16 cases.

The group $G'$ is taken from the simplest case, and we associate with
this case all cases, which contain a group $G$, simply isomorphic with
$G'$. Then a single group $G$, in the selected case, simply isomorphic
with $G'$, is chosen as a type.

$G'$ is then taken from the simplest of the remaining cases and we proceed
as above until all the cases are exhausted.

Let $\kappa = \kappa_1 p^{\kappa_2}$, and $dv_1[\kappa_1 ,\, p] = 1$
$(\kappa = a,\, b,\, \alpha ,\, \beta ,\, c,$ and $d)$.

The six sets are given in the table below.

\begin{center}
\large I. \normalsize

\smallskip
\begin{tabular}{|c|c|c||c|c|c|}
\hline
   &$\alpha_2$&$d_2$&   &$\alpha_2$&$d_2$ \\ \hline
$A$&     0    &  0  &$D$&     2    &  0   \\ \hline
$B$&     0    &  1  &$E$&     1    &  1   \\ \hline
$C$&     1    &  0  &$F$&     2    &  1   \\ \hline
\end{tabular}
\end{center}

\medskip
The subdivision into cases and the results are given in Table II.

\begin{center}
\large II. \normalsize

\smallskip
\begin{tabular}{|c|c|c|c|c|c|c|c|c|c|c|}
\hline
  &$a_2$&$b_2$&$\beta_2$&$c_2$& $A$ &  $B$   & $C$ & $D$ &   $E$  &   $F$   \\ \hline
 1&  1  &  1  &    1    &  1  &     &        &     &     &        &         \\ \hline
 2&  0  &  1  &    1    &  1  &$A_1$& $B_1$  &     &$C_2$&        & $E_2$   \\ \hline
 3&  1  &  0  &    1    &  1  &$A_1$&        &$C_1$&$D_1$&        &         \\ \hline
 4&  1  &  1  &    0    &  1  &$A_1$&        &$C_1$&$D_1$&        & $E_4$   \\ \hline
 5&  1  &  1  &    1    &  0  &$A_1$&        &$C_1$&$D_1$&        & $E_5$   \\ \hline
 6&  0  &  0  &    1    &  1  &$A_1$& $B_3$  &$C_2$&$C_2$&  $E_3$ & $F_3$   \\ \hline
 7&  0  &  1  &    0    &  1  &$A_1$& $B_4$  &$C_2$&$C_2$&        & $E_7$   \\ \hline
 8&  0  &  1  &    1    &  0  &$A_1$& $B_5$  &$C_2$&$C_2$&  $E_5$ & $E_5$   \\ \hline
 9&  1  &  0  &    0    &  1  &$A_1$& $B_3$  &$C_1$&$D_1$&  $E_3$ & $F_3$   \\ \hline
10&  1  &  0  &    1    &  0  &$A_1$&        &$C_2$&$C_2$&        &$E_{10}$ \\ \hline
11&  1  &  1  &    0    &  0  &$A_1$&        &  *  &$C_1$&        &$E_{11}$ \\ \hline
12&  0  &  0  &    0    &  1  &$A_1$& $B_3$  &$C_2$&$C_2$&    *   & $E_3$   \\ \hline
13&  0  &  0  &    1    &  0  &$A_1$&$B_{10}$&  *  &  *  &$E_{10}$&$E_{10}$ \\ \hline
14&  0  &  1  &    0    &  0  &$A_1$&$B_{11}$&$C_2$&$C_2$&$E_{11}$&$E_{11}$ \\ \hline
15&  1  &  0  &    0    &  0  &$A_1$&$B_{10}$&$C_2$&$C_2$&$E_{10}$&$E_{10}$ \\ \hline
16&  0  &  0  &    0    &  0  &$A_1$&$B_{10}$&  *  &  *  &$E_{10}$&$E_{10}$ \\ \hline
\end{tabular}

\footnotesize The groups marked (*) divide into two or three parts. \normalsize
\end{center}

\medskip
Let $ad - bc = \theta_1 p^{\theta_2}$, $\alpha_1 d - \beta c =
\phi_1 p^{\phi_2}$ and $\alpha_1 b - a\beta = \chi_1 p^{\chi_2}$ with
$\theta_1$, $\phi_1$, and $\chi_1$ prime to $p$.

\begin{center}
\large III. \normalsize

\smallskip
\begin{tabular}{|c|c|c|c|c||c|c|c|c|c|}
\hline
    *   &$\theta_2$&$\phi_2$&$\chi_2$&    &    *   &$\theta_2$&$\phi_2$&$\chi_2$&      \\ \hline
$C_{11}$&          &    1   &       &$D_1$&$D_{13}$&    1     &        &        &$D_1$ \\ \hline
$C_{11}$&          &    0   &       &$C_1$&$D_{13}$&    0     &        &        &$C_2$ \\ \hline
$C_{13}$&    1     &        &       &$C_1$&$D_{16}$&    1     &        &        &$C_1$ \\ \hline
$C_{13}$&    0     &        &       &$C_2$&$D_{16}$&    0     &        &        &$C_2$ \\ \hline
$C_{16}$&    1     &    1   &       &$D_1$&$E_{12}$&          &        &    1   &$F_3$ \\ \hline
$C_{16}$&    1     &    0   &       &$C_1$&$E_{12}$&          &        &    0   &$E_3$ \\ \hline
$C_{16}$&    0     &        &       &$C_2$&        &          &        &        &      \\ \hline
\end{tabular}

\newpage
6. \textit{Types.}
\end{center}

The type groups are given by equations (15), (16) and (17) with the
values of the constants given in Table IV.

\begin{center}
\large IV. \normalsize

\smallskip
\begin{tabular}{|c|c|c|c|c|c|c||c|c|c|c|c|c|c|}
\hline
        &   $a$  &$b$&$\alpha$&$\beta$&   $c$  &$d$&        &$a$&$b$&$\alpha$&$\beta$&   $c$  &$d$ \\ \hline
$A_1$   &    0   & 0 &    1   &   0   &    0   & 1 &$E_1$   & 0 & 0 &   $p$  &   0   &    0   & 0  \\ \hline
$B_1$   &    0   & 0 &    1   &   0   &    0   & 0 &$E_2$   & 1 & 0 &   $p$  &   0   &    0   & 0  \\ \hline
$B_3$   &    0   & 1 &    1   &   0   &    0   & 0 &$E_3$   & 0 & 1 &   $p$  &   0   &    0   & 0  \\ \hline
$B_4$   &    0   & 0 &    1   &   1   &    0   & 0 &$E_4$   & 0 & 0 &   $p$  &   1   &    0   & 0  \\ \hline
$B_5$   &    0   & 0 &    1   &   0   &    1   & 0 &$E_5$   & 0 & 0 &   $p$  &   0   &    1   & 0  \\ \hline
$B_{10}$&    0   & 1 &    1   &   0   &$\kappa$& 0 &$E_7$   & 1 & 0 &   $p$  &   1   &    0   & 0  \\ \hline
$B_{11}$&    0   & 0 &    1   &   1   &    1   & 0 &$E_{10}$& 0 & 1 &   $p$  &   0   &$\kappa$& 0  \\ \hline
$C_1$   &    0   & 0 &   $p$  &   0   &    0   & 1 &$E_{11}$& 0 & 0 &   $p$  &   1   &    1   & 0  \\ \hline
$C_2$   &$\omega$& 0 &   $p$  &   0   &    0   & 1 &$F_1$   & 0 & 0 &   0    &   0   &    0   & 0  \\ \hline
$D_1$   &    0   & 0 &    0   &   0   &    0   & 1 &$F_3$   & 0 & 1 &   0    &   0   &    0   & 0  \\ \hline
\end{tabular}

\footnotesize
\begin{align*}
\kappa &= 1, \hbox{ and a non-residue } \pmod{p}, \\
\omega &= 1, 2, \cdots, p - 1.
\end{align*}
\normalsize
\end{center}

\medskip
The congruences for three of these cases are completely analyzed as
illustrations of the methods used.

\medskip
\begin{equation*} B_{10}. \end{equation*}

The congruences for this case have the special forms.
\begin{gather*}
b'xz' \equiv \beta y' \pmod{p},                                            \tag{I} \\
\alpha'y' \equiv \alpha \pmod{p},                                          \tag{II} \\
b'xz'' \equiv by' \pmod{p},                                                \tag{III} \\
\alpha'xy'' + \alpha'b'\tbinom{x}{2}z'' + c'yz'' \equiv ax + bx' \pmod{p}, \tag{IV} \\
d \equiv 0 \pmod{p},                                                       \tag{V} \\
c'y'z'' \equiv cx \pmod{p}.                                                \tag{VI}
\end{gather*}

Since $z'$ is unrestricted (I) gives $\beta \equiv 0$ or $\not\equiv 0 \pmod{p}$.

From (II) since $y' \not\equiv 0, \alpha \not\equiv 0 \pmod{p}$.

From (III) since $x, y', z'' \not\equiv 0$, $b \not\equiv 0 \pmod{p}$.

In (IV) $b \not\equiv 0$ and $x'$ is contained in this congruence alone,
and, therefore, $a$ may be taken $\equiv 0$ or $\not\equiv 0 \pmod{p}$.

(V) gives $d \equiv 0 \pmod{p}$ and (VI), $c \not\equiv 0 \pmod{p}$.

Elimination of $y'$ between (III) and (VI) gives
\begin{equation*}
b'c'z''^{2} \equiv bc \pmod{p}
\end{equation*}
\noindent so that $bc$ is a quadratic residue or non-residue (mod $p$)
according as $b'c'$ is a residue or non-residue.

The types are given by placing $a = 0$, $b = 1$, $\alpha = 1$, $\beta = 0$,
$c = \kappa$, and $d = 0$ where $\kappa$ has the two values, 1 and a
representative non-residue of $p$.

\medskip
\begin{equation*} C_2. \end{equation*}

The congruences for this case are
\begin{gather*}
d'(yz' - y'z) \equiv \beta y' \pmod{p},                      \tag{I} \\
\alpha'_1 xy' + a'xz' \equiv \alpha_1 x + \beta x' \pmod{p}, \tag{II} \\
d'yz'' \equiv by' \pmod{p},                                  \tag{III} \\
a'xz'' \equiv ax + bx' \pmod{p},                             \tag{IV} \\
d'z'' \equiv d \pmod{p},                                     \tag{V} \\
cx + dx' \equiv 0 \pmod{p}.                                  \tag{VI}
\end{gather*}

Since $z$ appears in (I) alone, $\beta$ can be either $\equiv 0$ or
$\not\equiv 0 \pmod{p}$. (II) is linear in $z'$ and, therefore, $\alpha
\equiv 0$ or $\not\equiv 0 \pmod{p}$, (III) is linear in $y$ and,
therefore, $b \equiv 0$ or $\not\equiv 0$.

Elimination of $x'$ and $z''$ between (IV), (V), and (VI) gives
\begin{equation*}
a'd^2 \equiv d'(ad - bc) \pmod{p}.
\end{equation*}
\noindent Since $z''$ is prime to $p$, (V) gives $d \not\equiv 0 \pmod{p}$, so that
$ad - bc \not\equiv 0 \pmod{p}$. We may place $b = 0$, $\alpha = p$,
$\beta = 0$, $c = 0$, $d = 1$, then $a$ will take the values $1, 2, 3, \cdots,
p - 1$ giving $p - 1$ types.

\medskip
\begin{equation*} D_1. \end{equation*}

The congruences for this case are
\begin{align*}
d'(yz' - y'z) &\equiv \beta y' \pmod{p},  \tag{I} \\
\alpha_1 x + \beta x' &\equiv 0 \pmod{p}, \tag{II} \\
d'yz'' &\equiv by' \pmod{p},              \tag{III} \\
ax + bx' &\equiv 0 \pmod{p},              \tag{IV} \\
d'z'' &\equiv d \pmod{p},                 \tag{V} \\
cx + dx' &\equiv 0 \pmod{p}.              \tag{VI}
\end{align*}
\noindent $z$ is contained in (I) alone, and therefore $\beta \equiv 0$ or
$\not\equiv 0 \pmod{p}$.

(III) is linear in $y$, and $b \equiv 0$ or $\not\equiv 0 \pmod{p}$.

(V) gives $d \not\equiv 0 \pmod{p}$.

Elimination of $x'$ between (II) and (VI) gives $\alpha_1 d - \beta c
\equiv 0 \pmod{p}$, and between (IV) and (VI) gives $ad - bc \equiv 0
\pmod{p}$. The type group is derived by placing $a = 0$, $b = 0$, $\alpha = 0$,
$\beta = 0$, $c = 0$ and $d = 1$.

\bigskip
\begin{center}
\large\textit{Section} 2. \normalsize
\end{center}
\setcounter{equation}{0}

1. \textit{Groups with dependent generators.} In this section, $G$ is generated
by $Q_1$ and $P$ where
\begin{equation}
Q{}_1^{p^2} = P^{hp^2}. %% 1
\end{equation}
\noindent There is in $G$, a subgroup $H_1$, of order $p^{m-2}$, which contains $\{P\}$
self-con\-ju\-gate\-ly.\footnote{\textsc{Burnside}, \textit{Theory of Groups},
Art.\ 54, p.\ 64.} $H_1$ either contains, or does not contain $Q{}_1^p$. We will
consider the second possibility in the present section, reserving the first for
the next section.

\medskip
2. \textit{Determination of $H_1$.} $H_1$ is generated by $P$ and some other operator
$R_1$ of $G$. $R{}_1^p$ is contained in $\{P\}$. Let
\begin{equation}
R{}_1^p = P^{lp}. %% 2
\end{equation}
\noindent Since $\{P\}$ is self-conjugate in $H_1$,\footnote{\textsc{Burnside},
\textit{Theory of Groups}, Art.\ 56, p.\ 66.}
\begin{equation}
R{}_1^{-1}\, P\, R_1 = P^{1 + kp^{m-4}} %% 3
\end{equation}
\noindent Denoting $R{}_1^a\, P^b\, R{}_1^c\, P^d \cdots$ by the symbol $[a,\, b,\, c,\, d,\,
\cdots]$ we derive from (3)
\begin{gather}
[-y,\, x,\, y] = [0,\, x(1 + kyp^{m-4})] \qquad (m > 4), %% 4
\intertext{and}
[y,\, x]^s = \Bigl[sy,\, x\bigl\{s + k\tbinom{s}{2}yp^{m-4}\bigr\}\Bigr] %% 5
\end{gather}
\noindent Placing $s = p$ and $y = 1$ in (5) we have, from (2)
\begin{gather*}
[R_1\, P^x]^p = R{}_1^p P^{xp} = P^{(l + x)p}.
\intertext{Choosing $x$ so that}
x + l \equiv 0 \pmod{p^{m-4}},
\end{gather*}
\noindent $R = R_1 P^x$ is an operator of order $p$, which will be used in the place
of $R_1$, and $H = \{R, P\}$ with $R^p = 1$.

\medskip
3. \textit{Determination of $H_2$.} We will now use the symbol $[a,\, b,\, c,\, d,\, e,\, f,\,
\cdots]$ to denote $Q{}_1^a\, R^b\, P^c\, Q{}_1^d\, R^e\, P^f \cdots$.

$H_1$ and $Q_1$ generate $G$ and all the operations of $G$ are given by
$[x,\, y,\, z]$ ($z = 0,\, 1,\, 2,\, \cdots,\, p^2 - 1$; $y = 0,\, 1,\, 2,\, \cdots,\, p - 1$;
$x = 0,\, 1,\, 2,\, \cdots,\, p^{m-3} - 1$), since these are $p^m$ in number and
are all distinct. There is in $G$ a subgroup $H_2$ of order $p^{m-1}$
which contains $H_1$ self-conjugately. $H_2$ is generated by $H_1$ and
some operator $[z,\, y,\, x]$ of $G$. $Q{}_1^z$ is then in $H_2$ and $H_2$
is the subgroup $\{Q{}_1^p, H_1\}$. Hence,
\begin{gather}
Q{}_1^{-p}\, P\, Q{}_1^p = R^\beta P^{\alpha_1}, \\ %% 6
Q{}_1^{-p}\, P\, Q{}_1^p = R^{b_1} P^{ap^{m-4}}.    %% 7
\end{gather}
\noindent To determine $\alpha_1$ and $\beta$ we find from (6), (5) and (7)
\begin{multline*}
[-p^2,\, 0,\, 1,\, p^2] = \biggl[ 0,\, \frac{\alpha{}_1^p - b{}_1^p}{\alpha_1 - b_1}
    \beta,\, \alpha{}_1^p\Bigl\{1 + \frac{\beta k}{2}\frac{\alpha{}_1^p - 1}
    {\alpha_1 - 1}p^{m-4} \Bigr\} \\
  + a\beta\Bigl\{ p\frac{\alpha{}_1^{p-1}}{\alpha_1 - b_1} -
    \frac{\alpha{}_1^p - b{}_1^p}{(\alpha_1 - b_1)^2}
    \Bigr\}p^{m-4} \biggr].
\end{multline*}
\noindent By (1)
\begin{gather*}
Q{}_1^{-p^2}\, P\, Q{}_1^{p^2} = P,
\intertext{and, therefore,}
\frac{\alpha{}_1^p - b{}_1^p}{\alpha_1 - b_1}\beta \equiv 0 \pmod{p}, \\
  \alpha{}_1^p \equiv 1 \pmod{p^{m-4}}, \qquad \hbox{ and } \qquad
  \alpha_1 \equiv 1 \pmod{p^{m-5}} \quad  (m > 5).
\end{gather*}

Let $\alpha_1 = 1 + \alpha_2 p^{m-5}$ and equation (6) is replaced by
\begin{equation}
Q{}_1^{-p}\, P\, Q{}_1^p = R^\beta P^{1 + \alpha_2 p^{m-5}}. %% 8
\end{equation}

To find $a$ and $b_1$ we obtain from (7), (8) and (5)
\begin{gather*}
[-p^2,\, 1,\, 0,\, p^2] = \Bigl[0,\, b{}_1^p,\, a\frac{b{}_1^p - 1}{b_1 - 1}p^{m-4} \Bigr].
\intertext{By (1) and (4)}
Q{}_1^{-p^2}\, R\, Q{}_1^{p^2} = P^{-lp^2} R\, P^{lp^2} = R,
\intertext{and, hence,}
b{}_1^p \equiv 1 \pmod{p}, \qquad a\frac{b{}_1^p - 1}{b_1 - 1} \equiv 0 \pmod{p},
\end{gather*}
\noindent therefore $b_1 = 1$.

Substituting $b_1 = 1$ and $\alpha_1 = 1 + \alpha_2 p^{m-5}$ in the
congruence determining $\alpha_1$ we obtain $(1 + \alpha_2 p^{m-5})^p
\equiv 1 \pmod{p^{m-3}}$, which gives $\alpha_2 \equiv 0 \pmod{p}$.

Let $\alpha_2 = \alpha p$ and equations (8) and (7) are now replaced by
\begin{align}
Q{}_1^p\,    P\, Q{}_1^p &= R^\beta P^{1 + \alpha p^{m-4}}, \\ %% 9
Q{}_1^{-p}\, R\, Q{}_1^p &= RP^{ap^{m-4}}.                     %% 10
\end{align}

From these we derive
\begin{align}
[-yp,\, 0,\, x,\, yp] &= \Bigl[0,\, \beta xy,\, x + \bigl\{\alpha xy
  + a\beta x\tbinom{y}{2} + \beta k\tbinom{x}{2}y\bigr\}p^{m-4}\Bigr], \\ %% 11
[-yp,\, x,\, 0,\, yp] &= [0,\, x,\, axyp^{m-4}].                          %% 12
\end{align}

A continued use of (4), (11), and (12) yields
\begin{equation}
[zp,\, y,\, x]^s = [szp,\, sy + U_s,\, sx + V_sp^{m-4}]  %% 13
\end{equation}
\noindent where
\begin{align*}
U_s &= \beta\tbinom{s}{2}xz, \\
V_s &= \tbinom{s}{2}\Bigl\{\alpha xz + \beta k\tbinom{s}{2}z + kxy
       + ayz\Bigr\} + \beta k\tbinom{s}{3}x^2 z \\
 & \qquad \qquad \qquad + \frac{1}{2}a\beta\Bigl\{\frac{1}{3!}s(s - 1)(2s - 1)z^2
       - \tbinom{s}{2}z\Bigr\}.
\end{align*}

\medskip
4. \textit{Determination of $G$.}

Since $H_2$ is self-conjugate in $G_1$ we have
\begin{align}
Q{}_1^{-1}\, P\, Q_1 &= Q{}_1^{\gamma p} R^\delta P^{\epsilon_1}, \\ %% 14
Q{}_1^{-1}\, R\, Q_1 &= Q{}_1^{cp} R^d P^{ep^{m-4}}.                 %% 15
\end{align}

From (14), (15) and (13)
\begin{gather*}
[-p,\, 0,\, 1,\, p] = [\lambda p,\, \mu,\, \epsilon{}_1^p + vp^{m-4}]
\intertext{and by (9) and (1)}
\lambda p \equiv 0 \pmod{p^2}, \qquad
\epsilon{}_1^p + \nu p^{m-4} + \lambda hp \equiv 1 + \alpha p^{m-4}
  \pmod{p^{m-3}},
\intertext{from which}
\epsilon{}_1^p \equiv 1 \pmod{p^2}, \quad \hbox{ and } \quad \epsilon_1 \equiv 1 \pmod{p}
  \qquad (m > 5).
\end{gather*}

Let $\epsilon_1 = 1 + \epsilon_2 p$ and equation (14) is replaced by
\begin{equation}
Q{}_1^{-1}\, P\, Q_1 = Q{}_1^{\gamma p} R^\delta P^{1 + \epsilon_2 p}. %% 16
\end{equation}

From (15), (16), and (13)
\begin{gather*}
[-p,\, 1,\, 0,\, p] = \left[c\frac{d^p - 1}{d - 1}p,\, d^p,\, Kp^{m-4} \right]
\intertext{where}
K = \frac{d^p - 1}{d - 1}e + \sum_{1}^{p-1} acd\frac{d^n(d^n - 1)}{2}.
\end{gather*}

By (10)
\begin{gather*}
d^p \equiv 1 \pmod{p}, \qquad \hbox{ and } \qquad d = 1
\intertext{and by (1)}
chp^2 \equiv ap^{m-4} \pmod{p^{m-3}}.
\end{gather*}

Equation (15) is now replaced by
\begin{equation}
Q{}_1^{-1}\, R\, Q_1 = Q{}_1^{cp} R P^{ep^{m-4}}. %% 17
\end{equation}

A combination of (17), (16) and (13) gives
\begin{equation*}
[-p,\, 0,\, 1,\, p] = \Bigl[\bigl\{\gamma\frac{(1 + \epsilon_2 p)^p - 1}
{\epsilon_2 p^2} + c\delta\frac{p - 1}{2} \bigr\}p^2,\, 0,\, (1 +
\epsilon_2 p)^p \Bigr].
\end{equation*}

By (9)
\begin{equation*}
\Bigl\{\gamma\frac{(1 + \epsilon_2 p)^p - 1}{\epsilon_2 p^2} + c\delta
\frac{p - 1}{2}\Bigr\}hp^2 + (1 + \epsilon_2 p)^p \equiv 1 + \alpha p^{m-4}
\pmod{p^{m-3}},
\end{equation*}
\noindent $\beta \equiv 0 \pmod{p}.$

A reduction of the first congruence gives
\begin{gather*}
\frac{(1 + \epsilon_2 p)^p - 1}{\epsilon p^2}\bigl\{\epsilon_2 + \gamma h\bigr\}p^2
  \equiv \Bigl\{\alpha - a\delta\frac{p - 1}{2}\Bigr\}p^{m-4} \pmod{p^{m-3}}
\intertext{and, since}
\frac{(1 + \epsilon_2 p)^p - 1}{\epsilon_2 p^2} \equiv 1 \pmod{p}, \qquad
(\epsilon_2 + \gamma h)p^2 \equiv 0 \pmod{p^{m-4}}
\end{gather*}
\noindent and
\begin{equation}
(\epsilon_2 + \gamma h)p^2 \equiv \bigl(\alpha + \frac{a\delta}{2}\bigr)p^{m-4}
  \pmod{p^{m-3}}. %% 18
\end{equation}

From (17), (16), (13) and (18)
\begin{align}
[-y,\, x,\, 0,\, y] &= \Bigl[cxyp,\, x,\, \bigl\{exy + ac\tbinom{x}{2}y\bigr\}p^{m-4}\Bigr], \\ %% 19
[-y,\, 0,\, x,\, y] &= \Bigl[x\bigl\{\gamma y + c\delta\tbinom{y}{2}\bigr\}p,\,
                 \delta xy,\, x(1 + \epsilon_2 yp) + \theta p^{m-4}\Bigr] %% 20
\end{align}
\noindent where
\begin{multline*}
\theta = \Bigl\{e\delta x + a\delta\gamma x + \epsilon_2 \left(\alpha
    + \frac{a\delta}{2}\right)x\Bigr\}\tbinom{y}{2} \\
  + \frac{1}{2}ac \Bigl\{\frac{1}{3!}y(y-1)(2y-1)\delta^2
    - \tbinom{y}{2}\delta\Bigr\} \\
  + \bigl\{\alpha\gamma y + \delta ky + a\delta xy^2
    + (ac\delta^2 y + ac\delta) \tbinom{y}{2}\bigr\} \tbinom{x}{2}.
\end{multline*}

From (19), (20), (4) and (18)
\begin{equation*}
\{Q_1\, P^x\}^{p^2} = Q{}_1^{p^2} P^{xp^2} = P^{(h+x)p^2}.
\end{equation*}

If $x$ be so chosen that
\begin{equation*}
h + x \equiv 0 \pmod{p^{m-5}}
\end{equation*}
\noindent $Q = Q_1\, P^x$ is an operator of order $p^2$ which will be used in place
$Q_1$ and $Q^{p^2} = 1$.

Placing $h = 0$ in (18) we get
\begin{equation*}
\epsilon_2 p^2 \equiv 0 \pmod{p^{m-4}}.
\end{equation*}

Let $\epsilon_2 = \epsilon p^{m-6}$ and equation (16) is replaced by
\begin{equation}
Q^{-1}\, P\, Q = Q^{\gamma p} R^\delta P^{1 + \epsilon p^{m-5}} %% 21
\end{equation}

The congruence
\begin{gather*}
ap^{m-4} \equiv chp^2 \pmod{p^{m-3}}
\intertext{becomes}
ap^{m-4} \equiv 0 \pmod{p^{m-3}}, \qquad \hbox{ and } \qquad a \equiv 0 \pmod{p}.
\end{gather*}
\noindent Equations (19) and (20) are replaced by
\begin{align}
[-y,\, x,\, 0,\, y] &= [cxyp,\, x,\, exyp^{m-4}] \\  %% 22
[-y,\, 0,\, x,\, y] &= \Bigl[ \bigl\{\gamma y + c\delta\tbinom{y}{2} \bigr\}xp,\,
   \delta xy,\, x(1 + \epsilon yp^{m-5}) + \theta p^{m-4} \Bigr] %% 23
\end{align}
\noindent where
\begin{equation*}
\theta = e\delta x\tbinom{y}{2} + \bigl\{\alpha\gamma y + \delta ky +
  \alpha c\delta\tbinom{y}{2}\bigr\}\tbinom{x}{2}.
\end{equation*}

A formula for any power of an operation of $G$ is derived from (4),
(22) and (23)
\begin{equation}
[z,\, y,\, x]^s = [sz + U_s p,\, sy + V_s,\, sx + W_s p^{m-5}] %% 24
\end{equation}
\noindent where
\begin{align*}
U_s &= \tbinom{s}{2}\bigl\{\gamma xz + cyz\bigr\} + \frac{1}{2}c\delta x
   \Bigl\{\frac{1}{3!}s(s - 1)(2s - 1)z^2 - \tbinom{s}{2}z\Bigr\}, \\
V_s &= \delta \tbinom{s}{2}xz, \\
W_s &= \tbinom{s}{2} \Bigl\{\epsilon xz + \bigl[(a\gamma + \delta k)\tbinom{x}{2}z + eyz + kxy\bigr]p\Bigr\} \\
   & \qquad \qquad + \tbinom{s}{3}\bigl\{\epsilon \gamma x + \epsilon y + \delta kx \bigr\}xzp
      + \frac{1}{2}c \delta \epsilon \bigl\{\frac{1}{2}(s - 1)z^2 - z\bigr\} \tbinom{s}{3}xp \\
   & \qquad \qquad + \frac{1}{2}\bigl\{\delta ex + \alpha c \delta \tbinom{x}{2}\bigr\}
      \bigl\{\frac{1}{3!}s(s - 1)(2s - 1)z^2 - \tbinom{s}{2}z\bigr\}p.
\end{align*}

\medskip
5. \textit{Transformations of the groups.} Placing $y = 1$ and $x = -1$ in (22)
we obtain (17) in the form
\begin{equation*}
R^{-1}\, Q\, R = Q^{1-cp} P^{-ep^{m-4}}.
\end{equation*}
\noindent A comparison of the generational equations of the present section with
those of Section 1, shows that groups, in which $\delta \equiv 0 \pmod{p}$,
are simply isomorphic with those in Section 1, so we need consider only
those cases in which $\delta \not\equiv 0 \pmod{p}$.

All groups of this section are given by
\begin{equation*}
G: \left\{ \begin{aligned}
                R^{-1}\, P\, R &= P^{1 + kp^{m-4}}, \\
                Q^{-1}\, P\, Q &= Q^{\gamma p} R^\delta P^{1 + \epsilon p^{m-5}}, \\
                Q^{-1}\, R\, Q &= Q^{cp} R\, P^{\epsilon p^{m-4}}. \\ \end{aligned}
\right. \tag*{(25), (26), (27)}
\end{equation*}
\noindent $R^p = 1$, $Q^{p^2} = 1$, and $P^{p^{m-3}} = 1$, $(k,\, \gamma,\, c,\, e = 0,\, 1,\,
2,\, \cdots ,\, p - 1$; $\delta = 1,\, 2,\, \cdots,\, p - 1$; $\epsilon = 0,\, 1,\, 2,\, \cdots,\,
p^2 - 1)$.

Not all these groups, however, are distinct. Suppose that $G$ and $G'$ of
the above set are simply isomorphic and that the correspondence is given by
\begin{equation*}
C = \left[\begin{matrix}R,    & Q,    & P \\
                        R'_1, & Q'_1, & P'_1 \\ \end{matrix} \right] .
\end{equation*}
\noindent Since $R^p = 1$, $Q^{p^2} = 1$, and $P^{p^{m-3}} = 1$, $R'{}_1^p = 1$,
$Q'{}_1^{p^2} = 1$ and $P'{}_1^{p^{m-3}}$.

The forms of these operators are then
\begin{align*}
P'_1 &= Q'^z R'^y P'^x, \\
R'_1 &= Q'^{z'p} R'^{y'} P'^{x'p^{m-4}}, \\
Q'_1 &= Q'^{z''} R'^{y''}P'^{x''p^{m-5}}, \
\end{align*}
\noindent where $dv[x,\, p] = 1$.

Since $R$ is not contained in $\{P\}$, and $Q^p$ is not contained in
$\{R, P\}$ $R'_1$ is not contained in $\{P'_1\}$, and $Q'{}_1^p$ is not contained in
$\{R'_1, P'_1\}$.

Let
\begin{gather*}
{R'}_1^{s'} = {P'}_1^{sp^{m-4}}.
\intertext{This becomes in terms of $Q'$, $R'$ and $P'$}
[s'z'p,\, s'y',\, s'x'p^{m-4}] = [0,\, 0,\, sxp^{m-4}],
\intertext{and}
s'y' \equiv 0 \pmod{p}, \qquad s'z' \equiv 0 \pmod{p}.
\end{gather*}
\noindent Either $y'$ or $z'$ is prime to $p$ or $s'$ may be taken $= 1$.

Let
\begin{gather*}
{Q'}_1^{s''p} = {R'}_1^{s'} P'{}_1^{sp^{m-4}},
\intertext{and in terms of $Q'$, $R'$ and $P'$}
[s''z''p,\, 0,\, s''x''p^{m-4}] = [s'z'p,\, s'y',\, (s'x' + sx)p^{m-4}],
\intertext{from which}
s''z'' \equiv s'z' \pmod{p}, \qquad \hbox{ and } \qquad s'y' \equiv 0 \pmod{p}.
\intertext{Eliminating $s'$ we find}
s''y'z'' \equiv 0 \pmod{p},
\end{gather*}
\noindent $dv[y'z'',\, p] = 1$ or $s''$ may be taken $= 1$. We have then $z''$, $y'$
and $x$ \textit{prime to} $p$.

Since $R$, $Q$ and $P$ satisfy equations (25), (26) and (27) $R'_1$, $Q'_1$
and $P'_1$ do also. These become in terms of $R'$, $Q'$ and $P'$.
\begin{align*}
[z + \Phi'_1 p,\, y,\, x + \Theta'_1 p^{m-4}] &= [z,\, y,\, x(1 + kp^{m-4})], \\
[z + \Phi'_2 p,\, y + \delta'xz'',\, x + \Theta'_2 p^{m-5}] &= [z + \Phi_2 p,\,
  y + \delta y',\, x + \Theta_2 p^{m-5}], \\
[(z' + \Phi'_3)p,\, y',\, \Theta'_3 p^{m-4}] &= [(z' + \Phi_3)p,\, y,\, \Theta'_3 p^{m-4}],
\end{align*}
\noindent where
\begin{align*}
\Phi'_1 &= -c'yz', \quad \Theta'_1 = \epsilon'xz' + k'xy' - e'y'z, \\
\Phi'_2 &= \Bigl\{\gamma'z'' + c'\delta'\tbinom{z}{2}\Bigr\}x + c'(yz'' - y''z), \\
\Theta'_2 &= \epsilon'xz'' + \Bigl\{\tbinom{x}{2}\bigl[\alpha'\gamma'z''
     + \alpha'c'\delta'\tbinom{z''}{2} + \delta'k'z''\bigr] \\
  & \qquad \qquad \qquad + \delta'e'x \tbinom{z''}{2} + e'(yz'' - y''z) + k'xy''\Bigr\}p, \\
\Phi_2 &= \gamma z'' + \delta z' + c'\delta y'z, \quad \Theta_2 \equiv
  \epsilon x + (\gamma x'' + \delta x + e'\delta y'z)p, \\
\Phi'_3 &= c'y'z'', \quad \Theta'_3 = e'y'z'', \quad \Phi_3 = cz'', \quad
  \Theta_3 = ex + cx''.
\end{align*}

A comparison of the members of these equations give seven congruences
\begin{align*}
\Phi'_1     &\equiv 0         \pmod{p},   \tag{I} \\
\Theta'_1   &\equiv kx        \pmod{p},   \tag{II} \\
\Phi'_2     &\equiv \Phi_2    \pmod{p},   \tag{III} \\
\delta'xz'' &\equiv \delta y' \pmod{p},   \tag{IV} \\
\Theta'_2   &\equiv \Theta_2  \pmod{p^2}, \tag{V} \\
\Phi_3'     &\equiv cz''      \pmod{p},   \tag{VI} \\
\Theta'_3   &\equiv \Theta_3  \pmod{p}.   \tag{VII}
\end{align*}

The necessary and sufficient condition for the simple isomorphism of $G$
and $G'$ is, \textit{that these congruences be consistent and admit of solution
for the nine indeterminants with $x$, $y'$, and $z''$ prime to $p$}.

Let $\kappa = \kappa_1 p^{\kappa_2},\, dv[\kappa_1,\, p] = 1\; (\kappa = k,\,
\delta,\, \gamma,\, \epsilon,\, c,\, e)$.

The groups are divided into three parts and each part is subdivided into
16 cases.

The methods used in discussing the congruences are the same as those
used in Section 1.

\medskip
6. \textit{Reduction to types.} The three parts are given by

\begin{center}
\large I. \normalsize

\smallskip
\begin{tabular}{|c|c|c|} \hline
     & $\epsilon_2$ & $\delta_2$ \\ \hline
 $A$ &      0       &      0     \\ \hline
 $B$ &      1       &      0     \\ \hline
 $C$ &      2       &      0     \\ \hline
\end{tabular}
\end{center}

The subdivision into cases and the results of the discussion of the
congruences are given in Table II.

\medskip
\begin{center}
\large II. \normalsize

\smallskip
\begin{tabular}{|c|c|c|c|c|c|c|c|} \hline
    &$k_2$&$\gamma_2$&$c_2$&$e_2$&  $A$  &  $B$  &  $C$  \\ \hline
 1  &  1  &    1     &  1  &  1  &       &       & $B_1$ \\ \hline
 2  &  0  &    1     &  1  &  1  &       &       & $B_2$ \\ \hline
 3  &  1  &    0     &  1  &  1  & $A_2$ & $B_1$ & $B_1$ \\ \hline
 4  &  1  &    1     &  0  &  1  &       &       & $B_4$ \\ \hline
 5  &  1  &    1     &  1  &  0  &       &       & $B_5$ \\ \hline
 6  &  0  &    0     &  1  &  1  &   *   & $B_2$ & $B_2$ \\ \hline
 7  &  0  &    1     &  0  &  1  & $A_4$ &       & $B_7$ \\ \hline
 8  &  0  &    1     &  1  &  0  & $A_5$ & $B_5$ & $B_5$ \\ \hline
 9  &  1  &    0     &  0  &  1  & $A_4$ & $B_4$ & $B_4$ \\ \hline
 10 &  1  &    0     &  1  &  0  & $A_5$ & $B_5$ & $B_5$ \\ \hline
 11 &  1  &    1     &  0  &  0  & $A_4$ & $B_4$ & $B_4$ \\ \hline
 12 &  0  &    0     &  0  &  1  & $A_4$ & $B_7$ & $B_7$ \\ \hline
 13 &  0  &    0     &  1  &  0  & $A_5$ & $B_5$ & $B_5$ \\ \hline
 14 &  0  &    1     &  0  &  0  & $A_4$ & $B_7$ & $B_7$ \\ \hline
 15 &  1  &    0     &  0  &  0  & $A_4$ & $B_4$ & $B_4$ \\ \hline
 16 &  0  &    0     &  0  &  0  & $A_4$ & $B_7$ & $B_7$ \\ \hline
\end{tabular}
\end{center}

$A_6$ divides into two parts.

The groups of $A_6$ in which $\delta k + \epsilon\gamma \equiv 0 \pmod{p}$
are simply isomorphic with the groups of $A_1$ and those in which $\delta
k + \epsilon\gamma \not\equiv 0 \pmod{p}$ are simply isomorphic with the
groups of $A_2$. The types are given by equations (25), (26) and (27) where
the constants have the values given in Table III.

\begin{center}
\large III. \normalsize

\smallskip
\begin{tabular}{|c|c|c|c|c|c|c|} \hline
       & $k$ & $\delta$ & $\gamma$ & $\epsilon$ &    $c$   &    $e$   \\ \hline
 $A_1$ &  0  &     1    &    0     &      1     &     0    &     0    \\ \hline
 $A_2$ &  1  &     1    &    0     &      1     &     0    &     0    \\ \hline
 $A_4$ &  0  &     1    &    0     &      1     &     1    &     0    \\ \hline
 $A_5$ &  0  &     1    &    0     &      1     &     0    & $\omega$ \\ \hline
 $B_1$ &  0  &     1    &    0     &     $p$    &     0    &     0    \\ \hline
 $B_2$ &  1  &     1    &    0     &     $p$    &     0    &     0    \\ \hline
 $B_4$ &  0  &     1    &    0     &     $p$    &     1    &     0    \\ \hline
 $B_5$ &  0  &     1    &    0     &     $p$    &     0    & $\kappa$ \\ \hline
 $B_7$ &  1  &     1    &    0     &     $p$    & $\omega$ &     0    \\ \hline
\end{tabular}

\footnotesize
\begin{align*}
\kappa &= 1, \hbox { and a non-residue } \pmod{p}, \\
\omega &= 1, 2, \cdots, p - 1.
\end{align*}
\normalsize
\end{center}

A detailed analysis of several cases is given below, as a general
illustration of the methods used.

\medskip
\begin{equation*} A_1. \end{equation*}

The special forms of the congruences for this case are
\begin{align*}
          \epsilon'xz' &\equiv kx         \pmod{p}, \tag{II} \\
\gamma z'' + \delta z' &\equiv 0          \pmod{p}, \tag{III} \\
           \delta'xz'' &\equiv \delta y'  \pmod{p}, \tag{IV} \\
         \epsilon'xz'' &\equiv \epsilon x \pmod{p}, \tag{V} \\
                  cz'' &\equiv 0          \pmod{p}, \tag{VI} \\
                    ex &\equiv 0          \pmod{p}. \tag{VII}
\end{align*}
\noindent Congruence (IV) gives $\delta \not\equiv 0 \pmod{p}$, from (II) $k$ can be
$\equiv 0$ or $\not\equiv 0 \pmod{p}$, (III) gives $\gamma \equiv 0$
or $\not\equiv 0$, (V) gives $\epsilon \not\equiv 0$, (VI) and (VII)
give $c \equiv e \equiv 0 \pmod{p}$. Elimination of $x$, $z'$ and $z''$
between (II), (III) and (V) gives $\delta k + \gamma\epsilon \equiv 0
\pmod{p}$. If $k \equiv 0$, then $\gamma \equiv 0 \pmod{p}$ and
if $k \not\equiv 0$, then $\gamma \not\equiv 0 \pmod{p}$.

\medskip
\begin{equation*} A_2. \end{equation*}

The congruences for this case are
\begin{align*}
  \epsilon'xz' + k'xy' &\equiv kx         \pmod{p}, \tag{II} \\
\gamma x'' + \delta z' &\equiv 0          \pmod{p}, \tag{III} \\
           \delta'xz'' &\equiv \delta y'  \pmod{p}, \tag{IV} \\
         \epsilon'xz'' &\equiv \epsilon x \pmod{p}, \tag{V} \\
                  cz'' &\equiv 0          \pmod{p}, \tag{VI} \\
                    ex &\equiv 0          \pmod{p}. \tag{VII}
\end{align*}
\noindent Congruence (III) gives $\gamma \equiv 0$ or $\not\equiv 0$, (IV) gives
$\delta \not\equiv 0$, (V) $\epsilon \not\equiv 0$, (VI) and (VII) give $c
\equiv e \equiv 0 \pmod{p}$. Elimination of $x$, $z'$, and $z''$ between
(II), (III) and (V) gives
\begin{gather*}
\delta k + \gamma\epsilon \equiv k'\delta y' \pmod{p}
\intertext{from which}
\delta k + \gamma\epsilon \not\equiv 0 \pmod{p}.
\end{gather*}
\noindent If $k \equiv 0$, then $\gamma\not\equiv 0$, and if $\gamma \equiv 0$ then
$k \not\equiv 0 \pmod {p}$.

Both $\gamma$ and $k$ can be $\not\equiv 0 \pmod{p}$ provided the above
condition is fulfilled.

\medskip
\begin{equation*} A_5. \end{equation*}

The congruences for this case are
\begin{align*}
    \epsilon'xz'-e'y'z &\equiv kx        \pmod p, \tag{II} \\
\gamma z'' + \delta z' &\equiv 0         \pmod p, \tag{III} \\
           \delta'xz'' &\equiv \delta y' \pmod p, \tag{IV} \\
         \epsilon'xz'' &\equiv ex        \pmod p, \tag{V} \\
                  cz'' &\equiv 0         \pmod p, \tag{VI} \\
               e'y'z'' &\equiv ex        \pmod p. \tag{VII}
\end{align*}
\noindent (II) and (III) are linear in $z$ and $z'$ so $k$ and $\gamma$ are $\equiv
\hbox{ or } \not\equiv 0 \pmod{p}$ independently, (IV) gives $\delta \not
\equiv 0$, (V) give $\epsilon \not\equiv 0$, (VI) $c \equiv 0$, and (VII)
$e \not\equiv 0$.

Elimination between (IV), (V), and (VII) gives
\begin{equation*}
\delta'e'\epsilon^2 \equiv \delta e \epsilon'^2 \pmod{p}.
\end{equation*}

The types are derived by placing $\epsilon = \delta = 1$, and $e = 1, 2,
\cdots, p - 1$.

\begin{equation*} B_5. \end{equation*}

The congruences for this case are
\begin{align*}
                -e'y'z &\equiv kx        \pmod{p}, \tag{II} \\
\gamma z'' + \delta z' &\equiv 0         \pmod{p}, \tag{III} \\
           \delta'xz'' &\equiv \delta y' \pmod{p}, \tag{IV} \\
\epsilon'_1 xz'' + \delta'e'x\tbinom{z''}{2} + e'yz''
                       &\equiv e_1 x + \gamma x'' + \delta x'
                                         \pmod{p}, \tag{V} \\
                  cz'' &\equiv 0         \pmod{p}, \tag{VI} \\
               e'y'z'' &\equiv ex        \pmod{p}. \tag{VII}
\end{align*}
\noindent (II), and (III) being linear in $z$ and $z'$ give $k \equiv 0 \hbox{ or }
\not\equiv 0$, and $\gamma \equiv 0 \hbox{ or } \not\equiv 0 \pmod{p}$,
(IV) gives $\delta \not\equiv 0$, (V) being linear in $x'$ gives
$\epsilon_1 \equiv 0 \hbox{ or } \not\equiv 0 \pmod{p}$, (VI) gives $c
\equiv 0$ and (VII) $e \not\equiv 0$.

Elimination of $x$ and $y'$ from (IV) and (VII) gives
\begin{equation*}
\delta'e'z''^2 \equiv \delta e \pmod{p}.
\end{equation*}

$\delta e$ is a quadratic residue or non-residue $\pmod{p}$ according as
$\delta'e'$ is a residue or non-residue.

The two types are given by placing $\delta = 1$, and $e = 1$ and a
non-residue $\pmod{p}$.

\bigskip
\begin{center}
\textit{Section} 3.
\end{center}
\setcounter{equation}{0}

1. \textit{Groups with dependent generators continued.} As in Section 2, $G$
is here generated by $Q_1$ and $P$, where
\begin{equation*}
Q{}_1^{p^2} = P^{hp^2}.
\end{equation*}
\noindent $Q{}_1^p$ is contained in the subgroup $H_1$ of order $p^{m-2}$, $H_1
= \{Q{}_1^p, P\}$.

\medskip
2. \textit{Determination of $H_1$.} Since $\{P\}$ is self-conjugate in $H_1$
\begin{equation}
Q{}_1^{-p}\, P\, Q{}_1^p = P^{1 + kp^{m-4}}. %% 1
\end{equation}
\noindent Denoting $Q{}_1^a\, P^b\, Q{}_1^c\, P^d \cdots$ by the symbol $[a, b, c, d,
\cdots]$, we have from (1)
\begin{equation}
[-yp,\, x,\, yp] = [0,\, x(1 + kyp^{m-4})] \qquad (m > 4). %% 2
\end{equation}
\noindent Repeated multiplication with (2) gives
\begin{equation}
[yp, x]^s = \Bigl[syp, x\bigl\{s + k\tbinom{s}{2}yp^{m-4}\bigr\}\Bigr]. %% 3
\end{equation}

\medskip
3. \textit{Determination of $H_2$.} There is a subgroup $H_2$ of order $p^{m-1}$
which contains $H_1$ self-conjugately.\footnote{\textsc{Burnside}, \textit{Theory of
Groups}, Art.\ 54, p.\ 64.} $H_2$ is generated by $H_1$ and some operator
$R_1$ of $G$. $R{}_1^p$ is contained in $H_1$, in fact in $\{P\}$,
since if $R{}_1^{p^2}$ is the first power of $R_1$ in $\{P\}$, then $H_2
= \{R_1, P\}$, which case was treated in Section 1.
\begin{equation}
R{}_1^p = P^{lp}. %% 4
\end{equation}

Since $H_1$ is self-conjugate in $H_2$
\begin{align}
R{}_1^{-1}\, P\, R_1   &= Q{}_1^{\beta p} P^{\alpha_1}, \\
R{}_1^{-1}\, Q^p\, R_1 &= Q{}_1^{bp} P^{\alpha_1 p}.
\end{align}

Using the symbol $[a, b, c, d, e, f, \cdots]$ to denote $R{}_1^a\, Q{}_1^b\,
P^c\, R{}_1^d\, Q{}_1^e\, P^f \cdots$, we have from (5), (6) and (3)
\begin{equation}
[-p,\, 0,\, 1,\, p] = [0,\, \beta Np,\, \alpha{}_1^p + Mp], %% 7
\end{equation}
\noindent and by (4)
\begin{equation*}
\alpha{}_1^p \equiv 1 \pmod{p}, \qquad \hbox{ and } \qquad \alpha_1 \equiv 1 \pmod{p}.
\end{equation*}

Let $\alpha_1 = 1 + \alpha_2 p$ and (5) is now replaced by
\begin{equation}
R{}_1^{-1}\, P\, R_1 = Q{}_1^{\beta p} P^{1 + \alpha_2 p}.
\end{equation}

From (6), (8) and (3)
\begin{gather*}
[-p,\, p,\, 0,\, p] = \bigl[0,\, b^p p,\, a_1 \frac{b^p - 1}{b - 1}p + a_1 Up^2\bigr],
\intertext{and by (4) and (2)}
R{}_1^{-p}\, Q{}_1^p\, R{}_1^p = Q{}_1^p
\end{gather*}
\noindent and therefore $b^p \equiv 1 \pmod{p}$, and $b = 1$. Placing
$b = 1$ in the above equation the exponent of $P$ takes the form
\begin{gather*}
a_1 p^2 (1 + U'p) = a_1 \frac{\left\{1 + (\alpha_2 + \beta h)p\right\}^p
- 1}{(\alpha_2 + \beta h)p^2}p^2
\intertext{from which}
a_1 p^2 (1 + U'p) \equiv 0 \pmod{p^{m-3}}
\intertext{or}
a_1 \equiv 0 \pmod{p^{m-5}} \quad (m > 5).
\end{gather*}

Let $a_1 = ap^{m-5}$ and (6) is replaced by
\begin{equation}
R{}_1^{-1}\, Q{}_1^p\, R_1 = Q{}_1^p\, P^{ap^{m-4}}. %% 9
\end{equation}

(7) now has the form
\begin{gather*}
[-p,\, 0,\, 1,\, p] = [0,\, \beta Np,\, (1 + \alpha_2 p)^p + Mp^2],
\intertext{where}
N = p \quad \hbox{ and } \quad M = \beta h \left\{ \frac{(1 + \alpha_2 p)^p - 1}
   {\alpha_2 p^2} -1 \right\},
\intertext{from which}
(1 + \alpha_2 p)^p + \frac{(1 + \alpha_2 p)^p - 1}{\alpha_2 p^2}\beta hp^2
   \equiv 1 \pmod{p^{m-3}}
\intertext{or}
\frac{(1 + \alpha_2 p)^p - 1}{\alpha_2 p^2}\{\alpha_2 + \beta h\}p^2
   \equiv 0 \pmod{p^{m-3}}
\intertext{and since}
\frac{(1 + \alpha_2 p)^p - 1}{\alpha_2 p^2} \equiv 1 \pmod{p}
\end{gather*}
\begin{equation}
(\alpha_2 + \beta h)p^2 \equiv 0 \pmod{p^{m-3}}. %% 10
\end{equation}

From (8), (9), (10) and (3)
\begin{align}
[-y,\,  0,\, x,\, y] &= [0,\, \beta xyp,\, x(1+\alpha_2 yp)+\theta p^{m-4}], \\ %% 11
[-y,\, xp,\, 0,\, y] &= [0,\, xp,\, axyp^{m-4}], %% 12
\end{align}
\noindent where
\begin{equation*}
\theta = a \beta x \tbinom{y}{2} + \beta k \tbinom{x}{2} y.
\end{equation*}

By continued use of (11), (12), (2) and (10)
\begin{equation}
[z,\, yp,\, x]^s = [sz,\, (sy + U_s)p,\, xs+ V_s p], %% 13
\end{equation}
\noindent where
\begin{align*}
U_s &= \beta \tbinom{s}{2} xz \\
V_s &= \tbinom{s}{2} \Bigl\{ \alpha_2 xz + \bigl[ayz + kxy + \beta k\tbinom{x}{2}z \bigr]
  p^{m-5} \Bigr\} \\ & \qquad \qquad +\Bigl\{\beta\tbinom{s}{3}x^2 z + \frac{1}{2}a\beta
  \Bigl[\frac{1}{3!} s(s - 1)(2s - 1)z^2 - \tbinom{s}{2}z\Bigr]x \Bigr\}p^{m-5}.
\end{align*}

Placing in this $y = 0$, $z = 1$ and $s = p$,\footnote{Terms of the form
$(Ax^2 + Bx)p^{m-4}$ in the exponent of $P$ for $p = 3$ and $m > 5$ do not
alter the result.}
\begin{gather*}
(R_1\, P^x)^p = R{}_1^p\, P^{xp} = P^{(x+l)p},
\intertext{determine $x$ so that}
x + l \equiv 0 \pmod{p^{m-4}},
\end{gather*}
then $R = R_1 P^x$ is an operator of order $p$ which will be used in the
place of $R_1$, $R^p = 1$.

\medskip
4. \textit{Determination of $G$.} Since $H_2$ is self-conjugate in $G$
\begin{align}
Q{}_1^{-1}\, P\, Q_1 &= R^\gamma\, Q{}_1^{\delta p}\, P^{\epsilon_1}, \\ %% 14
Q{}_1^{-1}\, R\, Q_1 &= R^c\, Q{}_1^{dp}\, P^{e_1 p}. %% 15
\end{align}

From (15)
\begin{gather*}
(R^c\, Q{}_1^{dp}\, P^{e_1 p})^p = 1,
\intertext{by (13)}
Q{}_1^{dp^2}\, P^{e_1 p^2} = P^{(e_1 + dh)p^2} = 1,
\end{gather*}
\noindent and
\begin{equation}
(e_1+ dh)p^2 \equiv 0 \pmod{p^{m-3}}. %% 16
\end{equation}

From (14), (15) and (13)
\begin{equation}
[0,\, -p,\, 1,\, 0,\, p] = [L,\, Mp,\, \epsilon_1^p + Np]. %% 17
\end{equation}

By (1)
\begin{equation*}
\epsilon{}_1^p \equiv 1 \pmod{p}, \qquad \hbox{ and } \qquad \epsilon_1 \equiv 1 \pmod{p}.
\end{equation*}

Let $\epsilon_1 = 1 + \epsilon_2 p$ and (14) is replaced by
\begin{equation}
Q{}_1^{-1}\, P\, Q_1 = R^\gamma\, Q{}_1^{\delta p}\, P^{1 + \epsilon_2 p}. %% 18
\end{equation}

From (15), (18), and (13)
\begin{equation*}
[0,\, -p,\, 0,\, 1,\, p] = \left[c_p,\, \frac{c^p - 1}{c - 1}dp,\, Kp \right].
\end{equation*}

Placing $x = 1$ and $y =-1$ in (12) we have
\begin{equation}
[0,\, -p,\, 0,\, 1,\, p] = [1,\, 0,\, -ap^{m-4}], %% 19
\end{equation}
and therefore $c^p \equiv 1 \pmod{p}$, and $c = 1$. (15) is now replaced by
\begin{equation}
Q{}_1^{-1}\, R\, Q_1 = R\, Q{}_1^{dp}\, P^{e_1 p}. %% 20
\end{equation}

Substituting $1 + \epsilon_2 p$ for $\epsilon_1$ and 1 for $c$ in (17)
gives, by (16)
\begin{gather*}
[0,\, -p,\, 1,\, p] = [0,\, Mp^2,\, (1 + \epsilon_2 p)^p + Np^2],
\intertext{where}
M = \gamma d \frac{p-1}{2} + \delta\frac{(1 + \epsilon_2 p)^p - 1}{\epsilon_2 p^2}
\intertext{and}
N = \frac{e_1\gamma}{(\epsilon_2 + \delta h)p^2} \left\{\frac{[1 + (\epsilon_2 +
\delta h)p]^p - 1}{(\epsilon_2 + \delta h)p} - p \right\}.
\end{gather*}

By (1)
\begin{gather*}
(1 + \epsilon_2 p)^p + (N + Mh)p^2 \equiv 1 + kp^{m-4} \pmod{p^{m-3}},
\intertext{or reducing}
\psi(\epsilon_2 + \delta h)p^2 \equiv kp^{m-4} \pmod{p^{m-3}},
\intertext{where}
\psi = \frac{(1 + \epsilon_2 p)^p - 1}{\epsilon_2 p^2} + N -
   e_1 \gamma \frac{p-1}{2}.
\end{gather*}

Since
\begin{equation*}
\psi = 1 \pmod{p}.
\end{equation*}
\begin{equation}
(\epsilon_2 + \delta h)p^2 \equiv kp^{m-4} \pmod{p^{m-3}}. %% 21
\end{equation}

From (18), (20), (13), (16) and (21)
\begin{align}
[0,\, -y,\, x,\, 0,\, y] &= [\gamma xy,\, \theta_1 p,\, x + \phi_1 p], \\ %% 22
[0,\, -y,\, 0,\, x,\, y] &= [x,\, dxyp,\, \phi_2 p], %% 23
\end{align}
\noindent where
\begin{align*}
\theta_1 &= d\gamma x\tbinom{y}{2} + \delta xy + \beta\gamma\tbinom{x}{2}y, \\
\phi_1 &= \epsilon_2 xy + \alpha_2\gamma\tbinom{x}{2}y + e_1\gamma
  \tbinom{y}{2}x + \bigl\{x\tbinom{y}{2}(\epsilon_2 k + \delta \gamma) \\
& \qquad + \frac{1}{2}ad\left[\frac{1}{3!}y(y - 1)(2y - 1)\gamma^2 -
  \frac{y}{2}\gamma\right]x + a\gamma^2 dx\frac{1}{3!}y(y + 1)(y - 1) \\
& \qquad + e_1\gamma k\tbinom{y}{3}x + \frac{1}{2}a\beta\left[\frac{1}{3!}x(x - 1)(2x - 1)\gamma^2 y^2
  - \tbinom{x}{2}\gamma y\right] \\
& \qquad \qquad + \tbinom{x}{2}(a + k)\left[dy\tbinom{y}{2}
  + \delta y\right] + \beta\gamma \tbinom{x}{3}\bigr\}p^{m-5}, \\
\phi_2 &= e_1 xy + \left\{e_1 k\tbinom{y}{2} + ad\tbinom{x}{2}y \right\}p^{m-5}.
\end{align*}

Placing $x = 1$ and $y = p$ in (23) and by (16)
\begin{gather*}
Q{}_1^{-p}\, R\, Q{}_1^p = R,
\intertext{and by (19)}
a \equiv 0 \pmod{p}.
\end{gather*}

A continued multiplication, with (11), (22), and (23), gives
\begin{gather*}
(Q_1\, P^x)^{p^2} = Q{}_1^{p^2}\, P^{xp^2} = P^{(x + l)p^2}.
\intertext{Let $x$ be so chosen that}
(x + l) \equiv 0 \pmod{p^{m-5}},
\end{gather*}
\noindent then $Q = Q_1\, P^x$ is an operator of order $p^2$ which will be used in
place of $Q_1$, $Q^{p^2} = 1$ and
\begin{equation*}
h \equiv 0 \pmod{p^{m-5}}.
\end{equation*}

From (21), (10) and (16)
\begin{equation*}
\epsilon_2 p^2 \equiv kp^{m-4}, \qquad \alpha_2 p^2 \equiv 0 \qquad \hbox{ and }
  \qquad e_1 p^2 \equiv 0 \pmod{p^{m-3}}.
\end{equation*}
\noindent Let $\epsilon_2 = \epsilon p^{m-6}$, $\alpha_2 = \alpha p^{m-5}$ and
$e_1 = ep^{m-5}$. Then equations (18), (20) and (8) are replaced by
\begin{gather*}
G: \left\{ \begin{aligned}
Q^{-1}\, P\, Q &= R^\gamma\, Q^{\delta p}\, P^{1 + \epsilon p^{m-5}}, \\
Q^{-1}\, R\, Q &= R\, Q^{dp}\, P^{ep^{m-4}}, \\
R^{-1}\, P\, R &= Q^{\beta p}\, P^{1 + \alpha p^{m-4}}, \\ \end{aligned} \right.
\tag*{(24), (25), (26)} \\
R^p = 1, \qquad Q^{p^2} = 1, \qquad P^{p^{m-3}} = 1.
\end{gather*}

\setcounter{equation}{26}
(11), (22) and (23) are replaced by
\begin{align}
     [-y,\, 0,\, x,\, y] &= [0,\, \beta xyp,\, x + \phi p^{m-4}], \\ %% 27
[0,\, -y,\, x,\, 0,\, y] &= [\gamma xy,\, \theta_1 p,\, x + \phi_1 p^{m-5}], \\ %% 28
[0,\, -y,\, 0,\, x,\, y] &= [x,\, dxyp,\, \phi_2 p^{m-4}],                      %% 29
\end{align}
\noindent where
\begin{gather*}
\phi = \alpha xy + \beta k\tbinom{x}{2}y, \quad
  \theta_1 = d\gamma\tbinom{y}{2}x + \delta xy + \beta\gamma\tbinom{x}{2}y, \\
\phi_1 = exy + \left\{e\gamma x\tbinom{y}{2} + \tbinom{x}{2}\left(\alpha\gamma y +
  d\gamma k\tbinom{y}{2} + \delta ky\right) + \beta\gamma y\tbinom{x}{3}\right\}p, \\
\phi_2 = exy.
\end{gather*}
\noindent A formula for a general power of any operator of $G$ is derived from (27),
(28) and (29)
\begin{equation}
[0,\, z,\, 0,\, y,\, 0,\, z]^s = [0,\, sz + U_s p,\, 0,\, sy + V_s,\, 0,\, sx + W_s p^{m-5}], %% 30
\end{equation}
\noindent where
\begin{align*}
U_s &= \tbinom{s}{2}\left\{\delta xz + dyz + \beta xy + \beta\gamma \tbinom{x}{2}z \right\} \\
 & \qquad + \frac{1}{2}dx \left\{ \frac{1}{3!}s(s - 1)(2s - 1)z^2 - \tbinom{s}{2}z \right\}x
   + \beta\gamma\tbinom{s}{2}x^2 z, \\
V_s &= \gamma\tbinom{s}{2}xz, \displaybreak \\
W_s &= \tbinom{s}{2} \left\{\epsilon xz + \left[axy + eyz + (\beta ky + \alpha\beta
  \gamma + \delta kz)\tbinom{x}{2}\right]p \right\} \\
 & \qquad + \tbinom{s}{3}\left\{\alpha\gamma x^2 z + dkxyz + \delta kx^2 z +
   \beta kx^2 y + 2\beta\gamma k\tbinom{x}{2}xz\right\}p \\
 & \qquad + \beta yk\tbinom{s}{4}x^3 zp + \frac{1}{2}\left\{\frac{1}{3!}s(s - 1)(2s - 1)z^2
   - \frac{s}{2}z \right\} \left\{e\gamma x + d\gamma k\tbinom{x}{2}\right\}p \\
 & \qquad + \frac{1}{2}d\gamma k\left[\frac{1}{2}(s - 1)z^2 - z\right]\tbinom{s}{3}x^2.
\end{align*}
\noindent A comparison of the generational equations of the present section with those
of Sections 1 and 2, shows that, $\gamma \equiv 0 \pmod{p}$ gives groups
simply isomorphic with those of Section 1, while $\beta \equiv 0 \pmod{p}$,
groups simply isomorphic with those of Section 2 and we need consider only
the groups in which $\beta$ and $\gamma$ are prime to $p$.

\medskip
5. \textit{Transformation of the groups.} All groups of this section are given by
equations (24), (25), and (26), where $\gamma, \beta = 1, 2, \cdots, p - 1$;
$\alpha, \delta, d, e = 0, 1, 2, \cdots, p - 1$; and $\epsilon = 0, 1,
2, \cdots, p^2 - 1$.

Not all of these, however are distinct. Suppose that $G$ is simply
isomorphic with $G'$ and that the correspondence is given by
\begin{equation*}
C = \left[\begin{matrix}R,    & Q,    & P \\
                        R'_1, & Q'_1, & P'_1 \\ \end{matrix} \right].
\end{equation*}
\noindent An inspection of (30) gives
\begin{align*}
R'_1 &= Q'^{z''p}\, R'^{y''}\, P'^{x''p^{m-4}}, \\
Q'_1 &= Q'^{z'}\, R'^{y'}\, P'^{x'p^{m-5}}, \\
P'_1 &= Q'^z\, R'^y\, P'^x,
\end{align*}
\noindent with $dv[x,\, p] = 1$. Since $Q^p$ is not in $\{P\}$, and $R$ is not in
$\{Q^p, P\}$, ${Q'}_1^p$ is not in $\{P'_1\}$ and $R'_1$ is not in
$\{{Q'}_1^p, P'_1\}$. Let
\begin{gather*}
{Q'}_1^{s'p} = {P'}_1^{sp^{m-4}}.
\intertext{This is in terms of $R'$, $Q'$, and $P'$,}
[0,\, s'z'p,\, s'x'p^{m-4}] = [0,\, 0,\, sxp^{m-4}].
\intertext{From which}
s'z'p \equiv 0 \pmod{p^2},
\intertext{and $z'$ must be prime to $p$, since otherwise $s' \hbox{ can } = 1$. Let}
{R'}_1^{s''} = {Q'}_1^{s'p}\, {P'}_1^{sp^{m-4}},
\intertext{or in terms of $R'$, $Q'$, and $P'$,}
[s''y'',\, s''z''p,\, s''x''p^{m-4}] = [0,\, s'z'p,\, (sx + s'x')p^{m-4}]
\intertext{and}
s''z'' \equiv s'z' \pmod{p}, \qquad s''y'' \equiv 0 \pmod{p},
\end{gather*}
\noindent and $y''$ is prime to $p$, since otherwise $s''$ can $= 1$. Since $R$,
$Q$, and $P$ satisfy equations (24), (25) and (26), $R'_1$, $Q'_1$, and
$P'_1$ must also satisfy them. These become when reduced in terms of $R'$,
$Q'$ and $P'$
\begin{align*}
[0,&\, z + \theta'_1 p,\, 0,\, y + \gamma'xz',\, 0,\, x + \psi'_1 p^{m-5}] \\
  & \qquad \qquad \qquad \qquad \qquad = [0,\, z + \theta_1 p,\, 0,\, y + \gamma y'',\, 0,\, x + \psi_1 p^{m-5}], \\
[0,&\, (z'' + \theta'_2)p,\, 0,\, y'',\, 0,\, (x'' + \psi_2)p^{m-4}] \\
  & \qquad \qquad \qquad \qquad \qquad = [0,\, (z'' + \theta_2)p,\, 0,\, y'',\, 0,\, (x'' + \psi_2)p^{m-4}], \\
[0,&\, z + \theta'_3p,\, 0,\, y,\, 0,\, x + \psi'_3 p^{m-4}]
  = [0,\, z + \theta_3p,\, 0,\, y,\, 0,\, x + \psi_3 p^{m-4}],
\end{align*}
\noindent where
\begin{align*}
\theta'_1 &= d'(yz' - y'z) + x\left\{d'\gamma'\tbinom{z'}{2} + \delta'z' +
\beta'y'\right\} + \beta'\gamma'\tbinom{x}{2}z', \\
\theta_1 &= \gamma z'' + \delta z' + d'\gamma y''z, \\
\psi'_1 &= \epsilon'xz' + \bigl\{e'\gamma'x\tbinom{z'}{2} + \tbinom{x}{2}\left[\alpha'
  \gamma'z' + \gamma'\epsilon'd'k'\tbinom{z'}{2} + \delta'\epsilon k'z' +
  \beta'k'y'\right] \\
 & \qquad + \beta'\gamma'\tbinom{x}{3}z' + e'(yz' - y'z) + \alpha'xy'\bigr\}p, \\
\psi_1 &= \epsilon x + \{\delta x' + \gamma x'' + e'\gamma y''z\}p, \\
\theta'_2 &= d'y''z', \qquad \theta_2 = dz', \qquad \psi'_2 = e'y''z, \qquad
  \psi_2 = dx' + ex, \\
\theta'_3 &= \beta'xy'' - d'y''z, \qquad \theta_3 = \beta z', \\
\psi_3 &= \epsilon'xz'' - e'y''z + \alpha'xy'' + \beta'\epsilon'\tbinom{x}{2}y'',
  \qquad \psi_3 = \alpha x + \beta x'.
\end{align*}

A comparison of the two sides of these equations give seven congruences
\begin{align*}
\theta'_1  &\equiv \theta_1   \pmod{p},   \tag{I} \\
\gamma'xz' &\equiv \gamma y'' \pmod{p},   \tag{II} \\
\psi'_1    &\equiv \psi_1     \pmod{p^2}, \tag{III} \\
\theta'_2  &\equiv \theta_2   \pmod{p},   \tag{IV} \\
\psi'_2    &\equiv \psi_2     \pmod{p},   \tag{V} \\
\theta'_3  &\equiv \theta_3   \pmod{p},   \tag{VI} \\
\psi'_3    &\equiv \psi_3     \pmod{p}.   \tag{VII}
\end{align*}

(VI) is linear in $z$ provided $d' \not\equiv 0 \pmod{p}$ and $z$ may be
so determined that $\beta \equiv 0 \pmod{p}$ and therefore all groups in
which $d' \not\equiv 0 \pmod{p}$ are simply isomorphic with groups in
Section 2.

Consequently we need only consider groups in which $d' \equiv 0 \pmod{p}$.

As before we take for $G'$ the simplest case and associate with it all
simply isomorphic groups $G$. We then take as $G'$ the simplest case left
and proceed as above.

Let $\kappa = \kappa_1 p^{\kappa_2}$ where $dv[\kappa_1, p] = 1, (\kappa =
\alpha, \beta, \gamma, \delta, \epsilon, d, e)$.

For convenience the groups are divided into three sets and each set is
subdivided into eight cases.

The sets are given by
\begin{equation*}
\begin{matrix}
A: & \epsilon_2 = 0, & \beta_2 = 0, & \gamma_2 = 0, \\
B: & \epsilon_2 = 1, & \beta_2 = 0, & \gamma_2 = 0, \\
C: & \epsilon_2 = 2, & \beta_2 = 0, & \gamma_2 = 0.
\end{matrix}
\end{equation*}

The subdivision into cases and results of the discussion are given in
Table I.

\begin{center}
\large I. \normalsize

\smallskip
\begin{tabular}{|c|c|c|c|c|c|c|} \hline
   & $\delta_2$ & $e_2$ & $\alpha_2$ &  $A$  &  $B$  &  $C$  \\ \hline
 1 &     1      &   1   &      1     &       &       & $B_1$ \\ \hline
 2 &     0      &   1   &      1     & $A_1$ & $B_1$ & $B_1$ \\ \hline
 3 &     1      &   0   &      1     &       &       & $B_3$ \\ \hline
 4 &     1      &   1   &      0     & $A_1$ & $B_1$ & $B_1$ \\ \hline
 5 &     0      &   0   &      1     & $A_3$ & $B_3$ & $B_3$ \\ \hline
 6 &     0      &   1   &      0     & $A_1$ & $B_1$ & $B_1$ \\ \hline
 7 &     1      &   0   &      0     & $A_3$ & $B_3$ & $B_3$ \\ \hline
 8 &     0      &   0   &      0     & $A_3$ & $B_3$ & $B_3$ \\ \hline
\end{tabular}
\end{center}

\medskip
6. \textit{Reduction to types.} The types of this section are given by equations
(24), (25) and (26) with $\alpha = 0, \beta = 1, \lambda = 1$ or a
quadratic non-residue (mod $p$), $\delta \equiv 0; \epsilon = l, e = 0, 1,
2, \cdots, p - 1;$ and $\epsilon = p, e = 0, 1,$ or a
non-residue (mod $p$), $2p+6$ in all.

The special forms of the congruences for these cases are given below.

\medskip
\begin{equation*} A_1. \end{equation*}
\begin{align*}
\beta'\gamma'\tbinom{x}{2}z' + \beta'xy'
             &\equiv \gamma z'' + \delta z' \pmod{p}, \tag{I} \\
  \gamma'xz' &\equiv \gamma y'' \pmod{p},             \tag{II} \\
\epsilon'xz' &\equiv \epsilon x \pmod{p},             \tag{III} \\
         dz' &\equiv 0 \pmod{p},                      \tag{IV} \\
          ex &\equiv 0 \pmod{p},                      \tag{V} \\
  \beta'xy'' &\equiv \beta z' \pmod{p},               \tag{VI} \\
\epsilon'xz'' + \beta'\epsilon'\tbinom{x}{2}y'
             &\equiv \alpha x + \beta x' \pmod{p}.    \tag{VII}
\end{align*}

(I) is linear in $z''$ and $\delta \equiv 0$ or $\not\equiv 0$, (II)
gives $\gamma \not\equiv 0$, (III) $\epsilon \not\equiv 0$, (IV) and (V)
$d \equiv e \equiv 0$, (VI) $\beta \not\equiv 0$, (VII) is linear in $x'$
and $\alpha \equiv 0$ or $\not\equiv 0 \pmod{p}$.

Elimination of $y''$ and $z'$ between (II) and (VI) gives
\begin{equation*}
\beta'\gamma'x^2 \equiv \beta\gamma \pmod{p}
\end{equation*}
\noindent and $\beta\gamma$ is a residue or non-residue $\pmod{p}$ according as
$\beta'\gamma'$ is a residue or non-residue.

\medskip
\begin{equation*} A_3. \end{equation*}
\begin{align*}
\beta'\gamma'\tbinom{x}{2}z' + \beta'xy'
            &\equiv \gamma z'' + \delta z' \pmod{p}, \tag{I} \\
 \gamma'xz' &\equiv \gamma y'' \pmod{p},             \tag{II} \\
\epsilon'z' &\equiv \epsilon \pmod{p},               \tag{III} \\
          d &\equiv 0 \pmod{p},                      \tag{IV} \\
    e'y''z' &\equiv ex \pmod{p},                     \tag{V} \\
 \beta'xy'' &\equiv \beta z' \pmod{p},               \tag{VI} \\
\epsilon'xz'' - e'y''z + \beta'\epsilon'\tbinom{x}{2}y'
            &\equiv \alpha x + \beta x' \pmod{p}.    \tag{VII}
\end{align*}

(I) is linear in $z''$ and $\delta \equiv 0$ or $\not\equiv 0$. (II)
gives $\gamma \not\equiv 0$, (III) $\epsilon \not\equiv 0$, (V) $e \not
\equiv 0$ and (VI) $\beta \not\equiv 0$. (VII) is linear in $x'$ and
$\alpha \equiv 0$ or $\not\equiv 0 \pmod{p}$.

Elimination between (II) and (VI) gives
\begin{gather*}
\beta'\gamma'x^2 \equiv \beta\gamma \pmod{p},
\intertext{and between (II), (III), and (IV) gives}
\epsilon'^2 \gamma e \equiv \epsilon^2\gamma'e' \pmod{p}.
\end{gather*}

$\beta\gamma$ is a residue, or non-residue, according as $\beta'\gamma'$ is
or is not, and if $\gamma$ and $\epsilon$ are fixed, $e$ must take the
$(p - 1)$ values $1, 2, \cdots, p - 1$.

\medskip
\begin{equation*} B_1. \end{equation*}
\begin{align*}
\beta'\gamma'\tbinom{x}{2}z' + \beta'xy'
           &\equiv \gamma z'' + \delta z' \pmod{p}, \tag{I} \\
\gamma'xz' &\equiv \gamma y'' \pmod{p},             \tag{II} \\
\epsilon'_1 xz' + \beta'xz'\tbinom{x}{3}
           &\equiv \epsilon_1 x + \delta x' + \gamma x'' \pmod{p}, \tag{III} \\
        ex &\equiv 0 \pmod{p},                      \tag{IV} \\
\beta'xy'' &\equiv \beta z' \pmod{p},               \tag{VI} \\
\alpha x + \beta x' &\equiv 0 \pmod{p}.             \tag{VII}
\end{align*}

(I) gives $\delta \equiv 0$ or $\not\equiv 0$, (II) $\gamma \not
\equiv 0$, (III) is linear in $x''$ and gives $\epsilon_1 \equiv 0$
or $\not\equiv 0$, (V) $e = 0$, (VI) $\beta \not\equiv 0$ and
(VII) is linear in $x'$ and gives $\alpha \equiv 0$ or $\not\equiv 0$.

Elimination between (II) and (VI) gives
\begin{equation*}
\beta'\gamma'x^2 \equiv \beta\gamma \pmod{p}.
\end{equation*}

\newpage
\begin{equation*} B_3. \end{equation*}
\begin{align*}
\beta'\gamma'\tbinom{x}{2}z' + \beta'xy'
                &\equiv \gamma\beta'' + \delta z' \pmod{p}, \tag{I} \\
     \gamma'xz' &\equiv \gamma y' \pmod{p},                 \tag{II} \\
 \epsilon'_1 xz' + e'\gamma'x\tbinom{z'}{2} + \beta'\gamma'\tbinom{x}{3} &+ e'(yz' - y'z) \\
                &\equiv \epsilon_1 x + \delta x' + \gamma x'' + e'\gamma zy'' \pmod{p},
                                                            \tag{III} \\
        e'y''z' &\equiv ex \pmod{p}.                        \tag{V} \\
     \beta'xy'' &\equiv \beta z' \pmod{p},                  \tag{VI} \\
        -e'y''z &\equiv \alpha x + \beta x' \pmod{p}.       \tag{VII} \\
\end{align*}
(I) gives $\delta\equiv 0$ or $\not\equiv 0$, (II) $\gamma \not
\equiv 0$, (III) is linear in $x''$ and gives $\epsilon_1 \equiv 0$ or
$\not\equiv 0$, (V) $e \not\equiv 0$, (VI) $\beta \not\equiv 0$, (VII) is
linear in $x'$ and gives $\alpha \equiv 0$ or $\not\equiv 0 \pmod{p}$.
Elimination of $y''$ and $z'$ between (II) and (VI) gives
\begin{gather*}
\beta'\gamma'x^2 \equiv \beta\gamma \pmod{p},
\intertext{and between (V) and (VI) gives}
\beta'e'y''^2 \equiv \beta e \pmod{p}
\end{gather*}
\noindent and $\beta\gamma$ and $\beta e$ are residues or non-residues, independently,
according as $\beta'\gamma'$ and $\beta'e'$ are residues or non-residues.

\bigskip \bigskip
\begin{center}
\Large\textit{Class} III.\normalsize
\end{center}
\setcounter{equation}{0}

1. \textit{General relations.} In this class, the $p$th power of every operator of
$G$ is contained in $\{P\}$. There is in $G$ a subgroup $H_1$ of order
$p^{m-2}$, which contains $\{P\}$ self-conjugately.\footnote{\textsc{Burnside},
\textit{Theory of Groups}, Art.\ 54, p.\ 64.}

\medskip
2. \textit{Determination of $H_1$.} $H_1$ is generated by $P$ and some operator
$Q_1$ of $G$.
\begin{equation*}
Q{}_1^p = P^{hp}.
\end{equation*}
\noindent Denoting $Q{}_1^a\, P^b\, Q{}_1^c\, P^d \cdots$ by the symbol $[a, b, c, d,
\cdots]$, all operators of $H_1$ are included in the set $[y, x]; (y = 0,
1, 2, \cdots, p - 1, x = 0, 1, 2, \cdots, p^{m-3} - 1)$.

Since $\{P\}$ is self-conjugate in $H_1$\footnote{\textit{Ibid.}, Art.\ 56, p.\ 66.}
\begin{gather}
Q{}_1^{-1}\, P\, Q_1 = P^{1 + kp^{m-4}}. %% 1
\intertext{Hence}
[-y,\, x,\, y] = [0,\, x(1 + kyp^{m-4})] \qquad (m > 4). %% 2
\intertext{and}
[y,\, x]^s = \left[sy, x\left\{s + ky \tbinom{s}{2}p^{m-4} \right\}\right]. %% 3
\end{gather}
\noindent Placing $y = 1$ and $s = p$ in (3), we have,
\begin{gather*}
[Q_1\, P^x]^p = Q{}_1^p\, P^{xp} = P^{(x + h)p}
\intertext{and if $x$ be so chosen that}
(x + h) \equiv 0 \pmod{p^{m-4}},
\end{gather*}
\noindent $Q = Q_1\, P^x$ will be an operator of order $p$ which will be used in place
of $Q_1$, $Q^p = 1$.

\medskip
3. \textit{Determination of $H_2$.} There is in $G$ a subgroup $H_2$ of order
$p^{m-1}$, which contains $H_1$ self-conjugately. $H_2$ is generated by
$H_1$, and some operator $R_1$ of $G$.
\begin{equation*}
R{}_1^p = P^{lp}.
\end{equation*}

We will now use the symbol $[a, b, c, d, e, f, \cdots]$ to denote $R{}_1^a$ $Q^b$
$P^c$ $R{}_1^d$ $Q^e$ $P^f$ $\cdots$.

The operations of $H_2$ are given by $[z, y, x];$ $(z, y = 0, 1, \cdots, p - 1$;
$x = 0, 1, \cdots, p^{m-3} - 1)$. Since $H_1$ is self-conjugate in $H_2$
\begin{align}
R{}_1^{-1}\, P\, R_1 &= Q{}_1^\beta P^{\alpha_1}, \\ %% 4
R{}_1^{-1}\, Q\, R_1 &= Q{}_1^{b_1} P^{\alpha p^{m-4}}.      %% 5
\end{align}

From (4), (5) and (3)
\begin{gather*}
[-p,\, 0,\, 1,\, p] = \left[0,\, \frac{\alpha{}_1^p - b{}_1^p}{\alpha_1 - b_1}\beta,\,
  \alpha{}_1^p + \theta p^{m-4} \right] = [0,\, 0,\, 1], \\
\intertext{where}
\theta = \frac{\alpha{}_1^p \beta k}{2}\frac{\alpha{}_1^p - 1}{\alpha_1-1}
  + a\beta \left\{\frac{\alpha{}_1^p - 1}{\alpha_1 - b_1}p -
  \frac{\alpha{}_1^p - b{}_1^p}{(\alpha_1 - b_1)^2}\right\}.
\end{gather*}
\noindent Hence
\begin{equation}
\frac{\alpha{}_1^p - b{}_1^p}{\alpha_1 - b_1}\beta \equiv 0 \pmod{p}, \qquad
\alpha{}_1^p + \theta p^{m-4} \equiv 1 \pmod{p^{m-3}},     %% 6
\end{equation}
\noindent and $\alpha{}_1^p \equiv 1 \pmod{p^{m-4}}$, or $\alpha_1 \equiv 1
\pmod{p^{m-5}} \qquad (m > 5)$, $\alpha_1 = 1 + \alpha_2 p^{m-5}.$ Equation (4)
is replaced by
\begin{equation}
R{}_1^{-1}\, P\, R_1 = Q^\beta P^{1 + \alpha_2 p^{m-5}},
\end{equation}

From (5), (7) and (3).
\begin{equation*}
[-p,\, 1,\, 0,\, p] = \left[0,\, b{}_1^p,\, a\frac{b{}_1^p - 1}{b - 1} p^{m-4} \right].
\end{equation*}
\noindent Placing $x = lp$ and $y = 1$ in (2) we have $Q^{-1} P^{lp} Q = P^{lp}$, and
\begin{equation*}
b{}_1^p \equiv 1 \pmod{p}, \qquad a\frac{b{}_1^p - 1}{b_1 - 1} \equiv 0
\pmod{p}.
\end{equation*}
\noindent Therefore, $b_1 = 1$.

Substituting 1 for $b_1$ and $1 + \alpha_2 p^{m-5}$ for $\alpha_1$ in
congruence (6) we find
\begin{equation*}
(1 + \alpha_2 p^{m-5})^p \equiv 1 \pmod{p^{m-3}}, \qquad \hbox{ or } \qquad
\alpha_2 \equiv 0 \pmod{p}.
\end{equation*}

Let $\alpha_2 = \alpha p$ and equations (7) and (5) are replaced by
\begin{align}
R{}_1^{-1}\, P\, R_1 &= Q^\beta P^{1 + \alpha p^{m-4}}, \\ %% 8
R{}_1^{-1}\, Q\, R_1 &= Q P^{\alpha p^{m-4}}.              %% 9
\end{align}

From (8), (9) and (3)
\begin{align}
[-y,\, 0,\, x,\, y] &= \Bigl[0,\, \beta xy,\, x + \bigl\{\alpha xy + a\beta x \tbinom{y}{2} +
  \beta ky\tbinom{x}{2}\bigr\}p^{m-4}\Bigr], \\ %% 10
[-y,\, x,\, 0,\, y] &= [0,\, x,\, axyp^{m-4}].     %% 11
\end{align}

From (2), (10), and (11)
\begin{equation}
[z,\, y,\, x]^s = [sz,\, sy + U_s,\, sx + V_s p^{m-4}], %% 12
\end{equation}
\noindent where
\begin{align*}
U_s &= \beta \tbinom{s}{2}xz, \\
V_s &= \tbinom{s}{2}\left\{\alpha xz + kxy + ayz + \beta k\tbinom{x}{2}z\right\} \\
  & \qquad \qquad + \beta k\tbinom{s}{3}x^2 z + \frac{1}{2}a\beta\tbinom{s}{2} \left\{\frac{1}{3!}
  (2s - 1)z - 1\right\}xz.
\end{align*}

Placing $z = 1$, $y = 0$, and $s = p$ in (12)\footnote{The terms of the form
$(Ax + Bx^2)p^{m-4}$ which appear in the exponent of $P$ for $p = 3$
do not alter the conclusion for $m > 5$.}
\begin{equation*}
[R_1\, P^x]^p = R{}_1^p\, P^{xp} = P^{(x+l)p}.
\end{equation*}

If $x$ be so chosen that
\begin{equation*}
x + l \equiv 0 \pmod{p^{m-4}}
\end{equation*}
\noindent then $R =R_1 P^x$ is an operator of order $p$ which will be used in
place of $R_1$, and $R^p = 1$.

\medskip
4. \textit{Determination of $G$.} $G$ is generated by $H_2$ and some operation
$S_1$.
\begin{equation*}
S{}_1^p = P^{\lambda p}.
\end{equation*}

Denoting $S{}_1^a\, R^b\, Q^c\, P^d \cdots$ by the symbol $[a, b, c, d, \cdots]$
all the operators of $G$ are given by
\begin{equation*}
[v,\, z,\, y,\, x];\, (v,\, z,\, y = 0, 1, \cdots, p - 1; x = 0, 1, \cdots, p^{m-3} - 1).
\end{equation*}

Since $H_2$ is self-conjugate in $G$
\begin{align}
S{}_1^{-1}\, P\, S_1 &= R^\gamma Q^s P^{\epsilon_1}, \\ %% 13
S{}_1^{-1}\, Q\, S_1 &= R^c Q^d P^{ep^{m-4}}, \\ %% 14
S_1\,        R\, S_1 &= R^f Q^g P^{jp^{m-4}}.                  %% 15
\end{align}

From (13), (14), (15), and (12)
\begin{gather*}
[-p,\, 0,\, 0,\, 1,\, p] = [0,\, L,\, M,\, \epsilon{}_1^p + Np^{m-4}] = [0,\, 0,\, 0,\, 1]
\intertext{and}
\epsilon{}_1^p \equiv 1 \pmod{p^{m-4}} \qquad \hbox{ or } \qquad
  \epsilon_1 \equiv 1 \pmod{p^{m-5}} \quad (m > 5).
\end{gather*}
\noindent Let $\epsilon_1 = 1 + \epsilon_2 p^{m-5}$. Equation (13) is
now replaced by
\begin{equation}
S{}_1^{-1}\, P\, S_1 = R^\gamma Q^\delta P^{1 + \epsilon_2 p^{m-5}}. %% 16
\end{equation}

If $\lambda = 0 \pmod{p}$ and $\lambda = \lambda'p,$
\begin{multline*}
[1,\, 0,\, 0,\, 1]^p = \left[p,\, 0,\, 0,\, p + \epsilon\tbinom{p}{2}p^{m-5} + Wp^{m-4}\right] \\
= [0,\, 0,\, 0,\, p + \lambda'p^2 + W'p^{m-4}]
\end{multline*}
\noindent and for $m>5$ $S_1 P$ is of order $p^{m-3}$. We will take this in place
of $S_1$ and assume $dv [\lambda, p] = 1$.
\begin{equation*}
S{}_1^{p^{m-3}} = 1.
\end{equation*}
\noindent There is in $G$ a subgroup $H'_1$ of order $p^{m-2}$ which contains
$\{S_1\}$ self-conjugately. $H'_1 = \{S_1,\, S^v_1\, R^z\, Q^y\, P^x\}$ and the
operator $T= R^z\, Q^y\, P^x$ is in $H'_1$.

There are two cases for discussion.

\smallskip
$1^\circ$. Where $x$ is prime to $p$.

$T$ is an operator of $H_2$ of order $p^{m-3}$ and will be taken as $P$.
Then
\begin{gather*}
H'_1 = \{S_1, P\}.
\intertext{Equation (16) becomes}
S{}_1^{-1}\, P\, S_1 = P^{1 + \epsilon p^{m-4}}.
\end{gather*}
\noindent There is in $G$ a subgroup $H'_2$ of order $p^{m-1}$ which contains $H'_1$
self-conjugately.
\begin{equation*}
H'_2 = \{H'_1,\, S{}_1^{v'}\, R^{z'}\, Q^{y'}\, P^{x'}\}.
\end{equation*}
\noindent $T' = R^{z'}Q^{y'}$ is in $H'_2$ and also in $H_2$ and is taken as $Q$,
since $\{P, T'\}$ is of order $p^{m-2}$.

$H'_2 = \{H'_1, Q\} = \{S_1, H_1\}$ and in this case $c$ may be taken
$\equiv 0 \pmod{p}$.

\smallskip
$2^\circ$. \textit{Where $x = x_1 p$.} $P^p$ is in $\{S_1\}$ since $\lambda$ is
prime to $p$. In the present case $R^z\, Q^y$ is in $H'_1$ and also in $H_2$.
If $z \not\equiv 0 \pmod{p}$ take $R^z\, Q^y$ as $R$; if $z \equiv 0
\pmod{p}$ take it as $Q$.
\begin{gather*}
H'_1 = \{S_1, R\} \quad \hbox{ or } \quad \{S_1,Q\},
\intertext{and}
R^{-1}\, S_1\, R = S{}_1^{1 + k'p^{m-4}} \qquad \hbox{ or } \qquad
Q^{-1}\, S_1\, Q = S{}_1^{1 + k''p^{m-4}}.
\intertext{On rearranging these take the forms}
S{}_1^{-1}\, R\, S_1 = R\,S{}_1^{np^{m-4}} = R\,P^{jp^{m-4}} \quad \hbox{ or }
  \quad S_1^{-1}\, Q\, S_1 = Q\,S{}_1^{n'p^{m-4}} = Q\,P^{ep^{m-4}},
\end{gather*}
\noindent and either $c$ or $g$ may be taken $\equiv 0 \pmod{p}$,
\begin{equation}
cg \equiv 0 \pmod{p}. %% 17
\end{equation}
\noindent From (14), (15), (16), (12) and (17)
\begin{equation*}
[-p,\, 0,\, 1,\, 0,\, p] = \left[0,\, c\frac{d^p - f^p}{d - f},\, d^p,\, Wp^{m-4}\right].
\end{equation*}

Place $x = \lambda p$ and $y = 1$ in (12)
\begin{gather*}
Q^{-1}\, P^{\lambda p}\,Q = P^{\lambda p} \qquad \hbox{ or } \qquad
  S{}_1^p\, Q\, S{}_1^p = Q,
\intertext{and}
d^p \equiv 1 \pmod{p}, \qquad d = 1.
\end{gather*}
\noindent Equation (14) is replaced by
\begin{equation}
S{}_1^{-1}\, Q\, S_1 = R^c\, Q\, P^{ep^{m-4}}. %% 18
\end{equation}

From (15), (18), (17), (16) and (12)
\begin{equation*}
[-p,\, 1,\, 0,\, 0,\, p] = \left[0,\, f^p,\, \frac{d^p - f^p}{d - f}g, W'p^{m-4}\right].
\end{equation*}
\noindent Placing $x = \lambda p,\, y = 1$ in (10)
\begin{equation*}
R^{-1}\, P^{\lambda p}\, R = P^{\lambda p},
\end{equation*}
\noindent and $f^p \equiv 1 \pmod{p},\, f = 1$. Equation (15) is replaced by
\begin{equation}
S{}_1^{-1}\, R\, S_1 = R\, Q^g\, P^{jp^{m-4}}. %% 19
\end{equation}
\noindent From (16), (18), (19) and (12)
\begin{equation*}
S{}_1^{-p}\, P\, S{}_1^p = P^{1 + \epsilon_2 p^{m-4}} = P
\end{equation*}
\noindent and $\epsilon_2 \equiv 0 \pmod{p}$. Let $\epsilon_2 = \epsilon p$ and (16)
is replaced by
\begin{equation}
S{}_1^{-1}\, P\, S_1 = R^\gamma\, Q^\delta\, P^{1 + \epsilon p^{m-4}}. %% 20
\end{equation}
\noindent Transforming both sides of (1), (8) and (9) by $S_1$
\begin{align*}
S{}_1^{-1} Q^{-1} S_1 \cdot S{}_1^{-1} P S_1 \cdot S{}_1^{-1} Q S_1 &=
  S{}_1^{-1} P^{1 + kp^{m-4}} S_1, \\
S{}_1^{-1} R^{-1} S_1 \cdot S{}_1^{-1} P S_1 \cdot S{}_1^{-1} R S_1 &=
  S{}_1^{-1} Q^\beta S_1 \cdot S{}_1^{-1} P^{1 + \alpha p^{m-4}} S_1, \\
S{}_1^{-1} R^{-1} S_1 \cdot S{}_1^{-1} Q S_1 \cdot S{}_1^{-1} R S_1 &=
  S{}_1^{-1} Q S_1 \cdot S{}_1^{-1} P^{ap^{m-4}} S_1.
\end{align*}
\noindent Reducing these by (18), (19), (20) and (12) and rearranging
\begin{align*}
\bigl[0,&\, \gamma,\, \delta + \beta c,\, 1 + \left \{ \epsilon + \alpha c + k + ac\delta +
    a\beta\tbinom{c}{2} - a\gamma \right \} p^{m-4} \bigr] \\
  & \qquad \qquad \qquad = [0,\, \gamma,\, \delta, 1 + (\epsilon + k)p^{m-4}]. \\
[0,&\, \gamma,\, \beta + \delta,\, 1 + \{kg + \epsilon + \alpha + a\delta -
    a\gamma g\}p^{m-4}] \\
  & \qquad \qquad \qquad = \Bigl[0,\, \gamma + \beta c,\, \beta + \delta,\, 1
    + \bigl\{\epsilon + \alpha + \beta e + \alpha\tbinom{\beta}{2}c
    + a\beta\gamma\bigr\}p^{m-4}\Bigr], \\
[0,&\, c,\, 1,\, (e + a)p^{m-4}] = [0,\, c,\, 1,\, (e + a)p^{m-4}].
\end{align*}

The first gives
\begin{gather}
\beta c \equiv 0 \pmod{p}, \\ %% 21
ac + ac\delta - a\gamma \equiv 0 \pmod{p}. %% 22
\intertext{Multiplying this last by $g$}
ag\gamma \equiv 0 \pmod{p}. %% 23
\intertext{From the second equation above}
gk + \alpha\delta \equiv \beta e + a\beta\gamma \pmod{p}. %% 24
\intertext{Multiplying by $c$}
ac\delta \equiv 0 \pmod{p}. %% 25
\end{gather}

These relations among the constants \textit{must be satisfied} in order that our
equations should define a group.

From (20), (19), (18) and (12)
\begin{align}
[-y,\, 0,\, 0,\, x,\, y] &= [0,\, \gamma xy + \chi_1 (x, y),\, \delta xy + \phi_1
  (x, y),\, x + \Theta_1 (x, y)p^{m-4}], \\ %% 26
[-y,\, 0,\, x,\, 0,\, y] &= [0,\, cxy,\, x,\, \Theta_2 (x, y)p^{m-4}], \\ %% 27
[-y,\, x,\, 0,\, 0,\, y] &= [0,\, x,\, gxy,\, \Theta_3 (x, y)p^{m-4}],    %% 28
\end{align}
\noindent where
\begin{align*}
\chi_1 (x,\, y) &= c\delta x\tbinom{y}{2}, \\
\phi_1 (x,\, y) &= \gamma gx\tbinom{y}{2} + \beta\gamma\tbinom{x}{2}y, \\
\Theta_1 (x,\, y) &= \epsilon xy + \tbinom{y}{2}\left[\gamma jx + e\delta x +
    a\delta\gamma + (\alpha\gamma + k\delta)\tbinom{x}{2}\right] \\
  & \qquad \qquad + \tbinom{y}{3} [c\delta j + eg\gamma]x
    + \tbinom{x}{2}[\alpha\gamma y + \delta ky + a\delta\gamma y^2]
    + \beta\gamma k\tbinom{x}{3}y^2, \\
\Theta_2 (x, y) &= exy + cjx\tbinom{y}{2} + ac\tbinom{x}{2}y, \\
\Theta_3 (x, y) &= jxy + egx\tbinom{y}{2} + ag\tbinom{x}{2}y.
\end{align*}

Let a general power of any operator be
\begin{equation}
[v,\, z,\, y,\, x]^s = [sv,\, sz + U_s,\, sy + V_s,\, sx + W_s p^{m-4}]. %% 29
\end{equation}

Multiplying both sides by $[v,\, z,\, y,\, x]$ and reducing by (2), (10), (11),
(26), (27) and (28), we find
\begin{align*}
U_{s+1} &\equiv U_s + (cy + \gamma x)sv + c \delta\tbinom{sv}{2}x \pmod{p}, \\
V_{s+1} &\equiv V_s + (gz + \delta x)sv + \gamma g\tbinom{sv}{2}x
  + \beta\gamma\tbinom{x}{2}sv + \beta (sz + U_s)x \pmod{p}, \\
W_{s+1} &\equiv W_s + \Theta_1 (x, sv) + \left\{ey + jz + a \gamma xy +
    ac\tbinom{y}{2} + ag\tbinom{z}{2} \right\}sv \\
 & \qquad \qquad + \left\{\alpha x + \beta k\tbinom{x}{2}
    + ay + a\delta sx + \alpha gsvz\right\}sz + ksxy \\
 & \qquad \qquad + \tbinom{sv}{2}\{cjy + egz\} + U_s \left\{\alpha x
    + \beta k\tbinom{x}{2} + ay + a(\delta x + gz)sv\right\} \\
 & \qquad \qquad + a\beta\tbinom{sz + Us}{2}x + kV_s x \pmod{p}.
\end{align*}

From (29)
\begin{equation*}
U_1 \equiv 0, \qquad V_1 \equiv 0, \qquad W_1 \equiv 0 \pmod{p}.
\end{equation*}

A continued use of the above congruences give
\begin{align*}
U_s &\equiv (cy + \gamma x)\tbinom{s}{2}v + \frac{1}{2} c\delta xv
      \{\frac{1}{3} (2s - 1)v - 1\}\tbinom{s}{2} \pmod{p}, \\
V_s &\equiv \{[gz + \delta x + \beta\gamma\tbinom{x}{2}v + \beta xz\}
      \tbinom{s}{2} \\ & \qquad + \frac{1}{2} \gamma gxv\{\frac13 (2s - 1)v -1\}
      \tbinom{s}{2} + \beta\gamma\tbinom{s}{3}x^2 v \pmod{p}, \displaybreak \\
%%
W_s &\equiv \tbinom{s}{2}
  \Bigl\{
    \epsilon xv + egv + (\alpha\gamma + \delta kv + \beta kz)\tbinom{s}{2}
    + \beta\gamma\ k\tbinom{x}{3}v + ac\tbinom{y}{2}v \\
    & \qquad + jvz + ag\tbinom{z}{2}v + \alpha xz + kxy + a\gamma xyv + ayz
  \Bigr\}
+ \tbinom{s}{3}
  \Bigl\{
    \alpha cxyv \\ & \qquad + \alpha\gamma x^2 v + 2\beta\gamma k\tbinom{x}{2} xv
    + gkxzv + \delta kx^2 v
    + \beta kx^2 z + acvy^2 \\ & \qquad + a\gamma xvy
  \Bigr\}
+ \beta k \gamma\tbinom{s}{4}x^3 v + \tbinom{s}{2}\frac{2s-1}{3}
  \Bigl\{
    a\delta\gamma\tbinom{x}{2}v^2 + a\delta xzv \\
    & \qquad + agvz^2
  \Bigr\}
+ \frac{1}{2}v\tbinom{s}{2}
  \Bigl\{
    \frac13(2s-1)v - 1
  \Bigr\}
  \Bigl\{
    \gamma jx + e\delta x + a\delta\gamma x \\
    & \qquad + \alpha c \delta\tbinom{x}{2} + \gamma gk \tbinom{x}{2} + cjy + egz
  \Bigr\}
+ \frac{1}{6}\tbinom{s}{2}
  \Bigl\{
    \tbinom{s}{2}v^2 - (2s-1)v \\ & \qquad + 2
  \Bigr\}
  \bigl\{
    c\delta jx + eg\gamma x
  \bigr\}v
+ \frac{1}{2}\tbinom{s}{3}
  \Bigl\{
    \frac{1}{2}(s-1)v-1
  \Bigr\}
  \bigl\{
    \alpha c \delta \\ & \qquad + \gamma gk
  \bigr\} x^2 v
+ \frac12 a\beta x \tbinom{s}{2}
  \Bigl\{
    \frac{1}{3}(2s-1)z - 1
  \Bigr\} z \\
& \qquad + \frac{1}{2} a\delta\gamma x^2 v\tbinom{s}{3} \frac{1}{2}(3s-1) \pmod{p}
\end{align*}

Placing $v = 1,\, z = y = s = p$ in (29)\footnote{For $p = 3$ and
$c\delta \equiv \gamma g \equiv \beta \gamma
\equiv 0 \pmod{p}$ there are terms of the form $(A + Bx + Cx^2 + Dx^3)
p^{m-4}$ in the exponent of $P$. For $m > 5$ these do not vitiate our
conclusion. For $p = 3$ and $c\delta$, $\gamma g$, or $\beta\gamma$ prime
to $p$, $[S_1\, P^x]^p$ is not contained in $\{P\}$ and the groups defined
belong to Class II.}
\begin{gather*}
[S_1\, P^x]^p = S{}_1^p P^{xp} = P^{(\lambda + x)p} \qquad (p > 3).
\intertext{If $x$ be so chosen that}
x + \lambda \equiv 0 \pmod{p^{m-4}}.
\intertext{$S = S_1\, P^x$ is an operator of order $p$ and is taken in place
of $S_1$.}
S^p = 1.
\end{gather*}

The substitution of $S$ for $S_1$ leaves congruence (17) invariant.

\medskip
5. \textit{Transformation of the groups.} All groups of this class are given by

\begin{equation}
G: \begin{cases}
Q^{-1} P\, Q = P^{1 + kp^{m-4}}, \\
R^{-1} P\, R = Q^\beta\, P^{1 + \alpha p^{m-4}}, \\
R^{-1} Q\, R = Q\, P^{ap^{m-4}}, \\
S^{-1} P\, S = R^\gamma\, Q^\delta P^{1 + \epsilon p^{m-4}}, \\
S^{-1} Q\, S = R^c\, Q\, P^{ep^{m-4}}, \\
S^{-1} R\, S = R\, Q^g\, P^{jp^{m-4}}, \\
\end{cases} %% 30
\end{equation}
\noindent with
\begin{equation*}
P^{p^{m-3}} = 1, \quad Q^p = R^p = S^p = 1,
\end{equation*}
\noindent$(k,\, \beta,\, \alpha,\, a,\, \gamma,\, \delta,\, \epsilon,\, c,\,
e,\, g,\, j = 0,\, 1,\, 2,\, \cdots,\, p - 1)$.

These constants are however subject to conditions (17), (21), (22), (23),
(24) and (25). Not all these groups are distinct. Suppose that $G$ and
$G'$ of the above set are simply isomorphic and that the correspondence is
given by

\begin{equation*}
C = \left[ \begin{matrix}S,    & R,    & Q,    & P \\
                         S'_1, & R'_1, & Q'_1, & P'_1 \\ \end{matrix} \right].
\end{equation*}

Inspection of (29) gives
\begin{align*}
S'_1 &= S'^{v'''} R'^{z'''} Q'^{y'''} P'^{x'''p^{m-4}}, \\
R'_1 &= S'^{v''} R'^{z''} Q'^{y''} P'^{x''p^{m-4}}, \\
Q'_1 &= S'^{v'} R'^{z'} Q'^{y'} P'^{x'p^{m-4}}, \\
P'_1 &= S'^v R'^z Q'^y P'^x,
\end{align*}
\noindent in which $x$ and one out of each of the sets $v'$, $z'$, $y'$, $x'$; $v''$,
$z''$, $y''$, $x''$; $v'''$, $z'''$, $y'''$, $x'''$ are prime to $p$.

Since $S$, $R$, $Q$, and $P$ satisfy equations (30), $S'_1$, $R'_1$, $Q'_1$
and $P'_1$ also satisfy them. Substituting these operators and reducing in
terms of $S'$, $R'$, $Q'$, and $P'$ we get the six equations
\begin{equation}
[V'_\kappa,\, Z'_\kappa,\, Y'_\kappa,\, X'_\kappa] = [V_\kappa,\, Z_\kappa,\,
Y_\kappa,\, X_\kappa] \qquad (\kappa = 1,\, 2,\, 3,\, 4,\, 5,\, 6), %% 31
\end{equation}
\noindent which give the following twenty-four congruences
\begin{equation}
  \begin{cases}
  V'_\kappa \equiv V_\kappa \pmod{p}, \\
  Z'_\kappa \equiv Z_\kappa \pmod{p}, \\
  Y'_\kappa \equiv Y_\kappa \pmod{p}, \\
  X'_\kappa \equiv X_\kappa \pmod{p^{m-3}},
  \end{cases} %% 32
\end{equation}
\noindent where
\begin{align*}
V'_1 &= v, \quad V_1 = v, \\
Z'_1 &= Z + c'(yv' - y'v) + \gamma'xv' + c\delta x\tbinom{v'}{2}, \quad Z_1 = z, \\
Y'_1 &= y + g'(zv' - z'v) + \delta'xv' + \gamma'g'x\tbinom{v}{2} + \beta'xz', \quad Y_1 = y, \\
%%
X'_1 &= x + \Bigl\{\epsilon'xv' + (\gamma'j'x + e'\delta'x + a'\delta'\gamma'x)
    \tbinom{v'}{2} + c'\delta'j'\tbinom{v'}{3} + (\alpha'\gamma'v' + \delta'k'v' \\
  &\quad+ a'\delta'\gamma'v^2 + \beta'k'z')\tbinom{x}{2} + j'(zv' - z'v) +
    e'g'[z\tbinom{v'}{2} - z'\tbinom{v}{2}] \\
  &\quad+ a'g'[\tbinom{z}{2}v' + \tbinom{z'}{2}v - zz'v] + e'(yv' - y'v) +
    c'j'[y\tbinom{v'}{2} - y'\tbinom{v}{2}] \\
  &\quad+ a'c'[\tbinom{y}{2}v' + v\tbinom{-y'}{2} - yy'v] + a'(yz' - y'z)
    - a'\beta'xz'^2 + \alpha'xz' \\
  &\quad+ \alpha'\beta'x\tbinom{z'}{2} + a'\gamma'x(y - y')v' + k'xy'\Bigr\}p^{m-4}, \\
%%
X_1 &= x + kxp^{m-4}, \\
V'_2 &= v, \quad V_2 = v + \beta v', \\
Z'_2 &= z + c'(yv'' - y''v) + \gamma'xv'' + e'\delta'\tbinom{v''}{2},
  \quad Z_2 = z + \beta z' + c'\beta y'v, \\
Y'_2 &= y + g'(zv'' - z''v) + \delta'xv'' + \gamma'g'x\tbinom{v''}{2} +
  \beta'\gamma'\tbinom{x}{2}v'' + \beta'xz'', \\
%%
Y_2 &= y + \beta y' + g'\beta z'v, \\
X'_2 &= x + \Bigl\{\Theta'_1(x, v'') + j'(zv'' - z''v) + e'g'[z\tbinom{v''}{2} -
    z''\tbinom{v}{2}] + a'g'[\tbinom{x}{2}v'' \\
 &\quad+ \tbinom{-z''}{2}v - zz''v] + e'(yv'' - y''v) + c'j'[y\tbinom{v''}{2}
    - y''\tbinom{v}{2}] + a'c'[\tbinom{y}{2}v'' \\
 &\quad+ \tbinom{-y''}{2}v - yy''v''] + a'g'(zv'' - z''v)z'' + a'(yz'' - y''z)
    + a'\delta'v''z'' \\
 &\quad+ a'\gamma'(y - y'')v''x + \alpha'xz'' + a'\beta'x\tbinom{z''}{2}
    + \beta'k'\tbinom{x}{2}z'' + k'xy''\Bigr\}p^{m-4}, \\
%%
X_2 &= x + \Bigl\{\alpha x + \beta x' + a'\tbinom{\beta}{2}y'z' + e'\beta vy' +
    (c'j'\beta + e'g'\beta z')\tbinom{v}{2} \\
  &+ a'c'\tbinom{\beta y'}{2}v + j'\beta vz' + a'g'\tbinom{\beta z'}{2}
     + a'\beta(g'z'v + y')z\Bigr\}p^{m-4}, \\
V'_3 &= v', \quad V_3 = v', \\
Z'_3 &= z' + c'(y'v'' - y''v'), \quad Z_3 = z', \\
Y'_3 &= y' + g'(z'v'' - z''v'), \quad Y_3 = y', \\
%%
X'_3 &= \Bigl\{x' + j'(z'v'' - z''v') + e'g'[\tbinom{v''}{2}z' -
    \tbinom{v'}{2}z''] + a'g'[\tbinom{z'}{2}v'' + \tbinom{-z''}{2}v' \\
 &\quad- z'z''v'] + e'(y'v'' - y''v') + c'j'[y'\tbinom{v''}{2} - y''\tbinom{v'}{2}]
    + a'c'[\tbinom{y'}{2}v'' + \tbinom{-y''}{2}v' \\
 &\quad- y''y'v''] + a'(y'z'' - y''z')\Bigr\}p^{m-4}, \\
X_4 &= (x' + a'x)p^{m-4}, \\
%%
V'_4 &= v, \quad V_4 = v + \gamma v'' + \delta v', \\
Z'_4 &= z + c'(yv''' - y'''v) + \gamma'xv''' + c'\delta'x\tbinom{v'''}{2}, \\
Z_4 &= z + \gamma z'' + \delta z' + c'[\tbinom{\gamma}{2}v''y''
    + \tbinom{\delta}{2}v'y'] + c'(\gamma y'' + \delta y')v + c'\gamma\delta y''v, \\
Y'_4 &= y + g'(zv''' - z'''v) + \delta' xv''' + \gamma' g'x\tbinom{v'''}{2}
    + \beta'\gamma'\tbinom{x}{2}v''' + \beta'xz''', \\
%%
Y_4 &= y + \gamma y'' + \delta y' + g'[\tbinom{\gamma}{2}v''z''
    + \tbinom{\delta}{2}v'z'] + g'(\gamma z'' + \delta z')v + g'\delta\gamma v'z'', \\
X'_4 &= x + \Bigl\{\Theta'_1 (x, v''') + j'(zv''' - z'''v) +
      e'g'[\tbinom{v'''}{2}z - \tbinom{v}{2}z'''] + a'g'\left[\tbinom{z}{2}v''' \right. \\
  &\quad \left. + \tbinom{-z'''}{2}v - zz'''v\right] + e'(yv''' - y'''v)
      + c'j'[y\tbinom{v'''}{2} -  y'''\tbinom{v}{2}] \\
  &\quad+ a'c'[\tbinom{y}{2}v''' + \tbinom{-y'''}{2}v - yy'''v''']
      + a'g'(v'''z - vz''')z''' \\
  &\quad+ a'(yz''' - y'''z) + a'\delta' xz'''v''' + a'\gamma' x(y - y''')v''' + \alpha' xz''' \\
  &\quad+ a'\beta' x\tbinom{z'''}{2} + \beta' k'z'''\tbinom{x}{2} + k'xy''' \Bigr\} p^{m-4}. \displaybreak \\
%%
X_4 &= x + \Bigl\{\epsilon x + \delta x' + \gamma x'' + \tbinom{\gamma}{2}
     [a'c'\tbinom{y''}{2}v'' + a'y''z'' + e'v''y'' + j'v''z'' \\
  &\quad+ a'g'\tbinom{z''}{2}v'' + (c'j'v''y'' + e'g'v''z'')(v + \delta v')
     + a'(z + \delta z')v''z'' \\
  &\quad+ \frac{2\gamma - 1}{3}a'g'v''z''^2 + \frac{1}{2}[\frac{1}{3}(2\gamma - 1)v''
     - 1](c'j'y'' + e'g'z'')v''] \\
  &\quad+ \tbinom{\gamma}{3}a'c'v''y'' + \tbinom{\delta}{2}[a'c'\tbinom{y'}{2}v'
     + a'y'z' + e'v'y' + j'v'z' \\
  &\quad+ a'g'\tbinom{z'}{2}v' + j'c'vv'y' + e'g'vv'z' + a'g'v'zz' + a'c'\gamma y'y''v' \\
  &\quad+ \frac{2\delta - 1}{3}a'g'v'z'^2 + \frac{1}{2}\{\frac{1}{3}(2\delta - 1)v'
     - 1\}(c'j'y' + e'g'z')] \\
  &\quad+ \tbinom{\delta}{3}a'c'v'y'^2 + (v + \delta v')[j'\gamma z''
     + \tbinom{\gamma z''}{2}a'g' + e'\gamma y'' + \tbinom{\gamma y''}{2}a'c' \\
  &\quad+ a'g'(z + \delta z')] + \tbinom{v + \delta v'}{2}[e'g'\gamma z''
     + c'j'\gamma y''] + \delta[(e'g'z' \\
  &\quad+ c'j'y')\tbinom{v}{2} + e'vy' + j'z' + a'zy + a'g'vzz' + a'\gamma z'y''
     + a'c'\gamma vy'y''] \\
  &\quad+ a'g'\tbinom{\delta z'}{2}v + a'c'\tbinom{\delta y'}{2} v
     + a'\gamma zy''\Bigr\}p^{m-4}, \\
%%
V'_5 &= v', \quad V_5 = v' + cv'', \\
Z'_5 &= z' + c'(y'v''' - y'''v'), \quad Z_5 = z' + cz'' + c'cy''v, \\
Y'_5 &= y' + g'(z'v''' - z'''v'), \quad Y_5 = y' + cy'' + g'cv'z'', \\
X'_5 &= \Bigl\{x' + j'(z'v''' - z'''v') + e'g'[\tbinom{v'''}{2}z'
     - \tbinom{v'}{2}z'''] + a'g'[\tbinom{z'}{2}v''' \\
  &\quad+ \tbinom{-z'''}{2}v' - z'z'''v'] + c'(y'v''' - y'''v')
     + c'j'[y'\tbinom{\delta'''}{2} - y'''\tbinom{v'}{2}] \\
  &\quad+ a'c'[\tbinom{y'}{2}v''' + \tbinom{-y'''}{2}v' - y'y'''v''']
     + a'(y'z''' - y'''z')\Bigr\}p^{m-4}, \\
%%
X_5 &= \Bigl\{x' + ex + cx'' + a'\tbinom{c}{2}y''z'' + j'cv'z''
     + (e'g'cz'' + c'cj'y'')\tbinom{v'}{2} + e'cy''v \\
  &\quad+ a'cy''z' + a'g'z'v' + a'g'\tbinom{cz''}{2} + a'c'\tbinom{cy''}{2}\Bigr\}p^{m-4}, \\
V'_6 &= v'', \quad V_6 = v'' + gv', \\
Z'_6 &= z'' + c'(y''v''' - y'''v''), \quad Z_6 = z'' + gz', \\
Y'_6 &= y'' + g'(z''v''' - z'''v''), \quad Y_6 = y'' + gy', \\
%%
X'_6 &= \Bigl\{x'' + j'(z''v''' - z'''v'') + e'g'[\tbinom{v'''}{2}z''
     - \tbinom{v''}{2}z'''] + a'g'\left[\tbinom{z''}{2}v''' \right. \\
  &\quad+ \left. \tbinom{-z'''}{2}v'' - z''z'''v''\right] + e'(y''v''' - y'''v'')
     + c'j'[y''\tbinom{v'''}{2} - y'''\tbinom{v''}{2}] \\
  &\quad+ a'c'[\tbinom{y''}{2}v''' + \tbinom{-y'''}{2}v'' - y''y'''v''']
     + a'(y''z''' - y'''z'')\Bigr\}p^{m-4}, \\
X_6 &= \{x'' + jx + gx' + a'gy''z'\}p^{m-4}.
\end{align*}

The necessary and sufficient condition for the simple isomorphism of the
two groups $G$ and $G'$ is \textit{that congruences (32) shall be consistent and
admit of solution} subject to conditions derived below.

\medskip
6. \textit{Conditions of transformation.} Since $Q$ is not contained in $\{P\}$,
$R$ is not contained in $\{Q, P\}$, and $S$ is not contained in $\{R, Q,
P\}$, then $Q'_1$ is not contained in $\{P'_1\}$, $R'_1$ is not contained
in $\{Q'_1, P'_1\}$, and $S'_1$ is not contained in $\{R'_1, Q'_1, P'_1\}$.

Let
\begin{equation*}
{Q'}_1^{s'} = {P'}_1^{sp^{m-4}}.
\end{equation*}

This equation becomes in terms of $S'$, $R'$, $Q'$ and $P'$
\begin{gather*}
[s'v',\, s'z' + c'\tbinom{s'}{2}v'y',\, s'y' + g'\tbinom{s'}{2}v'z',\, Dp^{m-4}]
   = [0,\, 0,\, 0,\, sxp^{m-4}],
\intertext{and}
s'v' \equiv s'z' \equiv s'y' \equiv 0 \pmod{p}.
\end{gather*}

At least one of the three quantities $v'$, $z'$ or $y'$ is prime to $p$,
since otherwise $s'$ may be taken $= 1$.

Let
\begin{equation*}
{R'}_1^{s''} = {Q'}_1^{s'} {P'}_1^{sp^{m-4}},
\end{equation*}
\noindent or in terms of $S'$, $R'$, $Q'$ and $P'$
\begin{multline*}
[s''v'',\, s''z'' + c'\tbinom{s''}{2}v''y'',\, s''y'' + g'\tbinom{s''}{2}v''
    z'',\, Ep^{m-4}] \\ = [s'v',\, s'z' + c'\tbinom{s'}{2}v'y',\, s'y' +
    g'\tbinom{s'}{2}v'z',\, E_1 p^{m-4}],
\end{multline*}
\noindent and
\begin{align*}
s''v'' &\equiv s'v' \pmod{p}, \\
s''z'' + c'\tbinom{s''}{2}v''y'' &\equiv s'z' + c'\tbinom{s'}{2}v'y' \pmod{p}, \\
s''y'' + g'\tbinom{s''}{2}v''z'' &\equiv s'y' + g'\tbinom{s'}{2}v'z' \pmod{p}.
\end{align*}

Since $c'g' \equiv 0 \pmod{p}$, suppose $g' \equiv 0 \pmod{p}$. Elimination
of $s'$ between the last two give by means of the congruence $Z'_3 \equiv
Z_3 \pmod{p}$,
\begin{equation*}
s''\{2(y'z'' - y''z') + c'y'y''(v' - v'')\} \equiv 0 \pmod{p},
\end{equation*}
\noindent between the first two
\begin{equation*}
s''\{2(v'z'' - v''z') + c'v'v''(y' - y'')\} \equiv 0 \pmod{p},
\end{equation*}
\noindent and between the first and last
\begin{equation*}
s''(y'v'' - y''v') \equiv 0 \pmod{p}.
\end{equation*}

At least one of the three above coefficients of $s''$ is prime to $p$,
since otherwise $s''$ may be taken $= 1$.

Let
\begin{equation*}
{S'}_1^{s'''} = {R'}_1^{s''} {Q'}_1^{s'} {P'}_1^{sp^{m-4}}
\end{equation*}
\noindent or, in terms of $S'$, $R'$, $Q'$, and $P'$
\begin{multline*}
[s'''v''', s'''z''' + c'\tbinom{s'''}{2}v'''y''', s'''y''' +
g'\tbinom{s'''}{2}v'''z''', E_2 p^{m-4}] \\ = [s''v'' + s'v', s''z'' + s'z' +
c'\{\tbinom{s''}{2}v''y'' + \tbinom{s'}{2}v'y' + s's''y''v'\}, \\ s''y'' +
s'y' + g'\{\tbinom{s''}{2}v''z'' + \tbinom{s'}{2}v'z' + s's''v'z''\}, E_3 p^{m-4}]
\end{multline*}
\noindent and
\begin{align*}
s'''v''' & \equiv s''v'' + s'v' \pmod{p}, \\
s'''z''' & + c'\tbinom{s'''}{2}v'''y''' \\
  & \qquad \equiv s''z'' + s'z' + c'\{\tbinom{s''}{2}v''y'' + \tbinom{s'}{2}v'y'
    + s's''y''v'\} \pmod{p}, \\
s'''y''' & + g'\tbinom{s'''}{2}v'''z''' \\
  & \qquad \equiv s''y'' + s'y' +  g'\{\tbinom{s''}{2}v''z'' + \tbinom{s'}{2}v'z'
    + s's''z''v'\} \pmod{p}.
\end{align*}

If $g' \equiv 0$ and $c' \not\equiv 0 \pmod{p}$ the congruence $Z'_3
\equiv Z_3 \pmod{p}$ gives
\begin{equation*}
(y'v'' - y''v') \equiv 0 \pmod{p}.
\end{equation*}

Elimination in this case of $s''$ between the first and last congruences
gives
\begin{equation*}
s'''(y''v''' - y'''v'') \equiv 0 \pmod{p}.
\end{equation*}

Elimination of $s''$ between the first and second, and between the second
and third, followed by elimination of $s'$ between the two results, gives
\begin{equation*}
s'''\left(z''^2 - c'y''z''v' + \frac{c'^2}{4}y''v''\right)
(y'v''' - y'''v') \equiv 0 \pmod{p}.
\end{equation*}

Either $(y''v''' - y'''v'')$, or $(y'v''' - y'''v')$ is prime to $p$,
since otherwise $s'''$ may be taken $= 1$.

A similar set of conditions holds for $c' \equiv 0$ and $g' \not\equiv 0
\pmod{p}$.

When $c' \equiv g' \equiv 0 \pmod{p}$ elimination of $s'$ and $s''$ between
the three congruences gives
\begin{equation*}
s'''\Delta \equiv s''' \left|\begin{matrix}
                              v' & v'' & v''' \\
                              y' & y'' & y''' \\
                              z' & z'' & z''' \\ \end{matrix}\right|
\equiv 0 \pmod{p}
\end{equation*}
\noindent and $\Delta$ is prime to $p$, since otherwise $s'''$
may be taken $= 1$.

\newpage
7. \textit{Reduction to types.} In the discussion of congruences (32), the
group $G'$ is taken from the simplest case and we associate with it all
simply isomorphic groups $G$.

\begin{center}
\large I. \normalsize

\smallskip
\begin{tabular}{|r|c|c|c|c|c|c||r|c|c|c|c|c|}
\multicolumn{7}{c}{A.}&\multicolumn{6}{c}{B.} \\ \hline
           &$a_2$&$\beta_2$&$c_2$&$g_2$&$\gamma_2$&$\delta_2$&           &$k_2$&$\alpha_2$&$\epsilon_2$&$e_2$&$j_2$ \\ \hline
\textbf{ 1}&  1  &    1    &  1  &  1  &    1     &     1    &\textbf{ 1}&  1  &     1    &      1     &  1  &  1   \\ \hline
\textbf{ 2}&  0  &    1    &  1  &  1  &    1     &     1    &\textbf{ 2}&  0  &     1    &      1     &  1  &  1   \\ \hline
\textbf{ 3}&  0  &    0    &  1  &  1  &    1     &     1    &\textbf{ 3}&  1  &     0    &      1     &  1  &  1   \\ \hline
\textbf{ 4}&  0  &    0    &  1  &  1  &    1     &     0    &\textbf{ 4}&  1  &     1    &      0     &  1  &  1   \\ \hline
\textbf{ 5}&  0  &    0    &  1  &  0  &    1     &     1    &\textbf{ 5}&  1  &     1    &      1     &  0  &  1   \\ \hline
\textbf{ 6}&  0  &    0    &  1  &  0  &    1     &     0    &\textbf{ 6}&  1  &     1    &      1     &  1  &  0   \\ \hline
\textbf{ 7}&  0  &    1    &  0  &  1  &    1     &     1    &\textbf{ 7}&  0  &     0    &      1     &  1  &  1   \\ \hline
\textbf{ 8}&  0  &    1    &  0  &  1  &    0     &     1    &\textbf{ 8}&  0  &     1    &      0     &  1  &  1   \\ \hline
\textbf{ 9}&  0  &    1    &  1  &  0  &    1     &     1    &\textbf{ 9}&  0  &     1    &      1     &  0  &  1   \\ \hline
\textbf{10}&  0  &    1    &  1  &  0  &    1     &     0    &\textbf{10}&  0  &     1    &      1     &  1  &  0   \\ \hline
\textbf{11}&  1  &    0    &  1  &  1  &    1     &     1    &\textbf{11}&  1  &     0    &      0     &  1  &  1   \\ \hline
\textbf{12}&  1  &    0    &  1  &  0  &    1     &     1    &\textbf{12}&  1  &     0    &      1     &  0  &  1   \\ \hline
\textbf{13}&  1  &    0    &  1  &  1  &    0     &     1    &\textbf{13}&  1  &     0    &      1     &  1  &  0   \\ \hline
\textbf{14}&  1  &    0    &  1  &  1  &    1     &     0    &\textbf{14}&  1  &     1    &      0     &  0  &  1   \\ \hline
\textbf{15}&  1  &    0    &  1  &  0  &    0     &     1    &\textbf{15}&  1  &     1    &      0     &  1  &  0   \\ \hline
\textbf{16}&  1  &    0    &  1  &  0  &    1     &     0    &\textbf{16}&  1  &     1    &      1     &  0  &  0   \\ \hline
\textbf{17}&  1  &    0    &  1  &  1  &    0     &     0    &\textbf{17}&  0  &     0    &      0     &  1  &  1   \\ \hline
\textbf{18}&  1  &    0    &  1  &  0  &    0     &     0    &\textbf{18}&  0  &     0    &      1     &  0  &  1   \\ \hline
\textbf{19}&  1  &    1    &  0  &  1  &    1     &     1    &\textbf{19}&  0  &     0    &      1     &  1  &  0   \\ \hline
\textbf{20}&  1  &    1    &  0  &  1  &    0     &     1    &\textbf{20}&  0  &     1    &      0     &  0  &  1   \\ \hline
\textbf{21}&  1  &    1    &  0  &  1  &    1     &     0    &\textbf{21}&  0  &     1    &      0     &  1  &  0   \\ \hline
\textbf{22}&  1  &    1    &  0  &  1  &    0     &     0    &\textbf{22}&  0  &     1    &      1     &  0  &  0   \\ \hline
\textbf{23}&  1  &    1    &  1  &  0  &    1     &     1    &\textbf{23}&  1  &     0    &      0     &  0  &  1   \\ \hline
\textbf{24}&  1  &    1    &  1  &  1  &    0     &     1    &\textbf{24}&  1  &     0    &      0     &  1  &  0   \\ \hline
\textbf{25}&  1  &    1    &  1  &  1  &    1     &     0    &\textbf{25}&  1  &     0    &      1     &  0  &  0   \\ \hline
\textbf{26}&  1  &    1    &  1  &  0  &    0     &     1    &\textbf{26}&  1  &     1    &      0     &  0  &  0   \\ \hline
\textbf{27}&  1  &    1    &  1  &  0  &    1     &     0    &\textbf{27}&  0  &     0    &      0     &  0  &  1   \\ \hline
\textbf{28}&  1  &    1    &  1  &  1  &    0     &     0    &\textbf{28}&  0  &     0    &      0     &  1  &  0   \\ \hline
\textbf{29}&  1  &    1    &  1  &  0  &    0     &     0    &\textbf{29}&  0  &     0    &      1     &  0  &  0   \\ \hline
           &     &         &     &     &          &          &\textbf{30}&  0  &     1    &      0     &  0  &  0   \\ \hline
           &     &         &     &     &          &          &\textbf{31}&  1  &     0    &      0     &  0  &  0   \\ \hline
           &     &         &     &     &          &          &\textbf{32}&  0  &     0    &      0     &  0  &  0   \\ \hline
\end{tabular}

\newpage
\large II. \normalsize

\smallskip
A.

B.\
\begin{tabular}{|r|c|c|c|c|c|c|c|c|c|c|} \hline
           &1       &2       &3       &4&5     &6     &7     &8     &9     &10     \\ \hline
\textbf{1} &$\times$&$\times$&$\times$& &$19_6$&      &$19_6$&      &$19_6$&       \\ \hline
\textbf{2} &$\times$&$2_1$   &$3_1$   & &      &$19_6$&$19_6$&      &      &$19_6$ \\ \hline
\textbf{3} &$1_2$   &$2_1$   &$3_1$   & &$19_6$&      &      &$19_6$&$19_6$&       \\ \hline
\textbf{4} &$1_2$   &$\times$&$\times$& &$19_6$&      &$19_6$&      &$19_6$&       \\ \hline
\textbf{5} &$2_1$   &$2_1$   &        &*&      &$19_6$&$19_6$&      &$19_6$&       \\ \hline
\textbf{6} &$2_1$   &$2_1$   &$3_1$   & &$19_6$&      &$19_6$&      &$19_6$&       \\ \hline
\textbf{7} &$1_2$   &$2_1$   &$3_1$   & &      &$19_6$&      &$19_6$&      &$19_6$ \\ \hline
\textbf{8} &$1_2$   &$2_4$   &$3_4$   & &      &$19_6$&$19_6$&      &      &$19_6$ \\ \hline
\textbf{9} &$2_1$   &$2_1$   &        &*&$19_6$&$19_6$&$19_6$&      &      &$19_6$ \\ \hline
\textbf{10}&$2_4$   &$2_4$   &$3_4$   & &      &$19_6$&$19_6$&      &      &$19_6$ \\ \hline
\textbf{11}&$1_2$   &$2_4$   &$3_4$   & &$19_6$&      &      &$19_6$&$19_6$&       \\ \hline
\textbf{12}&$2_4$   &$2_4$   &        &*&      &$19_6$&      &$19_6$&$19_6$&       \\ \hline
\textbf{13}&$2_1$   &$2_1$   &$3_1$   & &$19_6$&      &      &$19_6$&$19_6$&       \\ \hline
\textbf{14}&$2_1$   &$2_4$   &        &*&      &$19_6$&$19_6$&      &$19_6$&       \\ \hline
\textbf{15}&$2_1$   &$2_4$   &$3_4$   & &$19_6$&      &$19_6$&      &$19_6$&       \\ \hline
\textbf{16}&$2_1$   &$2_1$   &        &*&      &$19_6$&$19_6$&      &$19_6$&       \\ \hline
\textbf{17}&$1_2$   &$2_4$   &$3_4$   & &      &$19_6$&      &$19_6$&      &$19_6$ \\ \hline
\textbf{18}&$2_4$   &$2_4$   &        &*&$19_6$&$19_6$&      &$19_6$&      &$19_6$ \\ \hline
\textbf{19}&$2_4$   &$2_4$   &$3_4$   & &      &$19_6$&      &$19_6$&      &$19_6$ \\ \hline
\textbf{20}&$2_1$   &$2_4$   &        &*&$19_6$&$19_6$&$19_6$&      &      &$19_6$ \\ \hline
\textbf{21}&$2_4$   &*       &*       & &      &$19_6$&$19_6$&      &      &$19_6$ \\ \hline
\textbf{22}&$2_4$   &$2_4$   &        &*&$19_6$&$19_6$&$19_6$&      &      &$19_6$ \\ \hline
\textbf{23}&$2_4$   &*       &        &*&      &$19_6$&      &$19_6$&$19_6$&       \\ \hline
\textbf{24}&$2_1$   &$2_4$   &$3_4$   & &      &$19_6$&      &      &$19_6$&$19_6$ \\ \hline
\textbf{25}&$2_4$   &$2_4$   &        &*&      &$19_6$&      &$19_6$&$19_6$&       \\ \hline
\textbf{26}&$2_1$   &$2_4$   &        &*&      &$19_6$&$19_6$&      &$19_6$&       \\ \hline
\textbf{27}&$2_4$   &*       &        &*&$19_6$&$19_6$&      &$19_6$&      &$19_6$ \\ \hline
\textbf{28}&$2_4$   &*       &*       & &      &$19_6$&      &$19_6$&      &$19_6$ \\ \hline
\textbf{29}&*       &*       &        &*&$19_6$&$19_6$&      &$19_6$&      &$19_6$ \\ \hline
\textbf{30}&$2_4$   &*       &        &*&$19_6$&$19_6$&$19_6$&      &      &$19_6$ \\ \hline
\textbf{31}&$2_4$   &*       &        &*&      &$19_6$&      &$19_6$&$19_6$&       \\ \hline
\textbf{32}&*       &*       &        &*&$19_6$&$19_6$&      &$19_6$&      &$19_6$ \\ \hline
\end{tabular}

\newpage
\large II. \normalsize (continued)

\smallskip
A.

B.\
\begin{tabular}{|r|c|c|c|c|c|c|c|c|c|c|} \hline
           &11       &12    &13       &14       &15    &16    &17       &18    &19       \\ \hline
\textbf{1} &$\times$ &$19_1$&$\times$ &$11_1$   &$19_1$&$19_1$&$13_1$   &$19_1$&$\times$ \\ \hline
\textbf{2} &$25_2$   &      &$\times$ &$25_2$   &      &      &$13_2$   &      &$\times$ \\ \hline
\textbf{3} &$11_1$   &$19_2$&$13_1$   &$24_2$   &$21_2$&$19_2$&$13_1$   &$21_2$&         \\ \hline
\textbf{4} &$24_2$   &$19_2$&$13_1$   &$24_2$   &$19_1$&$19_2$&$13_1$   &$19_1$&$19_2$   \\ \hline
\textbf{5} &         &      &         &         &      &      &         &      &$19_1$   \\ \hline
\textbf{6} &$\times$ &$19_2$&$\times$ &$11_6$   &$21_2$&$19_2$&$13_6$   &$21_2$&$\times$ \\ \hline
\textbf{7} &$25_2$   &      &$13_2$   &$25_2$   &      &      &$13_2$   &      &         \\ \hline
\textbf{8} &$25_2$   &      &$13_2$   &$25_2$   &      &      &$13_2$   &      &$19_2$   \\ \hline
\textbf{9} &         &$19_6$&         &         &$21_6$&$19_6$&         &$21_6$&$19_2$   \\ \hline
\textbf{10}&$25_{10}$&      &$\times$ &$25_{10}$&      &      &$13_{10}$&      &$19_6$   \\ \hline
\textbf{11}&$24_2$   &$19_2$&$13_1$   &*        &$21_2$&$19_2$&$13_1$   &$21_2$&         \\ \hline
\textbf{12}&         &      &         &         &      &      &         &      &         \\ \hline
\textbf{13}&$11_6$   &*     &$13_6$   &$11_6$   &*     &$19_2$&$13_6$   &*     &         \\ \hline
\textbf{14}&         &      &         &         &      &      &         &      &$19_2$   \\ \hline
\textbf{15}&$11_6$   &$19_2$&$13_6$   &$11_6$   &$21_2$&$19_2$&$13_6$   &$21_2$&$19_6$   \\ \hline
\textbf{16}&         &      &         &         &      &      &         &      &$19_6$   \\ \hline
\textbf{17}&$25_2$   &      &$13_2$   &$25_2$   &      &      &$13_2$   &      &         \\ \hline
\textbf{18}&         &$19_6$&         &         &$21_6$&$19_6$&         &$21_6$&         \\ \hline
\textbf{19}&$25_{10}$&      &$13_{10}$&$25_{10}$&      &      &$13_{10}$&      &         \\ \hline
\textbf{20}&         &$19_6$&         &         &$21_6$&$19_6$&         &$21_6$&$19_2$   \\ \hline
\textbf{21}&$25_{10}$&      &$13_{10}$&$25_{10}$&      &      &$13_{10}$&      &$19_6$   \\ \hline
\textbf{22}&         &$19_6$&         &         &$21_6$&$19_6$&         &$21_6$&$19_6$   \\ \hline
\textbf{23}&         &      &         &         &      &      &         &      &         \\ \hline
\textbf{24}&         &$11_6$&$19_2$   &$13_6$   &$11_6$&*     &*        &$13_6$&*        \\ \hline
\textbf{25}&         &      &         &         &      &      &         &      &         \\ \hline
\textbf{26}&         &      &         &         &      &      &         &      &$19_6$   \\ \hline
\textbf{27}&         &$19_6$&         &         &$21_6$&$19_6$&         &$21_6$&         \\ \hline
\textbf{28}&$25_{10}$&      &$13_{10}$&$25_10$  &      &      &$13_{10}$&      &         \\ \hline
\textbf{29}&         &$19_6$&         &         &$21_6$&$19_6$&         &$21_6$&         \\ \hline
\textbf{30}&         &$19_6$&         &         &$21_6$&$19_6$&         &$21_6$&$19_6$   \\ \hline
\textbf{31}&         &      &         &         &      &      &         &      &         \\ \hline
\textbf{32}&         &$19_6$&         &         &$21_6$&$19_6$&         &$21_6$&         \\ \hline
\end{tabular}

\newpage
\large II. \normalsize (concluded)

\smallskip
A.

B.\
\begin{tabular}{|r|c|c|c|c|c|c|c|c|c|c|} \hline
           &20    &21      &22    &23    &24        &25       &26     &27    &28    &29     \\ \hline
\textbf{1} &$19_1$&$19_1$  &$19_1$&$19_1$&$11_1$    &$11_1$   &$19_1$ &$19_1$&$11_1$&$19_1$ \\ \hline
\textbf{2} &$19_2$&$\times$&$21_2$&      &$\times$  &$\times$ &       &      &$25_2$&       \\ \hline
\textbf{3} &      &        &      &$19_2$&$25_2$    &$24_2$   &$21_2$ &$19_2$&$25_2$&$21_2$ \\ \hline
\textbf{4} &$19_2$&$19_1$  &$19_1$&$19_2$&$11_1$    &$11_1$   &$19_1$ &$19_2$&$11_1$&$19_1$ \\ \hline
\textbf{5} &$19_2$&$19_1$  &$19_1$&$19_6$&$11_6$    &$3_1$    &$21_6$ &$19_6$&$3_1$ &$21_6$ \\ \hline
\textbf{6} &$19_6$&$\times$&$21_6$&$19_1$&$3_1$     &$11_6$   &$19_1$ &$19_2$&$3_1$ &$19_1$ \\ \hline
\textbf{7} &      &        &      &      &$25_2$    &$25_2$   &       &      &*     &       \\ \hline
\textbf{8} &$19_2$&$21_2$  &$21_2$&      &$24_2$    &$25_2$   &       &      &$25_2$&       \\ \hline
\textbf{9} &$19_2$&$21_2$  &$21_2$&      &$11_6$    &$3_1$    &       &      &$3_1$ &       \\ \hline
\textbf{10}&$19_6$&$21_6$  &$21_6$&      &$3_4$     &$\times$ &       &      &$3_4$ &       \\ \hline
\textbf{11}&      &        &      &$19_2$&$25_2$    &$24_2$   &$21_2$ &$19_2$&$25_2$&$21_2$ \\ \hline
\textbf{12}&      &        &      &$19_6$&$25_{10}$ &$3_4$    &$21_6$ &$19_6$&$3_4$ &$21_6$ \\ \hline
\textbf{13}&      &        &      &$19_2$&$3_1$     &$11_6$   &$21_2$ &$19_2$&$3_1$ &$21_2$ \\ \hline
\textbf{14}&*     &$19_1$  &$19_1$&$19_6$&$11_6$    &$3_1$    &$21_6$ &$19_6$&$3_1$ &$21_6$ \\ \hline
\textbf{15}&$19_6$&$21_6$  &$21_6$&$19_2$&$3_1$     &$11_6$   &$19_1$ &*     &$3_1$ &$19_1$ \\ \hline
\textbf{16}&$19_6$&$21_6$  &$21_6$&$19_6$&$3_1$     &$3_1$    &$21_6$ &$19_6$&*     &$21_6$ \\ \hline
\textbf{17}&      &        &      &      &$25_2$    &$25_2$   &       &      &*     &       \\ \hline
\textbf{18}&      &        &      &      &$25_{10}$ &$3_4$    &       &      &$3_4$ &       \\ \hline
\textbf{19}&      &        &      &      &$3_4$     &$25_{10}$&       &      &$3_4$ &       \\ \hline
\textbf{20}&$19_2$&$21_2$  &$21_2$&      &$11_6$    &$3_1$    &       &      &$3_1$ &       \\ \hline
\textbf{21}&$19_6$&$21_6$  &$21_6$&      &$3_4$     &$25_{10}$&       &      &$3_4$ &       \\ \hline
\textbf{22}&$19_6$&$21_6$  &$21_6$&      &$3_4$     &$3_4$    &       &      &*     &       \\ \hline
\textbf{23}&      &        &      &$19_6$&$25_{10}$ &$3_4$    &$21_6$ &$19_6$&$3_4$ &$21_6$ \\ \hline
\textbf{24}&      &        &      &$19_2$&$3_1$     &$11_6$   &$21_2$ &$19_2$&$3_1$ &$21_2$ \\ \hline
\textbf{25}&      &$21_6$  &$21_6$&$19_6$&$3_4$     &$3_4$    &$21_6$ &$19_6$&*     &$21_6$ \\ \hline
\textbf{26}&$19_6$&$21_6$  &$21_6$&$19_6$&$3_1$     &$3_1$    &$21_6$ &$19_6$&*     &$21_6$ \\ \hline
\textbf{27}&      &        &      &      &$25_{10}$ &$3_4$    &       &      &$3_4$ &       \\ \hline
\textbf{28}&      &        &      &      &$3_4$     &$25_{10}$&       &      &$3_4$ &       \\ \hline
\textbf{29}&      &        &      &      &*         &*        &       &      &*     &       \\ \hline
\textbf{30}&$19_6$&$21_6$  &$21_6$&      &$3_4$     &$3_4$    &       &      &*     &       \\ \hline
\textbf{31}&      &        &      &$19_6$&$3_4$     &$3_4$    &$21_6$ &$19_6$&*     &$21_6$ \\ \hline
\textbf{32}&      &        &      &      &*         &*        &       &      &*     &       \\ \hline
\end{tabular}
\end{center}

For convenience the groups are divided into cases.

The double Table I gives all cases consistent with congruences (17), (21),
(23) and (25). The results of the discussion are given in Table II. The
cases in Table II left blank are inconsistent with congruences (22) and
(24), and therefore have no groups corresponding to them.

Let $\kappa = \kappa_1 p^{k_2}$ where $dv[\kappa_1,\, p] = 1\; (\kappa = a,\,
\beta,\, c,\, g,\, \gamma,\, d,\, k,\, \alpha,\, \epsilon,\, e,\, j)$.

In explanation of Table II the groups in cases marked
$\boxed{r_s}$ are simply isomorphic with groups in $A_r B_s$.

The group $G'$ is taken from the cases marked
$\boxed{\times}$. The types are also selected from these cases.

The cases marked $\boxed{*}$ divide into two or more parts. Let
\begin{align*}
a\epsilon - \alpha e + jk &= I_1,    & a\epsilon - jk &= I_2, \\
a\delta(a - e) + 2I_1 &= I_3,        & \alpha g - \beta j &= I_4, \\
\alpha\delta - \beta\epsilon &= I_5, & \epsilon g - \delta j &= I_6, \\
c\epsilon - e\gamma &= I_7,          & \alpha e - jk &= I_8, \\
\delta e + \gamma j &= I_9,          & \alpha\gamma + \delta k &= I_{10}.
\end{align*}

The parts into which these groups divide, and the cases with which they are
simply isomorphic, are given in Table III.

\begin{center}
\large III. \normalsize

\smallskip
\begin{tabular}{|l|l|c|l|c|} \hline
$A_{1,2}B^*$           &$dv[I_1,p]=p$        &$2_1$    &$dv[I_1,p]=1$        &$2_4$  \\ \hline
$A_{3}B^*$             &$dv[I_2,p]=p$        &$3_1$    &$dv[I_2,p]=1$        &$3_4$  \\ \hline
$A_{4}B^*$             &$dv[I_3,p]=p$        &$3_1$    &$dv[I_3,p]=1$        &$3_4$  \\ \hline
$A_{12}B_{13}$         &$dv[I_4,p]=p$        &$19_1$   &$dv[I_4,p]=1$        &$19_2$ \\ \hline
$A_{14}B_{11}$         &$dv[I_5,p]=p$        &$11_1$   &$dv[I_5,p]=1$        &$24_2$ \\ \hline
$A_{15,18}B^*$         &$dv[I_4,p]=p$        &$19_1$   &$dv[I_4,p]=1$        &$21_2$ \\ \hline
$A_{16}B_{24}$         &$dv[I_6,I_5,p]=p$    &$19_1$   &$dv[I_6,I_5,p]=1$    &$19_2$ \\ \hline
$A_{20}B_{14}$         &$dv[I_7,p]=p$        &$19_1$   &$dv[I_7,p]=1$        &$19_2$ \\ \hline
$A_{24,25}B^*$         &$dv[I_8,p]=p$        &$3_1$    &$dv[I_8,p]=1$        &$3_4$  \\ \hline
$A_{27}B_{15}$         &$dv[I_6,p]=p$        &$19_1$   &$dv[I_6,p]=1$        &$19_2$ \\ \hline
$A_{29}B_{7,17}$       &$dv[I_{10},p]=p$     &$24_2$   &$dv[I_{10},p]=1$     &$25_2$ \\ \hline
$A_{29}B_{16,26}$      &$dv[I_9,p]=p$        &$11_6$   &$dv[I_9,p]=1$        &$3_1$  \\ \hline
$A_{29}B_{22,25,30,31}$&$dv[I_9,p]=p$        &$25_{10}$&$dv[I_9,p]=1$        &$3_4$  \\ \hline
$A_{29}B_{29,32}$      &$dv[I_8,I_9,p]=p$    &$11_6$   &$[I_8,p]=p,[I_9,p]=1$&$3_1$  \\ \hline
$A_{29}B_{29,32}$      &$[I_8,p]=1,[I_9,p]=p$&$25_{10}$&$[I_8,p]=1,[I_9,p]=1$&$3_4$  \\ \hline
\end{tabular}
\end{center}

\newpage
8. \textit{Types.} The types for this class are given by equations (30) where the
constants have the values given in Table IV.

\begin{center}
\large IV. \normalsize

\begin{tabular}{|c|c|c|c|c|c|c|c|c|c|c|c|} \hline
       &  $a$   &$\beta$&$c$&$g$&$\gamma$&$\delta$&$k$&$\alpha$&$\epsilon$&$e$&$j$ \\ \hline
$1_1$  &    0   &    0  & 0 & 0 &    0   &    0   & 0 &    0   &     0    & 0 & 0  \\ \hline
$2_1$  &    1   &    0  & 0 & 0 &    0   &    0   & 0 &    0   &     0    & 0 & 0  \\ \hline
$3_1$  &$\kappa$&    1  & 0 & 0 &    0   &    0   & 0 &    0   &     0    & 0 & 0  \\ \hline
$11_1$ &    0   &    1  & 0 & 0 &    0   &    0   & 0 &    0   &     0    & 0 & 0  \\ \hline
$*13_1$&    0   &    1  & 0 & 0 &    1   &    0   & 0 &    0   &     0    & 0 & 0  \\ \hline
$19_1$ &    0   &    0  & 1 & 0 &    0   &    0   & 0 &    0   &     0    & 0 & 0  \\ \hline
$1_2$  &    0   &    0  & 0 & 0 &    0   &    0   & 1 &    0   &     0    & 0 & 0  \\ \hline
$*13_2$&    0   &    1  & 0 & 0 &    1   &    0   & 1 &    0   &     0    & 0 & 0  \\ \hline
$19_2$ &    0   &    0  & 1 & 0 &    0   &    0   & 1 &    0   &     0    & 0 & 0  \\ \hline
$*21_2$&    0   &    0  & 1 & 0 &    0   &    1   & 1 &    0   &     0    & 0 & 0  \\ \hline
$24_2$ &    0   &    0  & 0 & 0 &    1   &    0   & 1 &    0   &     0    & 0 & 0  \\ \hline
$25_2$ &    0   &    0  & 0 & 0 &    0   &    1   & 1 &    0   &     0    & 0 & 0  \\ \hline
$2_4$  &    1   &    0  & 0 & 0 &    0   &    0   & 0 &    0   &     1    & 0 & 0  \\ \hline
$3_4$  &$\kappa$&    1  & 0 & 0 &    0   &    0   & 0 &    0   &     1    & 0 & 0  \\ \hline
$11_6$ &    0   &    1  & 0 & 0 &    0   &    0   & 0 &    0   &     0    & 0 & 1  \\ \hline
$*13_6$&    0   &    1  & 0 & 0 &    1   &    0   & 0 &    0   &     0    & 0 & 1  \\ \hline
\end{tabular}

\footnotesize \noindent $\kappa = 1$, and a non-residue (mod $p$).

\noindent $*$For $p=3$ these groups are isomorphic in Class II.
\end{center}

A detailed analysis of congruences (32) for several cases is given below
as a general illustration of the methods used.

\medskip
\begin{equation*} A_3 B_1. \end{equation*}

The special forms of the congruences for this case are
\begin{gather*}
\beta'xz' \equiv 0 \pmod{p}, \tag{II} \\
a'(yz' - y'z) \equiv kx \pmod{p}, \tag{III} \\
\beta v' \equiv 0, \qquad \beta z' \equiv 0, \qquad \beta y' \equiv
   \beta'xz'' \pmod{p}, \tag*{(IV),(V),(VI)} \\
a'(yz'' - y''z) + a'\beta'x\tbinom{z''}{2} \equiv
%%
\alpha x + \beta x' + a'\beta y'z \pmod{p}, \tag{VII} \\
a'(y'z'' - y''z') \equiv ax \pmod{p}, \tag{X} \\
\gamma v'' + \delta v' \equiv 0 \pmod{p}, \tag{XI} \\
\gamma z'' + \delta z' \equiv 0 \pmod{p}, \tag{XII} \\
%%
\gamma y'' + \delta y' \equiv \beta'xz'' \pmod{p}, \tag{XIII} \\
\begin{split}
  a'(yz''' - y'''z) + a'\beta'x\tbinom{z'''}{2} \equiv \epsilon x &+
  \gamma x'' + \delta x + a'\delta y'z \\ &+ a'\gamma y''z
  + a'\tbinom{\gamma}{2}y''z'' \pmod{p},
\end{split} \tag{XIV} \\
cv'' \equiv 0, \qquad cz'' \equiv 0, \qquad cy'' \equiv 0 \pmod{p},
  \tag*{(XV),(XVI),(XVII)} \\
%%
a'(y'z''' - y'''z') \equiv ex \pmod{p}, \tag{XVIII} \\
gv' \equiv 0, \qquad gz' \equiv 0, \qquad gy' \equiv 0 \pmod{p},
  \tag*{(XIX),(XX),(XXI)} \\
a'(y''z''' - y'''z'') \equiv jx \pmod{p}, \tag{XXII}
\end{gather*}

From (II) $z' \equiv 0 \pmod{p}$.

The conditions of isomorphism give
\begin{equation*}
  \Delta \equiv \left| \begin{matrix}
                          v' & v'' & v''' \\
                          y' & y'' & y''' \\
                          z' & z'' & z''' \\ \end{matrix}\right| \not\equiv 0 \pmod{p}.
\end{equation*}

Multiply (IV), (V), (VI) by $\gamma$ and reduce by (XII), $\beta\gamma v'
\equiv 0$, $\beta\gamma z' \equiv 0$, $\beta\gamma y' \equiv 0 \pmod{p}$. Since
$\Delta \not\equiv 0 \pmod{p}$, one at least of the quantities, $v'$, $z'$
or $y'$ is $\not\equiv 0 \pmod{p}$ and $\beta\gamma \equiv 0 \pmod{p}$.

From (XV), (XVI) and (XVII) $c \equiv 0 \pmod{p}$, and from (XIX), (XX) and
(XXI) $g \equiv 0 \pmod{p}$.

From (IV), (V), (VI) and (X) if $a \equiv 0$, then $\beta \equiv 0$
and if $a \not\equiv 0$, then $\beta \not\equiv 0 \pmod{p}$.

At least one of the three quantities $\beta$, $\gamma$ or $\delta$ is
$\not\equiv 0 \pmod{p}$ and one, at least, of $a$, $e$ or $j$ is $\not
\equiv 0 \pmod{p}$.

\smallskip
$A_3$: Since $z''' \equiv 0 \pmod{p}$, (XVIII) gives $e \equiv 0$.
Elimination between (III), (X), (XIV) and (XXII) gives $a\epsilon - kj
\equiv 0 \pmod{p}$. Elimination between (VI) and (X) gives
$a'\beta'{z''}^2 \equiv a\beta \pmod{p}$ and $a\beta$ is a quadratic
residue or non-residue according as $a'\beta'$ is or is not, and there are
two types for this case.

\smallskip
$A_4$: Since $y'$ and $z''$ are $\not\equiv 0 \pmod{p}$, $e \not\equiv 0
\pmod{p}$. Elimination between (VI), (X), (XIII) and (XVIII) gives $a\delta
- \beta e \equiv 0 \pmod{p}$.

This is a special form of (24).

Elimination between (III), (VII), (X), (XIII), (XIV), (XVIII) and (XXII)
gives
\begin{equation*}
2jk + a\delta(a - e) + 2(a\epsilon - \alpha e) \equiv 0 \pmod{p}.
\end{equation*}

\smallskip
$A_{24}$: Since from (XI), (XII) and (XIII) $y''$ and $z''' \not\equiv 0
\pmod{p}$, and $z'' \equiv v'' \equiv 0 \pmod{p}$, (xxii) gives $j \not
\equiv 0 \pmod{p}$.

Elimination between (III), (X), (XVIII) and (XXII) gives
\begin{equation*}
\alpha e - jk \equiv 0 \pmod{p}.
\end{equation*}

\smallskip
$A_{25}$: (XI), (XII) and (XIII) give $v' \equiv z' \equiv 0$ and $y', z'''
\not\equiv 0 \pmod{p}$ and this with (XVIII) gives $e \not\equiv 0$.

Elimination between (III), (VII), (XVIII) and (XXII) gives
\begin{equation*}
\alpha e - jk \equiv 0 \pmod{p}.
\end{equation*}

\smallskip
$A_{28}$: Since $a \equiv 0$ then $e$ or $j \not\equiv 0 \pmod{p}$.

Elimination between (III), (VII), (XVIII) and (XXII) gives
\begin{equation*}
\alpha e - jk \equiv 0 \pmod{p}.
\end{equation*}

Multiply (XIII) by $a'z'''$ and reduce
\begin{equation*}
\delta e + \gamma j \equiv a'\beta'{z'''}^2 \not\equiv 0 \pmod{p}.
\end{equation*}

\medskip
\begin{equation*} A_{11} B_1. \end{equation*}

The special forms of the congruences for this case are
\begin{gather*}
\beta'xz' \equiv 0 \pmod{p}, \tag{II} \\
kx \equiv 0 \pmod{p}, \tag{III} \\
\beta v' \equiv \beta z' \equiv 0, \quad \beta y' \equiv \beta'xz'',
  \tag*{(IV),(V),(VI)} \\
\alpha x + \beta x' \equiv 0 \pmod{p}, \tag{VII} \\
%%
ax \equiv 0 \pmod{p}, \tag{X} \\
\gamma v'' + \delta v \equiv 0 \pmod{p}, \tag{XI} \\
\gamma z'' + \delta z \equiv 0 \pmod{p}, \tag{XII} \\
\gamma y'' + \delta y \equiv \beta'xz''' \pmod{p}, \tag{XIII} \\
%%
\epsilon x + \gamma x'' + \delta x' \equiv 0 \pmod{p}, \tag{XIV} \\
cv'' \equiv cz'' \equiv cy'' \equiv 0 \pmod{p}, \tag*{(XV),(XVI),(XVII)} \\
ex \equiv 0 \pmod{p}, \tag{XVIII} \\
gv' \equiv gz' \equiv gy' \equiv 0 \pmod{p}, \tag*{(XIX),(XX),(XXI)} \\
jx \equiv 0 \pmod{p}, \tag{XXII}
\end{gather*}

(II) gives $z' = 0$, (III) gives $k \equiv 0$, (X) gives $a \equiv 0$,
(XV), (XVI), (XVII) give $c \equiv 0 (\Delta \not\equiv 0)$, (XVIII) gives
$e \equiv 0$, (XIX), (XX), (XXI) give $g \equiv 0$, (XXII) gives $j \equiv
0$. One of the two quantities $z''$ or $z''' \not\equiv 0 \pmod{p}$,
and by (VI) and (XIII) one of the three quantities $\beta$, $\gamma$ or
$\delta$ is $\not\equiv 0$.

\smallskip
$A_{11}$: (XIV) gives $\epsilon \equiv 0 \pmod{p}$. Multiplying (IV), (V),
(VI) by $\gamma$ gives, by (XII), $\beta\gamma v' \equiv \beta\gamma z'
\equiv \beta\gamma y' \equiv 0 \pmod{p}$, and $\beta\gamma \equiv 0
\pmod{p}$.

\smallskip
$A_{14}$: Elimination between (VII) and (XIV) gives $\alpha\delta -
\beta\epsilon \equiv 0 \pmod{p}$.

\smallskip
$A_{24}$: (VII) gives $\alpha \equiv 0 \pmod{p}$, (XIV) $\epsilon \equiv 0$
or $\not\equiv 0 \pmod{p}$.

\smallskip
$A_{25}$: (VII) gives $\alpha \equiv 0 \pmod{p}$, (XIV) $\epsilon \equiv$
or $\not\equiv 0 \pmod{p}$.

\smallskip
$A_{28}$: (VII) gives $\alpha \equiv 0 \pmod{p}$, (XIV) $\epsilon \equiv$
or $\not\equiv 0 \pmod{p}$.

\newpage
\begin{equation*} A_{19} B_1. \end{equation*}

The special forms of the congruences for this case are
\begin{gather*}
c'(yv' - y'v) \equiv 0 \pmod{p}, \tag{I} \\
kx \equiv 0 \pmod{p}, \tag{III} \\
\beta v \equiv 0, \quad \beta z \equiv c'(yv'' - y''v), \quad \beta y'
\equiv 0 \pmod{p}, \tag*{(IV),(V),(VI)} \\
%%
\alpha x + \beta x' \equiv 0 \pmod{p}, \tag{VII} \\
c'(y'v'' - y''v') \equiv 0 \pmod{p}, \tag{VIII} \\
ax \equiv 0 \pmod{p}, \tag{X} \\
\gamma v'' + \delta v' \equiv 0 \pmod{p}, \tag{XI} \\
%%
\gamma z'' + \delta z' + c'\gamma\delta y''v + c'\tbinom{\delta}{2}v'y' +
c'\tbinom{\gamma}{2}v''y'' \equiv c'(yv''' - y'''v) \pmod {p}, \tag{XII} \\
\gamma y'' + \delta y' \equiv 0 \pmod{p}, \tag{XIII} \\
\epsilon x + \gamma x'' + \delta x' \equiv 0 \pmod{p}, \tag{XIV} \\
%%
cv'' \equiv 0, \quad cz'' \equiv c'(y'v''' - y'''v'), \quad cy'' \equiv 0
  \pmod{p}, \tag*{(XV),(XVI),(XVII)} \\
ex + cx'' \equiv 0 \pmod{p}, \tag{XVIII} \\
gv' \equiv 0, \quad gz' \equiv c'(y''v''' - y'''v''), \quad gy' \equiv 0
  \pmod{p}, \tag*{(XIX),(XX),(XXI)} \\
jx + gx' \equiv 0 \pmod{p}. \tag{XXII} \\
\end{gather*}

(III) gives $k\equiv 0$, (X) gives $a \equiv 0$.

Since $dv[(y'v''' - y'''v'), (y''v''' - y'''v''), p] = 1$ then $dv[c, g, p]
= 1$.

If $c \not\equiv 0$, $v'' \equiv y'' \equiv 0 \pmod{p}$ and therefore $g
\equiv 0 \pmod{p}$ and if $g \not\equiv 0$, then $c \equiv 0 \pmod{p}$.

\smallskip
$A_{12}$: (XVIII) gives $e \equiv 0 \pmod{p}$. Elimination between (VII)
and (XXII) gives $\alpha g - \beta j \equiv 0 \pmod{p}$, (XIV) gives
$\epsilon \equiv 0 \pmod{p}$.

\smallskip
$A_{15}$: (XVIII) gives $e \equiv 0 \pmod{p}$. Elimination between (VII)
and (XXII) gives $\alpha g - \beta j \equiv 0 \pmod{p}$, (XIV) gives
$\epsilon \equiv 0$ or $\not\equiv 0 \pmod{p}$.

\smallskip
$A_{16}$: (XVIII) gives $e \equiv 0$. Elimination between (XIV) and (XXII)
gives $\epsilon g - \delta j \equiv 0 \pmod{p}$, between (VII) and (XIV)
gives $\alpha\delta - \beta\epsilon \equiv 0$.

\smallskip
$A_{18}$: (XVIII) gives $e \equiv 0 \pmod{p}$. Elimination between (VII)
and (XXII) gives $\alpha g - \beta j \equiv 0 \pmod{p}$, (XIV) gives
$\epsilon \equiv 0$ or $\not\equiv 0 \pmod{p}$.

\smallskip
$A_{19}$: (VII) gives $\alpha \equiv 0 \pmod{p}$, (XIV) gives $\epsilon
\equiv 0$, (XXII) gives $j \equiv 0 \pmod{p}$, (XVIII) gives $e \equiv 0$
or $\not\equiv 0 \pmod{p}$.

\smallskip
$A_{20}$: (VII) gives $\alpha \equiv 0$, (XXII) gives $j \equiv 0$.
Elimination between (XIV) and (XVIII) gives $\epsilon c - e\gamma \equiv 0
\pmod{p}$.

\smallskip
$A_{21}$: (VII) gives $\alpha\equiv 0$, (XIV) gives $\epsilon \equiv 0$
or $\not\equiv 0 \pmod{p}$, (XVIII) gives $e \equiv 0,$ or
$\not\equiv 0$, and (XXII) gives $j \equiv 0 \pmod{p}$.

\smallskip
$A_{22}$: (VII) gives $\alpha \equiv 0$, (XIV) gives $\epsilon \equiv 0$
or $\not\equiv 0$, (XVIII) gives $epsilon \equiv 0$ or $\not\equiv 0$,
(XXII) gives $j \equiv 0 \pmod{p}$.

\smallskip
$A_{23}$: (VII) gives $\alpha \equiv 0$, (XIV) gives $\epsilon \equiv 0$,
(XVIII) gives $\epsilon \equiv 0$, (XXII) gives $j \equiv 0$ or
$\not\equiv 0 \pmod{p}$.

\smallskip
$A_{26}$: (VII) $\alpha \equiv 0$, (XIV) $\epsilon \equiv 0$ or
$\not\equiv 0$, (XVIII) $\epsilon \equiv 0$, (XXII) $j \equiv 0$ or
$\not\equiv 0 \pmod{p}$.

\smallskip
$A_{27}$: (VII) $\alpha \equiv 0$, (XIV) $\epsilon \equiv 0$ or
$\not\equiv 0$, (XVIII) $\epsilon \equiv 0$, (XXII) $j \equiv 0$ or
$\not\equiv 0 \pmod{p}$. Elimination between (XIV) and (XXII) gives
$\epsilon g - \delta j \equiv 0 \pmod{p}$.

\smallskip
$A_{29}$: (VII) $\alpha \equiv 0$, (XIV) $\epsilon \equiv 0$ or
$\not\equiv 0$, (XVIII) $\epsilon \equiv 0$, (XXII) $j \equiv 0$ or
$\not\equiv 0 \pmod{p}$.



\newpage
\small
\pagenumbering{gobble}
\begin{verbatim}



End of the Project Gutenberg EBook of Groups of Order p^m Which Contain
Cyclic Subgroups of Order p^(m-3), by Lewis Irving Neikirk

*** END OF THE PROJECT GUTENBERG EBOOK GROUPS OF ORDER P^M ***

This file should be named 9930-t.tex or 9930-t.zip

Produced by Cornell University, Joshua Hutchinson, Lee Chew-Hung,
John Hagerson, and the Online Distributed Proofreading Team.

Project Gutenberg eBooks are often created from several printed
editions, all of which are confirmed as Public Domain in the US
unless a copyright notice is included.  Thus, we usually do not
keep eBooks in compliance with any particular paper edition.

We are now trying to release all our eBooks one year in advance
of the official release dates, leaving time for better editing.
Please be encouraged to tell us about any error or corrections,
even years after the official publication date.

Please note neither this listing nor its contents are final til
midnight of the last day of the month of any such announcement.
The official release date of all Project Gutenberg eBooks is at
Midnight, Central Time, of the last day of the stated month.  A
preliminary version may often be posted for suggestion, comment
and editing by those who wish to do so.

Most people start at our Web sites at:
http://gutenberg.net or
http://promo.net/pg

These Web sites include award-winning information about Project
Gutenberg, including how to donate, how to help produce our new
eBooks, and how to subscribe to our email newsletter (free!).


Those of you who want to download any eBook before announcement
can get to them as follows, and just download by date.  This is
also a good way to get them instantly upon announcement, as the
indexes our cataloguers produce obviously take a while after an
announcement goes out in the Project Gutenberg Newsletter.

http://www.ibiblio.org/gutenberg/etext03 or
ftp://ftp.ibiblio.org/pub/docs/books/gutenberg/etext03

Or /etext02, 01, 00, 99, 98, 97, 96, 95, 94, 93, 92, 92, 91 or 90

Just search by the first five letters of the filename you want,
as it appears in our Newsletters.


Information about Project Gutenberg (one page)

We produce about two million dollars for each hour we work.  The
time it takes us, a rather conservative estimate, is fifty hours
to get any eBook selected, entered, proofread, edited, copyright
searched and analyzed, the copyright letters written, etc.   Our
projected audience is one hundred million readers.  If the value
per text is nominally estimated at one dollar then we produce $2
million dollars per hour in 2002 as we release over 100 new text
files per month:  1240 more eBooks in 2001 for a total of 4000+
We are already on our way to trying for 2000 more eBooks in 2002
If they reach just 1-2% of the world's population then the total
will reach over half a trillion eBooks given away by year's end.

The Goal of Project Gutenberg is to Give Away 1 Trillion eBooks!
This is ten thousand titles each to one hundred million readers,
which is only about 4% of the present number of computer users.

Here is the briefest record of our progress (* means estimated):

eBooks Year Month

    1  1971 July
   10  1991 January
  100  1994 January
 1000  1997 August
 1500  1998 October
 2000  1999 December
 2500  2000 December
 3000  2001 November
 4000  2001 October/November
 6000  2002 December*
 9000  2003 November*
10000  2004 January*


The Project Gutenberg Literary Archive Foundation has been created
to secure a future for Project Gutenberg into the next millennium.

We need your donations more than ever!

As of February, 2002, contributions are being solicited from people
and organizations in: Alabama, Alaska, Arkansas, Connecticut,
Delaware, District of Columbia, Florida, Georgia, Hawaii, Illinois,
Indiana, Iowa, Kansas, Kentucky, Louisiana, Maine, Massachusetts,
Michigan, Mississippi, Missouri, Montana, Nebraska, Nevada, New
Hampshire, New Jersey, New Mexico, New York, North Carolina, Ohio,
Oklahoma, Oregon, Pennsylvania, Rhode Island, South Carolina, South
Dakota, Tennessee, Texas, Utah, Vermont, Virginia, Washington, West
Virginia, Wisconsin, and Wyoming.

We have filed in all 50 states now, but these are the only ones
that have responded.

As the requirements for other states are met, additions to this list
will be made and fund raising will begin in the additional states.
Please feel free to ask to check the status of your state.

In answer to various questions we have received on this:

We are constantly working on finishing the paperwork to legally
request donations in all 50 states.  If your state is not listed and
you would like to know if we have added it since the list you have,
just ask.

While we cannot solicit donations from people in states where we are
not yet registered, we know of no prohibition against accepting
donations from donors in these states who approach us with an offer to
donate.

International donations are accepted, but we don't know ANYTHING about
how to make them tax-deductible, or even if they CAN be made
deductible, and don't have the staff to handle it even if there are
ways.

Donations by check or money order may be sent to:

Project Gutenberg Literary Archive Foundation
PMB 113
1739 University Ave.
Oxford, MS 38655-4109

Contact us if you want to arrange for a wire transfer or payment
method other than by check or money order.

The Project Gutenberg Literary Archive Foundation has been approved by
the US Internal Revenue Service as a 501(c)(3) organization with EIN
[Employee Identification Number] 64-622154.  Donations are
tax-deductible to the maximum extent permitted by law.  As fund-raising
requirements for other states are met, additions to this list will be
made and fund-raising will begin in the additional states.

We need your donations more than ever!

You can get up to date donation information online at:

http://www.gutenberg.net/donation.html


***

If you can't reach Project Gutenberg,
you can always email directly to:

Michael S. Hart <hart@pobox.com>

Prof. Hart will answer or forward your message.

We would prefer to send you information by email.


**The Legal Small Print**


(Three Pages)

***START**THE SMALL PRINT!**FOR PUBLIC DOMAIN EBOOKS**START***
Why is this "Small Print!" statement here? You know: lawyers.
They tell us you might sue us if there is something wrong with
your copy of this eBook, even if you got it for free from
someone other than us, and even if what's wrong is not our
fault. So, among other things, this "Small Print!" statement
disclaims most of our liability to you. It also tells you how
you may distribute copies of this eBook if you want to.

*BEFORE!* YOU USE OR READ THIS EBOOK
By using or reading any part of this PROJECT GUTENBERG-tm
eBook, you indicate that you understand, agree to and accept
this "Small Print!" statement. If you do not, you can receive
a refund of the money (if any) you paid for this eBook by
sending a request within 30 days of receiving it to the person
you got it from. If you received this eBook on a physical
medium (such as a disk), you must return it with your request.

ABOUT PROJECT GUTENBERG-TM EBOOKS
This PROJECT GUTENBERG-tm eBook, like most PROJECT GUTENBERG-tm eBooks,
is a "public domain" work distributed by Professor Michael S. Hart
through the Project Gutenberg Association (the "Project").
Among other things, this means that no one owns a United States copyright
on or for this work, so the Project (and you!) can copy and
distribute it in the United States without permission and
without paying copyright royalties. Special rules, set forth
below, apply if you wish to copy and distribute this eBook
under the "PROJECT GUTENBERG" trademark.

Please do not use the "PROJECT GUTENBERG" trademark to market
any commercial products without permission.

To create these eBooks, the Project expends considerable
efforts to identify, transcribe and proofread public domain
works. Despite these efforts, the Project's eBooks and any
medium they may be on may contain "Defects". Among other
things, Defects may take the form of incomplete, inaccurate or
corrupt data, transcription errors, a copyright or other
intellectual property infringement, a defective or damaged
disk or other eBook medium, a computer virus, or computer
codes that damage or cannot be read by your equipment.

LIMITED WARRANTY; DISCLAIMER OF DAMAGES
But for the "Right of Replacement or Refund" described below,
[1] Michael Hart and the Foundation (and any other party you may
receive this eBook from as a PROJECT GUTENBERG-tm eBook) disclaims
all liability to you for damages, costs and expenses, including
legal fees, and [2] YOU HAVE NO REMEDIES FOR NEGLIGENCE OR
UNDER STRICT LIABILITY, OR FOR BREACH OF WARRANTY OR CONTRACT,
INCLUDING BUT NOT LIMITED TO INDIRECT, CONSEQUENTIAL, PUNITIVE
OR INCIDENTAL DAMAGES, EVEN IF YOU GIVE NOTICE OF THE
POSSIBILITY OF SUCH DAMAGES.

If you discover a Defect in this eBook within 90 days of
receiving it, you can receive a refund of the money (if any)
you paid for it by sending an explanatory note within that
time to the person you received it from. If you received it
on a physical medium, you must return it with your note, and
such person may choose to alternatively give you a replacement
copy. If you received it electronically, such person may
choose to alternatively give you a second opportunity to
receive it electronically.

THIS EBOOK IS OTHERWISE PROVIDED TO YOU "AS-IS". NO OTHER
WARRANTIES OF ANY KIND, EXPRESS OR IMPLIED, ARE MADE TO YOU AS
TO THE EBOOK OR ANY MEDIUM IT MAY BE ON, INCLUDING BUT NOT
LIMITED TO WARRANTIES OF MERCHANTABILITY OR FITNESS FOR A
PARTICULAR PURPOSE.

Some states do not allow disclaimers of implied warranties or
the exclusion or limitation of consequential damages, so the
above disclaimers and exclusions may not apply to you, and you
may have other legal rights.

INDEMNITY
You will indemnify and hold Michael Hart, the Foundation,
and its trustees and agents, and any volunteers associated
with the production and distribution of Project Gutenberg-tm
texts harmless, from all liability, cost and expense, including
legal fees, that arise directly or indirectly from any of the
following that you do or cause:  [1] distribution of this eBook,
[2] alteration, modification, or addition to the eBook,
or [3] any Defect.

DISTRIBUTION UNDER "PROJECT GUTENBERG-tm"
You may distribute copies of this eBook electronically, or by
disk, book or any other medium if you either delete this
"Small Print!" and all other references to Project Gutenberg,
or:

[1]  Only give exact copies of it.  Among other things, this
     requires that you do not remove, alter or modify the
     eBook or this "small print!" statement.  You may however,
     if you wish, distribute this eBook in machine readable
     binary, compressed, mark-up, or proprietary form,
     including any form resulting from conversion by word
     processing or hypertext software, but only so long as
     *EITHER*:

     [*]  The eBook, when displayed, is clearly readable, and
          does *not* contain characters other than those
          intended by the author of the work, although tilde
          (~), asterisk (*) and underline (_) characters may
          be used to convey punctuation intended by the
          author, and additional characters may be used to
          indicate hypertext links; OR

     [*]  The eBook may be readily converted by the reader at
          no expense into plain ASCII, EBCDIC or equivalent
          form by the program that displays the eBook (as is
          the case, for instance, with most word processors);
          OR

     [*]  You provide, or agree to also provide on request at
          no additional cost, fee or expense, a copy of the
          eBook in its original plain ASCII form (or in EBCDIC
          or other equivalent proprietary form).

[2]  Honor the eBook refund and replacement provisions of this
     "Small Print!" statement.

[3]  Pay a trademark license fee to the Foundation of 20% of the
     gross profits you derive calculated using the method you
     already use to calculate your applicable taxes.  If you
     don't derive profits, no royalty is due.  Royalties are
     payable to "Project Gutenberg Literary Archive Foundation"
     the 60 days following each date you prepare (or were
     legally required to prepare) your annual (or equivalent
     periodic) tax return.  Please contact us beforehand to
     let us know your plans and to work out the details.

WHAT IF YOU *WANT* TO SEND MONEY EVEN IF YOU DON'T HAVE TO?
Project Gutenberg is dedicated to increasing the number of
public domain and licensed works that can be freely distributed
in machine readable form.

The Project gratefully accepts contributions of money, time,
public domain materials, or royalty free copyright licenses.
Money should be paid to the:
"Project Gutenberg Literary Archive Foundation."

If you are interested in contributing scanning equipment or
software or other items, please contact Michael Hart at:
hart@pobox.com

[Portions of this eBook's header and trailer may be reprinted only
when distributed free of all fees.  Copyright (C) 2001, 2002 by
Michael S. Hart.  Project Gutenberg is a TradeMark and may not be
used in any sales of Project Gutenberg eBooks or other materials be
they hardware or software or any other related product without
express permission.]

*END THE SMALL PRINT! FOR PUBLIC DOMAIN EBOOKS*Ver.02/11/02*END*

\end{verbatim}
\normalsize

\end{document}
